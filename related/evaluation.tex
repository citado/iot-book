\قسمت{ارزیابی کارایی}

در حوزه اینترنت اشیا پروتکل‌ها و معماری‌های مختلفی وجود دارد که می‌توان از آن‌ها استفاده کرد. هر یک از پروتکل‌ها یا معماری‌ها در شرایط خاصی کارآیی خوبی دارند بنابراین پژوهش‌های زیادی برای ارزیابی کارایی آن‌ها صورت پذیرفته است.
در این ارزیابی پروتکل‌ها و معماری‌ها بدون هیچ تغییر یا بهبودی ارزیابی می‌شوند.
این ارزیابی‌ها به صورت کلی در دو دسته واقعی یا شبیه‌سازی هستند. برخی از آن‌ها در یک لایه به خصوص مانند لایه دسترسی یا لایه هسته فعالیت کرده‌اند و برخی یک راه‌حل انتها به انتها اینترنت اشیا را ارزیابی کرده‌اند.

پارامترهای متنوعی مورد ارزیابی قرار می‌گیرند که از جمله آن‌ها می‌توان توان مصرفی، نرخ داده و جابجای اشیا را را نام برد.
در کنار این پارامترها پژوهش‌هایی در لایه فیزیکی \متن‌لاتین{LoRa} بحث تاثیر تداخل سایر پروتکل‌هایی که از باند \متن‌لاتین{ISM} استفاده می‌کنند، تاثیر پارامترهای منطقه‌ای و محیط عملیاتی، تاثیر پارامتر‌های جوی مانند باران
را بررسی کرده‌اند.
از سوی دیگر پژوهش‌هایی در لایه پیوند داده \متن‌لاتین{LoRaWAN} بحث استفاده از \متن‌لاتین{IPv6}، توزیع ترافیک، ارتباط با شبکه‌های دیگر مانند \متن‌لاتین{WiFi} را بررسی کرده‌اند.
در نهایت برخی از پژوهش‌ها پارامتر مشخصی در حوزه اینترنت اشیا را ارزیابی نکرده بلکه راهکار پیشنهادی در حوزه تخصصی مساله‌شان را مورد ارزیابی قرار داده‌اند، به طور مثال به دقت ارزیابی سرعت
وسایل نقلیه در یک راه کار انتها به انتها شامل سنسورهای متنوع پرداخته‌اند.

\زیرقسمت{مرجع \مرجع{FernandesCarvalho2019}}

\کوچک{تاخیر انتها به انتها}

پژوهش \مرجع{FernandesCarvalho2019} قصد دارد یک روش برای ارزیابی تاخیر در پیاده‌سازی‌های \متن‌لاتین{LoRaWAN} ارائه کند.
قصد این پژوهش ارزیابی یک پیاده‌سازی خاص نیست
اما برای نمایش روش پیشنهادی یک پیاده‌سازی را مورد آزمایش قرار می‌دهد. این روش به صاحبان زیرساخت اجازه می‌دهد بهینه‌سازی و تصمیمات بهتری انجام دهند.
این پژوهش بیان می‌کند که در پژوهش‌های حاضر تعریفی برای کیفیت سرویس و معیارهای آن در \متن‌لاتین{LoRaWAN} عرضه نشده است.

برای ارزیابی تاخیر انتها به انتها نیاز است که زمان ارسال و دریافت بسته در نقاط مختلفی از مسیر آن ثبت شوند.
در استاندارد \متن‌لاتین{LoRaWAN} مدل شبکه‌ای ارائه شده است که راه‌کارهای سازگار با \متن‌لاتین{LoRaWAN} می‌بایست از آن تبعیت کنند.
در این مدل نقاطی برای ثبت زمان دریافت و ارسال بسته وجود دارد
که می‌توان از آن‌ها در همه پیاده‌سازی‌ها و مستقل از جزئیات پیاده‌سازی استفاده کرد.

در این پژوهش اجزای شبکه به عنوان جعبه سیاه در نظر گرفته می‌شوند و تمرکز روی جریان اصلی داده صورت می‌گیرد.
این پژوهش معتقد است که ارسال داده از شی، دغدغه‌ی اصلی بیشتر شبکه‌های \متن‌لاتین{LoRaWAN} بوده و از همین رو
روش پیشنهادی این پژوهش بر روی ارسال داده از شی متمرکز است.
در این روش ترافیک دستگاه انتهایی به شبکه‌ی \متن‌لاتین{LoRaWAN} تزریق شده و از سرور برنامه کاربردی دریافت می‌شود.
خلاصه این روند در شکل \رجوع{شکل: روند روش ارزیابی شبکه LoRaWAN در مرچع FernandesCarvalho2019} آورده شده است.

\شروع{شکل}
\درج‌تصویر[width=\textwidth]{img/e2e-test-methodology.png}
\شرح{روند روش ارزیابی شبکه \متن‌لاتین{LoRaWAN} در مرجع \مرجع{FernandesCarvalho2019}}
\برچسب{شکل: روند روش ارزیابی شبکه LoRaWAN در مرچع FernandesCarvalho2019}
\پایان{شکل}

در این پژوهش برای سادگی تنها از یک سرور شبکه استفاده شده است و سایر سرورها مانند سرور \متن‌لاتین{Join} در نظر گرفته نشده‌اند.
از سوی دیگر تاثیر افزایش تعداد اشیا بر پارامترهای کیفیت سرویس و چگونگی ارسال داده توسط آن‌ها ارزیابی نشده است.
در این ارزیابی زیر جریان‌ها زیر برای یک جریان داده‌ای از شی، در نظر گرفته شده‌اند:
\شروع{شمارش}
\فقره ارسال داده از شی به دروازه که تنها قسمت بی‌سیم از جریان داده‌ای کلی است.
\فقره ارسال داده از دروازه به سرور شبکه یا اختصارا \متن‌لاتین{NS}
\فقره ارسال داده از سرور شبکه به سرور اپلیکشن یا اختصارا \متن‌لاتین{AS}
\فقره ارسال داده از سرور اپلیکیشن به کاربر نهایی
\پایان{شمارش}

در روش پیشنهادی این پژوهش ابتدا نقاط ورودی و خروجی با توجه به جریان‌داده‌ای که پیشنهاد شد مشخص می‌شوند.
در ادامه تنظیمات نقاط زمانی (نقاطی که زمان آن‌ها به بسته اضافه می‌شود) مشخص می‌شوند.
با توجه به ساختار سرور‌ها و گستره جغرافیایی آن‌ها نیاز به روش‌هایی برای همگام‌سازی زمان برای گارانتی نمودن
یکپارچگی اندازه‌گیری‌های زمانی وجود دارد.
در نهایت آنالیز و اندازه‌گیری‌ها صورت می‌گیرد. پارامترهای مورد ارزیابی با توجه به هدف آزمایش می‌تواند متغیر باشد.

\زیرقسمت{مرجع \مرجع{Fujdiak2022}}

\کوچک{ارزیابی استقرار خصوص و عمومی از منظر دسترسی‌پذیری، امنیت و هزینه راه‌اندازی}

پژوهش \مرجع{Fujdiak2022} به مقایسه میان دو پیاده‌سازی مختلف از شبکه‌های \متن‌لاتین{LoRaWAN} می‌پردازد. همانطور که پیشتر اشاره شد شبکه‌های \متن‌لاتین{LoRaWAN} می‌توانند به صورت عمومی یا خصوصی
راه‌اندازی شوند. پژوهش حاضر این دو پیاده‌سازی را از منظرهای کارایی شبکه، هزینه‌ی راه‌اندازی و امنیت ارزیابی می‌کند. این پژوهش مدعی است که اولین کار برای بررسی میان انواع مختلف از استقرارهای شبکه‌های \متن‌لاتین{LoRaWAN}
است.

در استقرار خصوصی، صاحب شبکه کاربر انتهایی بوده و دروازه‌ها عموما برای فراهم آوردن پوشش‌دهی در یک محیط کوچکتر نسبت به استقرار عمومی استفاده می‌شوند.
در این شبکه‌ها، پارامترهای شبکه‌ای برای کاربرای انتهایی قابل تنظیم بوده و محیط کنترل‌شده و شفاف است. در استقرار خصوصی سرمایه بیشتری نسبت به استقرار عمومی لازم است و هزینه
عملیاتی آن با توجه به ابعاد شبکه متغیر است.

در استقرار عمومی شبکه در اختیار یک شخص ثالث است و دروازه‌ها برای فراهم آوردن پوشش‌دهی روی یک منطقه جغرافیایی بزرگ استقرار پیدا می‌کنند.
برای کاربران انتهایی سرمایه اولیه ثابت و کمی لازم است. هزینه راه‌اندازی شبکه با توجه به ابعاد و آبونمان فراهم شده از شبکه، متفاوت است.
در این شبکه محیط کنترل شده نبوده و شفافیتی وجود ندارد.

این پژوهش شبکه‌ی خصوصی را در صحن دانشگاه با استفاده از سرور \متن‌لاتین{LORIOT} که با استفاده از اینترنت قابل دسترس است و تنها یک دروازه شبیه‌سازی می‌کند.
دروازه مورد استفاده از محدوده ارتباطی ۲۵ کیلومتر و نزدیک به ۶۰ هزار گره پشتیبانی می‌کند.

برای شبکه عمومی در این پژوهش از اپراتور ملی جمهوری چک (\متن‌لاتین{CRa}) استفاده می‌شود.
در مجموع ۱۰ دروازه از این اپراتور در محدوده ۱۶ کیلومتری صحن دانشگاه قرار دارند.
با وجود اینکه در جمهوری چک اپراتورهای عمومی زیاد برای \متن‌لاتین{LoRaWAN} وجود دارند، اپراتور ملی انتخاب شده،
تنها اپراتوری است که با استقرار صدها ایستگاه پایه، تمام جمهوری چک را پوشش می‌دهد.

گره انتهایی مورد استفاده در هر دو استقرار، یک گره ثابت با فاکتور گسترش ۱۲ و فعال‌سازی \متن‌لاتین{ABP} است.
با توجه به عدم پشتیبانی اپراتور عمومی، فعال‌سازی \متن‌لاتین{OTAA} و نرخ داده تطبیق‌پذیر در ارزیابی لحاظ نشده‌اند.
محدوده‌ی ارتباطی گره انتهایی ۱۵ کیلومتر است.

ابتدا این پژوهش نتایج ارزیابی کارایی را بیان می‌کند. این ارزیابی‌ها نشان می‌دهند در استقرار خصوصی شبکه می‌توان
با استفاده از پلن‌های فرکانسی شخصی‌سازی شده تداخل را کاهش داد که یک برتری مهم نسبت به استقرار عمومی است.
در ضمن با توجه به ابعاد کوچک این شبکه‌ها، امکان استفاده از پلن‌های فرکانسی متفاوت در قسمت‌های محتلف وجود دارد.

در ادامه ارزیابی کارایی شبکه‌ها در یک محیط بسته صورت پذیرفته است.
در محیط بسته عملکرد دو شبکه عمومی و خصوصی یکسان است و می‌توان از هر دو آن‌ها
در کاربردهایی با دسترسی‌پذیری بیش از ۹۰ درصد،
استفاده کرد. البته قدرت سیگنال
دریافتی از دروازه خصوصی مقداری بیشتر است که می‌توان از آن صرف نظر کرد.

در نهایت این پژوهش بیان می‌کند در شبکه‌های خصوصی می‌توان با استقرار دروازه‌های بیشتر پارامترهای
شبکه را بهبود داد و این امر در شبکه‌های عمومی ممکن نیست.

موضوع بعدی که این پژوهش به آن می‌پردازد امنیت شبکه‌های \متن‌لاتین{LoRaWAN} و تغییرات امنیتی آن‌ها در طی نسخه‌های مختلف است.
این پژوهش بیان می‌کند با توجه به نواقص امنیتی پروتکل \متن‌لاتین{LoRaWAN} به خصوص در نسخه $1.0$، استفاده از شبکه خصوصی امن‌تر
به نظر می‌رسد.

در نهایت این پژوهش هزینه راه‌اندازی و نگهداری را برای شبکه‌های خصوصی و عمومی بررسی می‌کند.
هزینه راه‌اندازی برای شبکه‌های خصوصی با توجه به هزینه تجهیزات و طراحی از شبکه‌های عمومی بیشتر خواهد بود.
هزینه نگهداری در شبکه‌های خصوصی عموما از شبکه‌های عمومی کمتر است البته میزان این هزینه بسیار به اندازه شبکه وابسته است.

\زیرقسمت{مرجع \مرجع{ElChall2019}}

\کوچک{مدل از دست رفت مسیر}

پژوهش \مرجع{ElChall2019} به ارزیابی پروتکل رادیویی \متن‌لاتین{LoRaWAN} در باند فرکانسی ۸۶۸ مگاهرتز می‌پردازد.
این پژوهش ارزیابی‌های گسترده‌ای را در محیط‌های داخل و خارجی در موقعیت‌های شهری و غیرشهری در لبنان صورت می‌دهد
و به وسیله این نتایج عملی، مدل‌های از دست رفت مسیر برای ارتباطات \متن‌لاتین{LoRaWAN} پیشنهاد شده و با مدل‌های مرسوم تجربی مقایسه می‌شوند.
از سوی دیگر این پژوهش کارایی و پوشش استقرار \متن‌لاتین{LoRaWAN} را به وسیله اندازه‌گیری‌های واقعی ارزیابی می‌کند.
در نهایت این پژوهش نشان می‌دهد مدل‌های از دست رفت پیشنهادی دقیق بوده و به سادگی می‌توانند در لبنان و محیط‌های مشابه مورد استفاده قرار گیرند.

این پژوهش بیان می‌کند مدل‌های از دست رفت مسیر فعلی برای شبکه‌های \متن‌لاتین{LoRaWAN} طراحی نشده‌اند و از این رو پژوهش‌های بسیاری دست به ارزیابی‌های عملی
برای بدست آوردن مدل‌های واقعی‌تر زده‌اند. پژوهش حاضر دست به مطالعه عمیق ويژگی‌های مدل‌های انتشار رادیویی زده است و محیط‌های مختلف در لبنان و ارتفاع‌های متفاوت
آنتن در باند فرکانسی ۸۶۸ مگاهرتزی را بررسی کرده است. این پژوهش‌ها نشان می‌دهد شبکه‌ی \متن‌لاتین{LoRaWAN} می‌تواند تا ۸ کیلومتر در محیط‌های شهری و تا ۴۵ کیلومتر
در محیط‌های غیرشهری برد داشته باشد.
در این پژوهش از سرور شبکه‌ای متن باز \متن‌لاتین{chirpstack} استفاده شده است.

دستگاه انتهایی هر ۱۰ ثانیه بسته‌ای شامل مختصات جغرافیایی ارسال می‌کند. این بسته‌ها علاوه بر مختصات شامل شماره ترتیب نیز هستند که از آن می‌تواند از دست رفتن بسته‌ها را تشخیص داد.
این پژوهش مکانیزمی برای کنترل و باز ارسال خودکار در نظر نگرفته است.
فاکتور گسترش برابر ۱۲ قرار گرفته و پهنای باند ۱۲۸ کیلوهرتز بوده است. سه کانال پیش‌فرض از باند فرکانسی ۸۶۸ مگاهرتز مورد استفاده بوده است.

در نهایت این پژوهش مدل‌های از دست رفت مسیر را بدست آمده از آزمایشات عملی را ارائه می‌کند. برای فضای بسته، مدل از دست رفت مسیر ارائه شده، به صورت زیر است:

\[
  PL = 10n\log_{10}(d) + PL_{0} + n_w L_w + n_f^{\frac{n_f+2}{n_f+1}-b}L_f
\]

برای محیط باز، مدل پیشنهادی عبارت است از:

\[
  PL = 10n\log_{10}(d) + PL_0 + L_h\log_{10}(h_{ED}) + X_{\sigma}
\]

که در آن $h_{ED}$ ارتفاع آنتن دستگاه انتهایی و $L_h$ از دست رفتن ناشی از آن است.

\زیرقسمت{مرجع \مرجع{Carvalho2019}}

\کوچک{تاخیر انتها به انتها}

در پژوهش \مرجع{Carvalho2019}، پژوهشگران بیان می‌کنند که در کارهای گذشته به بحث تاخیر تنها در شبکه‌ی بی‌سیم \متن‌لاتین{LoRaWAN} پرداخته شده است و شبکه \متن‌لاتین{IP}
پشتی آن و ارزیابی انتها به انتها مورد توجه نبوده است.
این پژوهش قصد دارد، روشی فرمال در جهت ارزیابی تاخیر در لایه کاربرد ارائه کند.

این پژوهش بیان می‌کند یکی از معیارهای مهم در ارزیابی تاخیر انتها به انتها، زمان مرجع است که با توجه به ماهیت توزیع‌شده معماری این زمان مرجع می‌بایست برای همه عامل‌های
سیستم فراهم شود. زمان مرجع در این ارزیابی، زمان \متن‌لاتین{UTC} است و می‌توان آن را به سادگی از اینترنت به واسطه پروتکل‌های همگام‌سازی چون \متن‌لاتین{NTP}
و یا از طریق سیستم موقعیت‌یاب جهانی (\متن‌لاتین{GPS}) بدست آورد.
عدم قطعیت در پروتکل \متن‌لاتین{NTP} به تاخیر ارتباط اینترنت بسیار وابسته بوده و عموما در مرتبه ده‌ها میلی‌ثانیه تخمین زده می‌شود.
این در حالی است که به وسیله دریافت کننده‌های ارزان قیمت \متن‌لاتین{GPS} می‌توان به دقت‌های زیر میلی‌ثانیه دست پیدا کرد.

این پژوهش برای همگام‌سازی زمانی دریافت کننده و ارسال کننده داده، از یک سرور \متن‌لاتین{NTP} استفاده می‌کند که زمان آن به واسطه یک دریافت‌کننده \متن‌لاتین{GPS} همگام‌سازی شده است.
گره ارسال کننده داده، یک سیستم کامپیوتر شخصی با پردازنده \متن‌لاتین{i3} و سیستم عامل لینوکس است. چهار گره انتهایی با سیستم لینوکس در نظر گرفته شده‌اند که دو عدد از آن‌ها در شبکه پرسرعت دانشگاه قرار داشته،
یک عدد در شبکه شهری با دسترسی \متن‌لاتین{ADSL} قرار داشته و گره آخر در شهر دیگری در فاصله ۷۰ کیلومتری با دسترسی \متن‌لاتین{ADSL} قرار دارد.
این پژوهش عدم قطعیت در اندازه‌گیری تاخیر را برای مبدا و گرههای انتهایی محاسبه کرده و گزارش می‌کند.

آزمایش در طول یک روز و با ۱۴۴۰ پیام صورت پذیرفته است که با نرخ یک پیام بر دقیقه ارسال شده‌اند.
اندازه بسته‌ها برابر ۳۰ بایت بوده و شامل شمارنده و زمان ارسال هستند.
زمان ارسال هر بسته تقریبا برابر ۲۲۶ میلی‌ثانیه است.

نتایج نشان می‌دهد که گره‌های انتهایی اول، دوم و سوم که در شهر محل آزمایش قرار داشتند نتایج نزدیک به هم گزارش کرده‌اند، این در حالی است که گره انتهایی چهارم
که در شهر دیگری قرار داشته است به خاطر وضعیت نامناسب اینترنت تغییرات بیشتری را تجربه کرده است.
تاخیر میانگین برابر با ۵۰۰ میلی‌ثانیه بوده است.
پژوهشگران برای تعداد کمی از بسته‌ها (حدودا ۱۰ بسته در روز) تاخیر نامتعارف حدود ۴ ثانیه را گزارش کرده‌اند که بیان می‌کنند این امر در سرور \متن‌لاتین{MQTT}
رخ نداده و این تاخیرها در سرورهای شبکه و اپلیکشن \متن‌لاتین{LoRaWAN} بوده است.

\زیرقسمت{مرجع \مرجع{Carvalho2018}}

\کوچک{تاخیر انتها به انتها}

در پژوهش \مرجع{Carvalho2018} به ارزیابی تاخیر در شبکه‌های \متن‌لاتین{LoRaWAN} پرداخته و بیان می‌کند تاخیر ناشی از زیرساخت، در کارهای این حوزه مورد توجه قرار نگرفته است
که این پژوهش قصد پرداختن به آن را دارد.
این پژوهش بیان می‌کند زمانی که از پیاده‌سازی مرجع استفاده می‌شود کاری که بر عهده سرور شبکه است یکی از پر مصرف‌ترین پروسه‌ها است.

برای ارزیابی این پژوهش، معماری مرجع \متن‌لاتین{Semtech} را در نظر می‌گیرد. این پژوهش دروازه‌ها را به واسطه شبکه اترنت به سرور شبکه متصل کرده است.
سرور شبکه بر روی ماشین \متن‌لاتین{M1} اجرا میشود. سرور اپلیکیشن روی ماشین \متن‌لاتین{M2} اجرا شده است که به وسیله همان شبکه اترنت به ماشین \متن‌لاتین{M1}
متصل است. هر ماشین‌ها پردازنده \متن‌لاتین{i3} داشته و از سیستم عامل ویندوز استفاده می‌کنند.
این معماری در شکل \رجوع{شکل: معماری ارزیابی تاخیر تاخیر شبکه LoRaWAN در مرچع Carvalho2018}
آورده شده است.

\شروع{شکل}
\درج‌تصویر[width=\textwidth]{img/lorawan-bench-arch-carvalho.png}
\شرح{معماری ارزیابی تاخیر تاخیر شبکه \متن‌لاتین{LoRaWAN} در \مرجع{Carvalho2018}}
\برچسب{شکل: معماری ارزیابی تاخیر تاخیر شبکه LoRaWAN در مرچع Carvalho2018}
\پایان{شکل}

در این پژوهش به جهت ارسال پیام، به واسطه یک لینک \متن‌لاتین{USB} به شی برای ارسال دستور داده می‌شود. این پژوهش خطای زمانی
در این لینک \متن‌لاتین{USB} را در قالب عدم قطعیت ارزیابی کرده است. پیام‌ها هر ۳۰ ثانیه ارسال می‌شوند و از دو طول ۱۸ و ۱۱۳ بایت استفاده می‌شود.
پهنای باند ۱۲۵ کیلوهرتز، فاکتور گسترش برابر ۷ و نرخ کدگذاری $4/5$ است.
هر آزمایش شامل ارسال ۳۶۰ است.

در نهایت این ارزیابی نشان می‌دهد بالاترین میانگین تاخیر مربوط به سرور شبکه است که ادعا این پژوهش را تایید می‌کند.

\زیرقسمت{مرجع \مرجع{Afzal2022}}

\کوچک{ارزیابی پروتکل‌های وب، اکوسیستم \متن‌لاتین{swarm}}

پژوهش \مرجع{Afzal2022} به ارزیابی استفاده از پروتکل‌های دنیای وب در اینترنت اشیا می‌پردازد و اکوسیستم \متن‌لاتین{Swarm} را هدف قرار می‌دهد که در آن اشیا
منابع خود را در قالب سرویس ارائه می‌دهند. این پژوهش از شِماهایی که برای کاهش لود ارتباطی پیشنهاد شده‌اند مانند \متن‌لاتین{SCHC} یا \متن‌لاتین{CBOR}
استفاده کرده و در نهایت نتجیه می‌گیرد استفاده از آن‌ها نسبت به حالت استاندارد، که در دنیای وب مورد استفاده قرار می‌گیرد، ۹۸ درصد کاهش حجم به ارمغان می‌آورد.

این پژوهش بیان می‌کند در پژوهش‌های پیشین در حوزه ارزیابی شِمای \متن‌لاتین{SCHC} بحث قطعه قطعه کردن مورد استفاده و ارزیابی نبوده است.

در اکوسیستم \متن‌لاتین{Swarm} سرویس‌های اینترنت اشیا در قالب مصرف کنندگان و فراهم آورندگان سرویس،
که در واقع عامل‌های نرم‌افزاری در یک اکوسیستم میکروسرویس و در حال تعامل با یکدیگر هستند،
ارائه می‌شود.
اگر دقیق‌تر نگاه کنیم، مصرف کنندگان سرویس که روی دستگاه‌های مصرف کننده اجرا می‌شوند، برای یک کار
به فراهم آورندگان سرویس که روی دستگاه‌های فراهم آورنده هستند، درخواست می‌دهند.
با وجود اینکه مصرف کنندگان سرویس می‌توانند بیش از یک سرویس را مصرف کنند اما فراهم آورندگان سرویس
تنها می‌توانند یک سرویس را فراهم بیاورند.

در معماری پیشنهادی این پروژه، در کنار هر یک از اشیا \متن‌لاتین{LoRaWAN} یک دلال \متن‌لاتین{Swarm} راه‌اندازی شده و به این شکل
این دستگاه‌های انتهایی به شبکه \متن‌لاتین{Swarm} متصل می‌شوند. سرور اپلیشکن در قالب پل ارتباطی میان شبکه \متن‌لاتین{HTTP}
و شبکه \متن‌لاتین{LoRaWAN} ایفای نقش می‌کند.

شبکه‌های \متن‌لاتین{Swarm} به صورت \متن‌لاتین{ReSTFul} بوده و از پروتکل‌های \متن‌لاتین{TCP}، \متن‌لاتین{HTTP} و \متن‌لاتین{IPv4}
استفاده می‌کنند. این پژوهش سربار این پروتکل‌ها را بسیار بیشتر از محدودیت‌های شبکه \متن‌لاتین{LoRaWAN} می‌داند.
در ادامه این پژوهش به بررسی راهکارهای کاهش اندازه بسته‌های می‌پردازد و در این راه ابتدا به پشته پروتکل \متن‌لاتین{UDP}، \متن‌لاتین{CoAP} و \متن‌لاتین{IP}
روی می‌آورد، در ادامه بدنه بسته‌ها را به واسطه \متن‌لاتین{CBOR} کدگذاری می‌کند
و در نهایت از شِمای \متن‌لاتین{SCHC} استفاده می‌کند.
اگر بخواهیم دقیق‌تر بیان کنیم استفاده از \متن‌لاتین{CoAP} باعث کاهش سرآیند بسته‌های \متن‌لاتین{HTTP} می‌شود و \متن‌لاتین{CBOR}
اندازه بدنه بسته‌ها را کاهش می‌دهد.

چالش این پژوهش در استفاده از \متن‌لاتین{SCHC} سازگاری آن با پروتکل \متن‌لاتین{IPv6} و عملکرد اکوسیستم \متن‌لاتین{Swarm}
بر پایه \متن‌لاتین{IPv4} بوده است.
با توجه به اینکه در \متن‌لاتین{SCHC} از \متن‌لاتین{RuleID} برای فشرده‌سازی سرآیندها استفاده می‌شود، این پژوهش دست به آزمایش
استفاده از \متن‌لاتین{SCHC} در پروتکل \متن‌لاتین{IPv4} می‌زند.
همانطور که پیشتر هم بیان شد در این پژوهش نقش سرور اپلیکیشن به عنوان یک پراکسی بوده و تبدیل پروتکل و در صورت نیاز
قطعه‌بندی را صورت می‌دهد.

این پژوهش بیان می‌کند تمامی شیوه‌هایی که برای کاهش اندازه بسته استفاده شده است به فشرده‌سازی سرآیند بسته اقدام می‌کنند
و تنها \متن‌لاتین{CBOR} برای فشرده‌سازی بدنه است.

در انتها ارزیابی‌ها نشان می‌دهد روش‌های فشرده‌سازی استفاده شده می‌توانند برای \متن‌لاتین{uplink} کافی باشند اما
محدودیت‌های \متن‌لاتین{downlink} سخت‌گیرانه‌تر است و نیاز است که روش‌های کاراتری برای آن پیشنهاد شود.

\زیرقسمت{مرجع \مرجع{Potsch2019}}

\کوچک{تاخیر انتها به انتها}

پژوهش \مرجع{Potsch2019} به بررسی تاخیر انتها به انتها در شبکه‌های \متن‌لاتین{LoRaWAN} پرداخته است و برخلاف سایر پژوهش‌ها هدف از این پژوهش
ارزیابی عملی تاخیر و تغییرات تاخیر انتها به انتها، در یک انتقال واقعی داده میان حسگر تا دریافت موفقیت‌آمیز آن به صورت از رمزگشایی شده در برنامه‌ی کاربردی است.

برای ارزیابی عملی از پیاده‌سازی \متن‌لاتین{loraserver.io} استفاده شده است. در این پیاده‌سازی یک جز جدید با نام \متن‌لاتین{Gateway Bridge}
برای ارتباط میان دروازه و سرور شبکه پیاده‌سازی شده است. معماری این ارزیابی در شکل \رجوع{شکل: معماری ارزیابی تاخیر انتها به انتها در شبکه LoRaWAN} آورده شده است.
در این ارزیابی نرخ کدگذاری و پهنای باند به ترتیب بر مقدارهای $4/5$ و ۱۲۵ کیلوهرتز ثابت شده است.
مقدار فاکتور گسترش از میان ۷، ۹ و ۱۲ و اندازه بسته نیز از میان ۸، ۲۲ و ۵۰ (اندازه بسته از ۵۱ بایت محدودیت فاکتور گسترش ۱۲ کمتر است) بایت انتخاب می‌شود.

دو سناریو شبیه‌سازی صورت گرفته است، در سناریو اول سرور اپلکیشن و شبکه \متن‌لاتین{LoRaWAN} هر دو بر روی همان \متن‌لاتین{Raspberry Pi}
قرار گرفته‌اند، که دروازه قرار گرفته است. در سناریو دوم سرور اپلکیشن، شبکه و \متن‌لاتین{MQTT} همگی روی ابر مستقر شده و ارتباط
دروازه به واسطه شبکه‌ی سلولی صورت پذیرفته است.

\شروع{شکل}
\درج‌تصویر[width=\textwidth]{./img/lorawan-e2e-delay.png}
\تنظیم‌ازوسط
\شرح{معماری ارزیابی تاخیر انتها به انتها در شبکه \متن‌لاتین{LoRaWAN} \مرجع{Potsch2019}}
\برچسب{شکل: معماری ارزیابی تاخیر انتها به انتها در شبکه LoRaWAN}
\پایان{شکل}

این پژوهش بیان می‌کند با مهاجرت از شبکه‌ی داخلی به شبکه‌ی اینترنت در سناریو دوم نه تنها تاخیر بلکه انحراف معیار افزایش چشم‌گیری
پیدا می‌کنند. در ادامه این پژوهش بیان می‌کند با افزایش مقدار فاکتور گسترش بر خلاف انتظار پارامترهایی به جز زمان ارسال نیز افزایش پیدا می‌کنند
و این در حالی است که پیش‌بینی می‌شد مقدارها ثابت باقی بمانند و این مساله نیازمند بررسی بیشتر است.

در انتها این پژوهش پروتکل \متن‌لاتین{LoRaWAN} با تاخیر اندازه‌گیری شده برای کاربردهای حساس مناسب نمی‌بیند و بیان می‌کند که می‌توان از آن
برای خودکارسازی پروسه‌هایی که نیازمندی‌های تاخیر حیاتی ندارند استفاده کرد. البته باید در نظر داشت که شبکه‌ی \متن‌لاتین{LoRaWAN} با استفاده از
ارتباط با سیم می‌تواند تاخیرهای زیر ۱۰ و ۶۰ ثانیه را برآورده کند و می‌توان از آن برای اندازه‌گیری هوشمند و ساختمان‌های هوشند استفاده کرد.

\زیرقسمت{مرجع \مرجع{Ferrari2018}}

\کوچک{تاخیر انتها به انتها}

پژوهش \مرجع{Ferrari2018} قصد دارد پروتکل‌های پیام‌رسانی را از نظر تاخیر مقایسه کند.
این پژوهش بیان می‌کند با وجود اهمیت تاخیر در راه‌کارهای صنعتی جهانی اما ارزیابی‌های صورت گرفته بر
پروتکل‌های پیام‌رسانی به خوبی تاخیر را مدنظر قرار نداده‌اند.
این پژوهش یک روش ساخت‌یافته برای ارزیابی عملیاتی تاخیر انتها به انتها یک پروتکل پیام‌رسانی در حالت کلی ارائه می‌دهد که بتوان از آن
در شرایط خاص بهره برد.

یکی از مسائل مهم در ارزیابی‌ها اختلاف زمانی بین مقصد و مبدا داده‌ها است. در این پژوهش زمان \متن‌لاتین{UTC} مرجع قرار داده شده است اما با توجه به نیاز به این پژوهش به صحت زمانی دقیق
برای همگام‌سازی از \متن‌لاتین{NTP} استفاده نشده است بلکه \متن‌لاتین{GPS} برای همگام‌سازی مورد استفاده قرار گرفته است.
به این ترتیب زمان دقیق توسط \متن‌لاتین{GPS} بدست آمده و با دستگاه‌ها به اشتراک گذاشته می‌شود. این اشتراک‌گذاری بهتر است بر پایه پروتکل‌هایی با دقت بالا مانند \متن‌لاتین{PTP} باشد اما
برای کاهش هزینه و پیچیدگی از پروتکل \متن‌لاتین{NTP} استفاده شده است.

ارزیابی به این شکل طراحی شده است که زمان مورد نیاز برای انتقال اطلاعات میان دو گره از طریق دلالی که بر ابر قرار گرفته است، را محاسبه کند.
رویه به این شکل است:

\شروع{شمارش}
\فقره گره ۱ در زمان $T1$ روی موضوع $X$ انتشار صورت می‌دهد.
\فقره گره ۲ (که روی موضوع $X$ مشترک شده است) پیام را در زمان $T2$ دریافت می‌کند.
\فقره گره ۲ به مدت زمان ثابت $K$ صبر کرده و در زمان $T3>T2+K$ روی موضوع $Y$ انتشار صورت می‌دهد.
\فقره گره ۱ (که روی موضوع $Y$ مشترک شده است) پیام را در زمان $T4$ دریافت می‌کند.
\فقره گره ۱ به مدت زمان ثابت $K$ صبر کرده و این رویه را از ابتدا آغاز می‌کند.
\پایان{شمارش}

زمان‌ها در این رویه با استفاده از زمان سیستم که پیشتر در رابطه با همگام‌سازی آن صحبت شد، بدست می‌آیند.
اگر زمان‌های ارسال $T1$ و $T3$ و زمان‌های دریافت $T2$ و $T4$ را مدنظر قرار دهیم تاخیر انتها به انتها به سادگی محاسبه می‌شود:

\[
  ED_{12} = T2 - T1\\
  ED_{21} = T4 - T3
\]

و برای کامل بودن نتایج، زمان تاخیر رفت و برگشت از رابطه زیر محاسبه می‌شود:

\[
  RD_{121} = ED_{12} + ED_{21}
\]

این زمان‌ها، همه شامل زمانی که در سرور دلال پیام سپری می‌شود، نیز هستند و این زمان به صورت جداگانه قابل محاسبه نیست.
از سوی دیگر می‌توان زمان رفت و برگشت را از منظر یک گره محاسبه کرد:

\[
  RD_{1B1} = T_{5} - T_{1}\\
  RD_{2B2} = T_{6} - T_{3}
\]

در این پژوهش دو عدم قطعیت زمانی مطرح می‌شود، یک عدم قطعیت مربوط به زمان ثبت یک رویداد است که نسبت به زمان در آن لحظه دارای عدم قطعیت است و از سوی
دیگر زمان محلی در یک گره همانطور که پیشتر هم به آن پرداخته شد نسبت به زمان استاندارد دارای عدم قطعیت است.

برای پیاده‌سازی این جریان و انجام ارزیابی از \متن‌لاتین{Node-RED} که محصولی شناخته شده از \متن‌لاتین{IBM} برای اتصال سرویس‌های آنلاین، \متن‌لاتین{API}ها و
دستگاه‌های سخت‌افزاری است، استفاده شده است. برای دلال پیام، از سرور‌های مختلف خصوصی و عمومی استفاده شده است.
از سوی دیگر \متن‌لاتین{Node-RED} در گره‌ها نصب بوده و الگوریتم ارزیابی به واسطه آن در این گرهها پیاده‌سازی شده است.
زمان‌های مشخص شده در این الگوریتم در پیام‌ها درج شده و در پایان زمان‌های رفت و برگشت و انتها به انتها با روابط بیان شده محاسبه و ذخیره می‌شوند.
این ارزیابی‌ها به دو تا سه روز زمان برای جمع‌آوری حداقل ۱۰۰۰ نمونه نیاز داشتند و با توجه به محدودیت سرورهای عمومی زمان بیشتر متصور نبوده است.

این پژوهش در ادامه به بررسی عدم قطعیت‌هایی که پیشتر ذکرشان رفت می‌پردازد. این پژوهش نشان می‌دهد که عدم قطعیت نرم‌افزاری به مراتب
بیشتر از عدم قطعیت همگام‌سازی زمانی است. یکی از مهم‌ترین دستاوردهای این پژوهش بررسی تاخیر به صورت بین‌المللی است. در این ارزیابی
نشان داده می‌شود که تاخیر تغییرات زیادی را تجربه می‌کند.

پژوهشگران تمامی کیفیت سرویس‌های پروتکل \متن‌لاتین{MQTT} را در ارزیابی خود لحاظ کرده‌اند. در نهایت کیفیت سرویس پیشنهادی توسط
این پژوهش مقدار ۱ است. از سوی دیگر مستقل از کیفیت سرویس، فاصله تاثیر بیشتری بر تاخیر دارد.

\زیرقسمت{مرجع \مرجع{Liang2020}}

\کوچک{کارایی شبکه دسترسی}

پژوهش \مرجع{Liang2020} ادعا می‌کند، برای اولین بار دست به ارزیابی عملیاتی کارآیی زمانی پروتکل \متن‌لاتین{LoRa} زده است.
در این پژوهش کارایی ارتباطی \متن‌لاتین{LoRa} در یک ساختمان ۱۶ طبقه با استفاده از ۱۱ گره \متن‌لاتین{LoRa}
و با تغییر پارامترهایی مانند قدرت ارسال، نرخ ارتباطی، اندازه بسته و مکان اشیا ارزیابی شده است.

این پژوهش با مقایسه فناوری‌های بی‌سیم \متن‌لاتین{WiFi}، \متن‌لاتین{IEEE 802.15.4}، \متن‌لاتین{Bluetooth}
و \متن‌لاتین{LoRa} که برای ساختمان‌های هوشمند مناسب به نظر می‌رسند، \متن‌لاتین{LoRa} را به عنوان بهترین گزینه انتخاب کرده
و بیان می‌کند با توجه به محیط پیچیده و از دست رفت زیاد سیگنال در ساختمان این فناوری ممکن است با مشکلاتی مواجه شود.

پژوهش حاضر مدعی است که بیشتر کارهای گذشته پارامترهایی مانند نرخ دریافت بسته را در شبکه‌های \متن‌لاتین{LoRa} ارزیابی کرده‌اند و این در حالی است
که تاخیر پارامتر مهمی برای کاربران انتهایی است که نیازمند بررسی دقیق است.

برای ارزیابی از ساختمان بتونی مربوط به اداره دانشگاه \متن‌لاتین{Dalian} در چین استفاده شده است. ماژول ارسال کننده در مرکز ساختمان قرار گرفته و بقیه ماژول‌ها
در قسمت‌های مختلف پخش شده‌اند، هدف از این کار ارزیابی نفود ماژول \متن‌لاتین{LoRa} به دیوارهای سیمانی و بتونی و تاثیر فواصل مختلف بر تاخیر ارسال است.
در این ارزیابی از پروتکل \متن‌لاتین{LoRaWAN} استفاده نشده و ماژول‌ها به صورت مستقیم با یکدیگر در ارتباط هستند.

در آزمایش‌های این پژوهش به توان مصرفی توجهی نشده است. در نهایت این پژوهش نشان می‌دهد پارامترهای زمان ارسال و اندازه بسته نسبت به موقعیت شی و توان ارسالی
تاثیر بیشتری بر تاخیر داشته‌اند. این پژوهش نشان می‌دهد که \متن‌لاتین{LoRa} نفوذ بهتری در دیوارهای سیمانی نسبت به سطوح بتونی دارد.
در نهایت این پژوهش بیان می‌کند در حوزه ارزیابی کارآیی شبکه‌های \متن‌لاتین{LoRa} پژوهش‌های اندکی باند فرکانسی ۴۳۴ مگاهرتز را در نظر گرفته‌اند و این در حالی است
که می‌توان نشان داد این باند فرکانسی نفوذ بهتری نسبت به باند فرکانسی ۸۶۸ مگاهرتز دارد و از این رو در کاربردهای ساختمان‌های هوشمند کاربردی‌تر است.

\زیرقسمت{مرجع \مرجع{Potsch2017}}

\کوچک{سربار استفاده از دروازه‌ها}

پژوهش \مرجع{Potsch2017} به ارزیابی سربار ایجاد شده توسط دروازه‌ها در شبکه‌ی \متن‌لاتین{LoRa}
می‌پردازد که باعث می‌شود حجم داده در شبکه‌ی \متن‌لاتین{Backhaul} شناسایی شود.
این پژوهش بیان می‌کند با توجه به این که شبکه‌های اینترنت اشیا برای حجم داده‌ی کم ساخته شده‌اند سرباری که توسط دروازه‌ها
اضافه می‌شود می‌تواند تاثیرگذار باشد.

راه‌کارهای نرم‌افزاری زیادی \متن‌لاتین{LoRaWAN} را پیاده‌سازی کرده‌اند.
برای ارزیابی این پژوهش از \متن‌لاتین{Chirpstack} استفاده می‌کند.
این پیاده‌سازی معماری را به سه قسمت سرور اپلیکیشن، سرور شبکه و \متن‌لاتین{Gateway Bridge}
می‌شکند.
وظیفه \متن‌لاتین{Gateway Bridge} برقرار ارتباط \متن‌لاتین{UDP} با دروازه و ترجمه پیام‌های آن در قالب
\متن‌لاتین{JSON} و انتشار آن بر بستر \متن‌لاتین{MQTT} است.

این پژوهش بیان می‌کند در دروازه پیام‌ها به همراه اطلاعات اضافه به صورت بسته‌هایی بر بستر \متن‌لاتین{UDP} ارسال می‌شوند.
در ارزیابی حجم این سر بار برای هر پیام ۱۹۰ بایت اندازه‌گیری شده است و در نهایت برای ارسال بسته‌ای ۱۰ بایتی هر ۵ دقیقه از یک گره نیاز به
حجم داده‌ی نزدیک به ۲ مگابایت در ماه است. این مقدار بدون احتساب پیام‌های \متن‌لاتین{heartbeat} و آماری است که توسط دروازه
به صورت دوره‌ای ارسال می‌شود.

\زیرقسمت{مرجع \مرجع{Moura2021}}

\کوچک{ارزیابی سیستم اینترنت اشیا، ساختمان هوشمند}

در پژوهش \مرجع{Moura2021} یک پلتفرم اینترنت اشیا با زیرساخت شبکه سنسور بی‌سیم برای کنترل گرمایش، تهویه مطبوع و سیستم‌های روشنایی در چهار ساختمان دانشکده
مهندسی مخابرات دانشگاه میلان پیاده‌سازی کرده است. این مقاله استقرار یک پلتفرم اینتنرت اشیا و پیاده‌سازی سرویس‌ها را ارائه می‌دهد.

این پژوهش بیان می‌کند برای کاهش انرژی مورد نیاز و استفاده از انرژی‌های تجدیدپذیر و از سوی دیگر فراهم آوردن سرویس‌های انرژی در قسمت‌های مختلف ساختمان با توجه به فعالیت
کاربران و شرایط آب هوایی پلتفرم نظارت و کنترل نقش حیاتی دارد. این پلتفرم نه تنها باید بتواند بر نیاز کلی انرژی و تولید آن نظارت کند بلکه باید بتواند بر مصرف قسمت‌های مختلف ساختمان
و سیستم‌های الکتریکی نیز نظارت کند و در کنار آن بر اطلاعات مرتبطی مانند اطلاعات آب و هوایی و حضور کاربران برای کنترل بار نظارت داشته باشد.
گزینه‌هایی برای پلتفرم‌های نظارت و کنترل در بازار وجود دارد اما در بیشتر موارد استفاده از در آن در یک ساختمان قدیمی و ساخته شده نه تنها آسان نبوده بلکه هزینه ابتدایی آن به قدری زیاد است
که بهبود اقتصادی ناشی از ناشی از نظارت و کنترل را کاهش می‌دهد.

پژوهش حاضر به عنوان منبع تجدید‌پذیر انرژی از فناوری فتوولتائیک استفاده می‌کند که با توجه به موقعیت و نوع ساختمان بهترین گزینه از نظر کارایی و هزینه است.
البته این پژوهش بیان می‌کند سیستم طراحی شده از مجموع انرژی تولیدی استفاده می‌کند و بنابراین امکان استفاده از فناوری‌های دیگر نیز وجود دارد.

در این پژوهش برای جایگذاری پنل‌های خورشیدی از نرم‌افزار \متن‌لاتین{SunnyDesign} استفاده شده است. برای پیش‌بینی انرژی تولیدی مورد نیاز از داده‌های مصرف
انرژی یک سال استفاده شده است که البته پژوهش ادعا می‌کند با توجه به تغیرات کم در سیستم‌ها این داده کفایت می‌کند. شبیه‌سازی انرژی خورشیدی تولید شده نیز
در بازه یکسال مورد شبیه‌سازی قرار گرفته است که عموما با توجه به تغییرات کم تابش‌های خورشیدی کفایت می‌کند.

این پژوهش نیازمندی‌های پلتفرم اینترنت اشیا را چنین برمی‌شمارد:

\شروع{فقرات}
\فقره \متن‌سیاه{هزینه پیاده‌سازی پایین}. به خاطر استفاده در یک ساختمان قدیمی راه‌کار می‌بایست از نظر هزینه بهینه بوده و کمترین تاثیر را در زمان نصب و راه‌اندازی
بر سیستم‌های حاضر داشته باشد.
\فقره \متن‌سیاه{نظارت و پشتیانی از کنترل سیستم‌های الکتریکی مختلف شامل نور و تهویه مطبوع}. سیستم می‌بایست بتواند دستگاه‌های تنها
و زیر سیستم‌های کامل را برای حفظ پویایی ریزدانگی، نظارت و کنترل کند.
\فقره \متن‌سیاه{اتوماسیون کنترل سیستم‌های الکتریکی با دخالت انسانی پایین}. سیستم می‌بایست از تعریف رفتارهای از پیش تنظیم شده و در عین حال دخالت‌های دستی
پشتیبانی کند.
\فقره \متن‌سیاه{ارزیابی مصرف انرژی}. سیستم می‌بایست توان و مصرف انرژی سیستم‌های مختلف تحت نظارت را گزارش کند.
\فقره \متن‌سیاه{ادغام با زیرساخت فعلی در قالب ارتباطات و انرژی}. پلتفرم اینترنت اشیایی که قرار است مستقر شود، نباید محدود به هیچ فناوری شبکه یا سرمایه‌گذاری
جدید در استقرار زیرساخت انرژی باشد بلکه می‌بایست با زیرساخت فعلی مانند \متن‌لاتین{WiFi} و \متن‌لاتین{Differential Switch} ادغام شود.
\پایان{فقرات}

پلتفرم اینترنت اشیا برای نظارت بر جریان جمعیت از \متن‌لاتین{WiFi} استفاده می‌کند. سنسورهایی به این منظور به صورت دوره‌ای آدرس‌های \متن‌لاتین{MAC} متصل به
\متن‌لاتین{WiFi} را در باندهای فرکانسی $2.4$ و $5$ گیگاهرتز که در محدوده پوشش‌دهی آن سنسور قرار دارند، جمع‌آوری می‌کنند.
برای نظارت‌های محیطی از سنسورهای \متن‌لاتین{Smart Citizen Kit} یا اختصارا \متن‌لاتین{SCK} استفاده شده است که بر پایه \متن‌لاتین{Arduino} بوده
و شامل یک \متن‌لاتین{shield} برای سنسورهای دما، رطوبت، نور، سطح نویز و کیفیت هوا است. این سنسورها می‌توانند داخل یا خارج از ساختمان استفاده شوند.
هر دو این سنسورها با استفاده از شبکه‌ی \متن‌لاتین{WiFi} کار می‌کنند و سرور برای ارتباط با آن‌ها از \متن‌لاتین{MQTT} استفاده می‌کند.

برای عملگرها، از عملگرهای شرکت \متن‌لاتین{Sonoff} استفاده شده که دارای چیپ \متن‌لاتین{ESP8266} بوده و از ارتباط \متن‌لاتین{WiFi} پشتیبانی می‌کنند.
سه نوع از این عملگرها وجود دارد، نوع اول یک سوئیچ روشن و خاموش ساده است، نوع دوم سوئیچ روشن و خاموشی است که توان مصرفی را نیز اندازه‌گیری می‌کند و
نوع سوم با قابلیت قرار گیری در رک‌های تجهیزات دارای چهار کانال کنترلی از یک ورودی برقی بوده و برای اندازه‌گیری توان مصرفی جریان‌های برقی اصلی است.
برای مدیریت این عملگرها در سرور محلی \متن‌لاتین{Firmware} این عملگرها تغییر پیدا کرده است.

این پژوهش بیان می‌کند انتخاب \متن‌لاتین{WiFi} به علت پوشش کامل ساختمان‌ها توسط این شبکه بوده است.
در نهایت پلتفرم متن‌باز \متن‌لاتین{Home Assistant} مورد استفاده قرار گرفته است و به همراه دلال پیام \متن‌لاتین{Mosquitto} روی یک برد \متن‌لاتین{RPi4}
نصب شده‌اند.

بر پایه این زیرساخت سرویس‌های جدید انرژی توسعه پیدا کرده‌اند. سرویس نظارت بر وضعیت، وضعیت روشن یا خاموش هر سیستم الکتریکی روشنایی و تهویه مطبوع را
گزارش می‌کند. سرویس نظارت بر درخواست، مصرف انرژی هر سیستم الکتریکی را به صورت لحظه‌ای و تاریخی گزارش می‌کند. سرویس‌های کنترل دستی و خودکار که به ترتیب
اجازه روشن و خاموش کردن سیستم‌ها از دور و با یک سیاست از پیش برنامه‌ریزی شده یا سایر پارامترها را می‌دهند.

\زیرقسمت{مرجع \مرجع{Almojamed2021}}

\کوچک{کارایی شبکه دسترسی، جابجایی اشیا، \متن‌لاتین{FLoRa}}

پژوهش \مرجع{Almojamed2021} به دنبال بررسی جابجایی در شبکه‌های \متن‌لاتین{LoRaWAN} است و از همین رو با استفاده از شبیه‌سازی با نرم‌افزار \متن‌لاتین{OMNET++}
دو مدل جابجایی شناخته شده را برای تاثیر جابجایی بر کارایی شبکه‌ی \متن‌لاتین{LoRaWAN} مورد تحقیق قرار می‌دهد. این پژوهش بیان می‌کند در نظر گرفتن جابجایی در شبکه‌های
\متن‌لاتین{LoRaWAN} ایده‌ی جدیدی نیست اما پژوهش‌های حاضر مدل‌های جابجایی کمی را در نظر گرفته‌اند و سرعت و تعداد اشیای آن‌ها محدود بوده است.
علاوه بر این، برخی از این پژوهش‌ها بر روی جابجایی از منظر فراگرد بین شبکه‌های مختلف تمرکز کرده‌اند یا جابجایی را تنها برای بخشی از اشیا در نظر گرفته‌اند.

نوآوری‌های این پژوهش شامل مواردی است که در ادامه آورده شده است:

\شروع{فقرات}
\فقره ادغام کردن \متن‌لاتین{Framework for LoRa} یا مختصرا \متن‌لاتین{FLoRa} با مدل‌های جابجایی مختلف در شبیه‌ساز \متن‌لاتین{OMNet++}
\فقره ارزیابی کارایی \متن‌لاتین{LoRaWAN} بر پایه مدل‌های جابجایی مختلف (مدل‌های ثابت، گاوس--مارکو و \متن‌لاتین{RWP})
\فقره انجام یک ارزیابی جامع روی سه مدل جابجایی با در نظر گرفتن شبکه‌ای با بیش از ۵۰۰۰ دستگاه انتهایی که با سرعت‌های مختلف (تا ۲۵ متر بر ثانیه) جابجا می‌شوند و در عین حال از تعداد
دروازه‌های مختلفی استفاده می‌کنند.
\فقره در نظر گرفتن تاثیر اندازه داده بر کارایی شبکه \متن‌لاتین{LoRaWAN} با جابجایی
\پایان{فقرات}

مدل جابجایی \متن‌لاتین{RWP} یا \متن‌لاتین{Random Waypoint} یکی از سه مدل جابجایی است که تحت عنوان جابجایی تصادفی طبقه‌بندی می‌شوند.
دو مدل دیگر مدل‌های \متن‌لاتین{Random Walk} و \متن‌لاتین{Random Direction} هستند.
از \متن‌لاتین{RWP} به صورت گسترده در شبیه‌سازی شبکه‌های تک کاره استفاده می‌شود.
\متن‌لاتین{RWP} به سه پارامتر محیط شبیه‌سازی، سرعت و زمان توقف نیاز دارد.
بر پایه این مدل، یک گره یک موقعیت تصادفی در محیط شبیه‌سازی و یک سرعت تصادفی را انتخاب می‌کند.
گره در ادامه به سوی آن موقعیت با سرعت انتخابی حرکت می‌کند. هنگامی که گره به آن موقعیت می‌رسد برای مدت زمانی که تحت عنوان زمان توقف
شناخته می‌شود صبر کرده و دوباره یک موقعیت و سرعت تصادفی انتخاب نموده و همین روند را ادامه می‌دهد. این رویه توسط تک تک گرهها تا انتهای
شبیه‌سازی صورت می‌پذریرد.

مدل جابجایی گاوس--مارکو متعلق به دسته مدل‌های جابجایی وابسته زمانی است که از داده‌های تاریخی جهت و سرعت گره برای محاسبه
مقدارهای جدید سرعت و جهت آن استفاده می‌نماید. مقدارهای تاریخی برای جلوگیری از تغییرات ناخواسته در جهت و سرعت مورد استفاده قرار می‌گیرد.
گرهها به صورت تصادفی در محیط شبیه‌سازی قرار می‌گیرند و به آن‌ها سرعت و جهت اولیه‌ای تخصیص پیدا می‌کند.
این مقدارهای اولیه برای محاسبه مقدارهای جدید مورد استفاده قرار می‌گیرد.
مدل سرعت و جهت جدید را در بازه‌های زمانی از پیش تعریف شده محاسبه می‌کند و با استفاده از \متن‌لاتین{Seed}های مختلف تصادفی بودن را گارانتی می‌نماید.

این پژوهش با استفاده از چهارچوب \متن‌لاتین{FLoRa} با ترکیب چهارچوب \متن‌لاتین{INET} مبتنی بر \متن‌لاتین{OMNet++} صورت پذیرفته است.
\متن‌لاتین{FLoRa} یک کتابخانه‌ی متن‌باز شبیه‌سازی است که پیاده‌سازی لایه‌های فیزیکی و پیوند داده \متن‌لاتین{LoRaWAN} را فراهم می‌کند.
این کتابخانه از ارتباطات دو طرفه میان اجزای معماری \متن‌لاتین{LoRaWAN} یعنی دستگاه‌های انتهایی، دروازه و سرور شبکه پشتیبانی می‌کند.
پژوهش توسط دو مدل جابجایی \متن‌لاتین{RWP} و گاوس--مارکو صورت می‌پذیرد و مدل ثابت به عنوان پایه این پژوهش در نظر گرفته می‌شود که مدل معمول
در شبیه‌سازی شبکه‌های \متن‌لاتین{LoRaWAN} بوده و اشیا در آن ثابت هستند.

شبیه‌سازی در دو محیط شبیه‌سازی بزرگ و کوچک صورت پذیرفته است. محیط بزرگ یک مربع ۲۰۰۰ متر در ۲۰۰۰ متر بوده که با ۴ دروازه
ثابت در گوشه‌ها پوشش داده شده است. در این محیط از ۵۰۰۰ دستگاه متحرک استفاده شده است.
محیط کوچک یک مربع ۱۰۰۰ متر در ۱۰۰۰ متر بوده که با ۲ دروازه پوشش داده شده است.
تعداد اشیا در این محدوده ۲۵۰۰ دستگاه انتهایی است.

دریافت صحیح در این شبیه‌سازی به معنی دریافت سیگنالی با قدرت بیشتر از حساسیت دستگاه دریافت کننده است.
قدرت سیگنال دریافتی به طور عمده وابسته به قدرت ارسال و از دست رفتن سیگنال است.
این پژوهش برای تصادم فرض می‌کند که دو ارسال همزمان بر روی دو کانال غیرعمود باعث ایجاد تصادم می‌گردد.
در شبیه‌سازی از مکانیزم نرخ داده تطبیق‌پذیر هم استفاده شده است و در ابتدا فاکتورهای گسترش تصادفی بوده و در طی اجرا با این مکانیزم
تغییر می‌کنند. پارامترهای مورد نظر این پژوهش نرخ دریافت بسته‌ها، نرخ تصادم و توان مصرفی هستند.

مساله مهم در نتایج این پژوهش عملکرد بهتر شبکه \متن‌لاتین{LoRaWAN} در زمان جابجایی گره‌ها با سرعت زیاد برای تعداد بالای گرهها است.
این پژوهش استدلال می‌کند جابجایی گره‌ها باعث می‌شود که به دروازه‌ها نزدیک شوند و بتوانند از فاکتور گسترش
کمتر در جهت ارتباط استفاده کنند و از همین رو تداخل در شبکه کاهش پیدا می‌کند و نرخ دریافت بسته‌ها افزایش پیدا می‌کند.
این افزایش سرعت البته تاثیر زیادی بر مصرف توان نداشته و مصرف توان به صورت کلی با افزایش تعداد اشیا، افزایش پیدا می‌کند.
در نهایت این پژوهش نتیجه‌گیری می‌کند نتایج با وجود استفاده از مکانیزم استاندارد نرخ داده تطبیق‌پذیر خوب بوده است اما
بهتر است از مکانیزمی استفاده شود که برای گرههای متحرک بهینه شده باشد.

\زیرقسمت{مرجع \مرجع{FerrndezPastor2018}}

\کوچک{ارزیابی سیستم اینترنت اشیا، کشاورزی دقیق}

در پژوهش \مرجع{FerrndezPastor2018} تلاش برای طراحی معماری در کشاورزی دقیق بوده است. این پژوهش بیان می‌کند که کشاورزان مهارت زیادی در کشاورزی بدست آورده‌اند اما آشنایی کمی با سیستم‌های اینترنت اشیا دارند
بنابراین کاربران اینترنت اشیا می‌بایست در بهبود استفاده و ترکیب آن مشارکت کنند. طراحی با مرکزیت کاربر یا اختصارا \متن‌لاتین{UCD}، متد توسعه‌ای است که گارانتی می‌کند محصول، نرم‌افزار و یا وب‌سایت به سادگی قابل استفاده باشند.
در بحث کشاورزی دقیق، طراحی با محوریت کاربر پروسه طراحی را تعریف می‌کند که کشاورزان بر چگونگی شکل‌گیری طراحی تاثیر می‌گزارند. اصول یک طراحی با مرکزیت کاربر را می‌توان به شرح زیر برشمرد:

\شروع{فقرات}
\فقره \متن‌سیاه{جمع‌آوری نیازمندی‌ها}: درک و مشخص کردن محیط استفاده
\فقره \متن‌سیاه{مشخص کردن نیازمندی‌ها}: مشخص کردن نیازمندی‌های کاربر و سازمان
\فقره \متن‌سیاه{طراحی}: تولید نمونه و طراحی
\فقره \متن‌سیاه{ارزیابی}: انجام ارزیابی‌های مبتنی بر کاربر
\پایان{فقرات}

پژوهش حاضر کشاورزی دقیق را مشتمل بر نظارت بر پارامترهای مزرعه، مانیتورینگ، نظارت بر زمین‌ها و نظارت بر انبار می‌داند. پژوهشگران سامانه \متن‌لاتین{SmartFarmNet} را طراحی کرده‌اند که پژوهش حاضر بر پایه تجربیات آن شکل گرفته است.
بستر \متن‌لاتین{SmartFarmNet} میتواند با هر سنسور، دوربین، ایستگاه‌های هواشناسی و \نقاط‌خ کار کرده و داده‌های آن‌ها را ذخیره کند.
این ذخیره‌سازی در ابر صورت می‌گیرد و از آن برای ارزیابی کارایی و ایجاد توصیه‌ها استفاد می‌گردد.
پژوهشگران مدعی هستند این سامانه اولین و بزرگترین (با توجه به تعداد سنسورهای متصل، کاربران فعال و محصولات تحت نظارت) در نوع خود است.

پژوهشگران برای فهم دقیق نیازمندی‌ها به گلخانه‌های هوشمند سر زده‌اند و به نظر می‌رسد سیستم‌های زیر در این گلخانه‌های ضروری است:
\شروع{فقرات}
\فقره سیستم‌های آبیاری و مواد مغذی
\فقره سیستم‌های خنک‌کننده و تهویه
\پایان{فقرات}
همانطور که پیشتر هم بیان شده بود این پژوهش قصد دارد از تجربه و دانش کشاورزان بهره ببرد و از همین رو با مصاحبه با کشاورزان پروسه‌ها را استخراج کرده و کار را آغاز می‌کنند.
در اولین گام پس از مصاحبه اشیا، روابط‌شان و سرویس‌های احتمالی مشخص می‌شوند. زمانی که سرویس‌ها و اشیا تشخیص داده شدند، آن‌ها می‌بایست به وسیله‌ی پروتکل‌های اینترنت اشیا متصل شوند.
این پروتکل‌ها می‌بایست استاندارد و همکنش‌پذیر باشند تا بتوان از برنامه‌های کاربردی آزاد و باز نیز بهره برد.
رابط‌های انسانی تنظیم می‌شوند،
قوانین خبره و سرویس‌های هوشمند آنالیز می‌شوند و در نهایت رویه‌ی نصب، مراقبت و راه‌اندازی مشخص می‌گردد.

این پژوهش با توجه به مواردی که ادامه می‌آید پروتکل \متن‌لاتین{MQTT} را برای لایه اپلیکشن انتخاب می‌کند.

\شروع{فقرات}
\فقره پروتکل \متن‌لاتین{MQTT} یک پروتکل اشتراک و انتشار است که برای دستگاه‌هایی با منابع محدود طراحی شده است. مدلی که توسط شرکت‌های بزرگ به صورت جهانی اجرا شده است
و می‌تواند با سیستم‌های قدیمی نیز کار کند.
\فقره تمام موضوعات از کلماتی تشکیل شده است که با ``/'' از یکدیگر جدا شده‌اند.
یک فرمت مرسوم \متن‌لاتین{/place/device-type/device-id/measurement-type/status}
است که به این ترتیب مشترکین می‌توانند روی اندازه‌گیری‌هایی که از یک کلاس خاص از اشیا می‌آید، مشترک شوند.
\فقره پهنای باند لازم برای پروتکل \متن‌لاتین{MQTT} بسیار کم بوده و ماهیت آن به گونه‌ای است که از منظر مصرف انرژی بسیار کارا است.
\فقره رابط برنامه نویسی آن بسیار ساده است و در سمت کلاینت حافظه کمی مصرف می‌کند. این باعث می‌شود که برای سیستم‌های نهفته انتخاب مناسبی باشد.
\فقره سطوح مختلف کیفیت سرویس در این پروتکل می‌تواند عملیات‌های قابل اطمینان فراهم آورد.
\پایان{فقرات}

این پژوهش بیان می‌کند که در صورت وجود تاسیسات در مزرعه، این تاسیسات می‌بایست با حفظ کارکرد قبلی وارد چرخه هوشمند‌سازی شوند. از این را با اتصال تعدادی \متن‌لاتین{Edge-Node} به این تاسیسات می‌توان
کارکرد سابق آن‌ها را حفظ کرده و از آن‌ها برای یک پروسه آموزش با نظارت استفاده کرد. این اتفاق در شکل \رجوع{شکل: استفاده از Fog-Node و Edge-Node در تاسیسات موجود کشاورزی دقیق} آورده شده است.

\شروع{شکل}
\درج‌تصویر[width=\textwidth]{./img/precision-agriculture-fog-edge-nodes.png}
\تنظیم‌ازوسط
\شرح{استفاده از \متن‌لاتین{Fog-Node} و \متن‌لاتین{Edge-Node} در تاسیسات موجود کشاورزی دقیق \مرجع{FerrndezPastor2018}}
\برچسب{شکل: استفاده از Fog-Node و Edge-Node در تاسیسات موجود کشاورزی دقیق}
\پایان{شکل}

یکی از موارد مهمی که این پژوهش مطرح می‌کند چگونگی تست و ارزیابی مدل یادگیری ماشین است. این پژوهش مراحلی را در جهت جمع‌اوری داده و تشکیل مجموعه‌های داده‌ای تست، ارزیابی و آموزش معرفی می‌کند.

برای ارزیابی از یک گلخانه هوشمند استفاده شده است. در این گلخانه آب، خاک، اقلیم و انرژی مدیریت و مانیتور می‌شوند. همانطور که پیشتر بیان شد از پروتکل \متن‌لاتین{MQTT} برای اشیا استفاده می‌شود و دلال پیام آن
در \متن‌لاتین{Fog Node}ها مستقر شده است. در این پیاده‌سازی عملی از دو \متن‌لاتین{Edge Node} و یک \متن‌لاتین{Fog Node} استفاده شده است. الگوریتم‌های یادگیری ماشین و هوش مصنوعی نیز به زبان پایتون و متن‌باز
توسعه پیدا کرده‌اند که در \متن‌لاتین{Fog Node} اجرا می‌شوند. در نهایت در این پیاده‌سازی یک قسمت ابری نیز وجود دارد که شامل داشبردها و محل نگهداری داده‌ها است. در نهایت زیرساخت ارتباطی با توجه به فضای محدود گلخانه شبکه
\متن‌لاتین{Zigbee} بوده است.

\زیرقسمت{مرجع \مرجع{Weber2016}}

\کوچک{ارزیابی پروتکل‌های وب}

پژوهش \مرجع{Weber2016} پیش از ارائه استانداردهایی مانند \متن‌لاتین{Static Context Header Compression} یا مختصرا \متن‌لاتین{SCHC} از \متن‌لاتین{IETF} در حوزه عملیاتی کردن \متن‌لاتین{IPv6} روی \متن‌لاتین{LoRaWAN}
با ارائه \متن‌لاتین{6LoRaWAN} کار کرده است و در آن زمان پژوهشگران این پژوهش در همکاری با \متن‌لاتین{IETF} قصد تهیه یک نسخه استاندارد از کارشان را داشته‌اند. این پژوهش روش پیشنهادی را به صورت عملی پیاده‌سازی و در عمل ارزیابی کرده است.

این پژوهش به دو مساله در ارتباط میان \متن‌لاتین{IPv6} و \متن‌لاتین{LoRaWAN} اشاره می‌کند. مساله اول حداقل واحد انتقالی (\متن‌لاتین{MTU}) در پروتکل \متن‌لاتین{IPv6} است که مقدار آن برابر با ۱۲۸۰ بایت است.
مساله بعدی اندازه متغیر بسته‌های \متن‌لاتین{LoRaWAN} بر پایه مقادیر مختلف برای فاکتور گسترش است. این پژوهش بیان می‌کند روش پیشنهادی \متن‌لاتین{6LoRaWAN}
دقیقا مشابه با \متن‌لاتین{6LoWPAN} یک پروتکل تطبیقی است که امکان ارتباط میان پروتکل \متن‌لاتین{IPv6} در لایه شبکه با دو لایه پیوند داده و فیزیکی \متن‌لاتین{LoRaWAN} و \متن‌لاتین{LoRa} را فراهم می‌آورد.

این پژوهش از فشرده‌سازی سرآیند برای رفع چالش اندازه بسته‌های \متن‌لاتین{IPv6} استفاده می‌کند و بیان می‌کند برای تبدیل پروتکل از \متن‌لاتین{6LoRaWAN} می‌توان از دروازه یا سرور شبکه استفاده کرد.
معماری پیشنهادی بر پایه استفاده از دروازه در قالب مدل لایه‌ای \متن‌لاتین{OSI} در شکل \رجوع{شکل: مدل لایه‌ای 6LoRaWAN} آورده شده است.

برای فشرده‌سازی سرآیند این پژوهش تنها سرآیند \متن‌لاتین{IPv6} را در نظر می‌گیرد و سرآیند پروتکل‌هایی چون \متن‌لاتین{UDP} را مورد بحث قرار نمی‌دهد.
در فرآیند فشرده‌سازی نیاز است اطلاعاتی مانند \متن‌لاتین{DevAddr} میان مبدل پروتکل و دستگاه هماهنگ شده باشند، برای همین نیاز است که این مبدل اطلاعات پروسه‌های عضویت \متن‌لاتین{ABP} و \متن‌لاتین{OTAA}
را داشته باشد. اطلاعات \متن‌لاتین{ABP} به صورت ایستا موجود هستند اما اطلاعات \متن‌لاتین{OTAA} نیاز است که پس تولید در فرآیند عضویت در این مبدل ذخیره شوند.
در نهایت این پژوهش روش پیشنهادی را پیاده‌سازی کرده و آن در یک مقیاس بسیار کوچک به صورت عملیاتی مورد ارزیابی قرار می‌دهد، در این ارزیابی فقط کارکرد صحیح مدنظر بوده است.

\شروع{شکل}
\درج‌تصویر[width=\textwidth]{./img/6lorawan.png}
\تنظیم‌ازوسط
\شرح{مدل لایه‌ای \متن‌لاتین{OSI} روش پیشنهادی \متن‌لاتین{6LoRaWAN} \مرجع{Weber2016}}
\برچسب{شکل: مدل لایه‌ای 6LoRaWAN}
\پایان{شکل}

\زیرقسمت{مرجع \مرجع{Augustin2016}}

\کوچک{ارزیابی شبکه دسترسی}

پژوهش \مرجع{Augustin2016} با هدف ارزیابی توانایی گیرنده \متن‌لاتین{LoRa} آزمایش عملی را انجام داده است. در این آزمایش دروازه در یک فضای بسته قرار گرفته است و فرستنده
به صورت متحرک در فضای شهری حرکت کرده است. پهنای باند مورد استفاده ۱۲۵ کیلوهرتز، نرخ کدگذاری $4/5$ و کمترین توان ارسالی مورد استفاده قرار گرفته است.

در آزمایش دیگری این پژوهش قصد ارزیابی پوشش شبکه \متن‌لاتین{LoRa} را داشته است و از این رو در فضای شهری پاریس دروازه را در طبقه دوم یک ساختمان مستقر کرده و فرستنده
درون یک ماشین در موقعیت‌های مشخصی قرار گرفته است. این آزمایش نشان داده است در فاصله ۳۴۰۰ متری با استفاده از فاکتور گسترش ۱۲ تا ۴۰ درصد بسته‌ها به گیرنده رسیده‌اند.
آز آنجایی که آزمایش در فضای شهری بوده است در نقاط مورد آزمایش گاه ساختمان‌های بلندی نیز وجود داشته‌اند که در نتیجه آن کارایی فاکتور گسترش ۱۲ برای افزایش برد به خوبی مشهود است.
این آزمایش پروتکل \متن‌لاتین{LoRa} را هدف قرار داده است و از بازارسال، \متن‌لاتین{Ack} یا \متن‌لاتین{ADR} استفاده‌ای نشده است.

پس از ارزیابی لایه فیزیکی و پروتکل \متن‌لاتین{LoRa} این پژوهش لایه \متن‌لاتین{LoRaWAN} را مورد ارزیابی قرار می‌دهد. در اولین ارزیابی این پژوهش قصد دارد بیشترین گذردهی یک دستگاه
با استفاده از \متن‌لاتین{LoRaWAN} را محاسبه کند. همانطور که انتظار می‌رود این گذردهی بیشتر وابسته به لایه فیزیکی بوده است و با لایه پیوند داده ارتباطی پیدا نمی‌کند اما یک دید کلی از
کارایی در هنگام استفاده از \متن‌لاتین{LoRaWAN} می‌دهد. در نتیجه این پژوهش دیده می‌شود در حالتی که اندازه بسته بسیار کوچک است \متن‌لاتین{Duty Cycle} محدود کننده نیست بلکه
محدودیت از سمت پنجره‌های دریافت است که باعث می‌شود ارسال بسته‌ی بعدی به تاخیر بیافتد. در استاندارد فعلی \متن‌لاتین{LoRaWAN} راهی برای شکستن بسته‌ها و ارسال آن‌ها در قطعات دیده نشده است.

در ادامه این پژوهش، \متن‌لاتین{LoRaWAN} را برای لود بالا ارزیابی می‌کند در نتیجه این ارزیابی مشخص می‌شود که پروتکل \متن‌لاتین{LoRaWAN} نسبت به افزایش نرخ بسته‌ها حساس است و نمی‌تواند برای لود بالا کارایی داشته باشد.
این پژوهش پیشنهاد تغییر پروتکل دسترسی همزمان از \متن‌لاتین{ALOHA} را مطرح می‌کند و از سوی دیگر بیان می‌کند این پروتکل برای مصارف حساس به تاخیر طراحی نشده است.
یک راه حل پیشنهادی برای افزایش گذردهی ارسال یک بسته
بیش از یکبار است که البته باید خطر افزایش تصادم را نیز در نظر گرفت.

\زیرقسمت{مرجع \مرجع{Bharadwaj2016}}

\کوچک{ارزیابی سیستم اینترنت اشیا، مدیریت هوشمند پسماند}

پژوهش \مرجع{Bharadwaj2016} در هندوستان انجام شده است و با توجه به حجم بالای زباله‌های شهری قصد دارد یک شیوه‌ی هوشمند برای مدیریت پسماند ارائه کند که در ابعاد شهری و عملیاتی قابل استفاده باشد.
سیستم پیشنهادی بر پایه شبکه ارتباطی \متن‌لاتین{LoRaWAN} و پروتکل \متن‌لاتین{MQTT} کار می‌کند. این پژوهش بیان می‌کند استفاده از زیرساخت ارتباطی \متن‌لاتین{GSM} با توجه به نیاز آن به سیم کارت قابلیت
استفاده در همه سطل زباله‌ها را ندارد و شیوه‌هایی که نیز بر پایه اهراز هویت اشخاص برای سطل‌های زباله کار می‌کنند در ابعاد بزرگ عملیاتی نیستند.

پارامترهای اندازه‌گیری شده در سطل‌های زباله مشابه با سایر کارها در این حوزه متشکل شده از وزن سطل، میزان پر بودن سطل و گازهای سمی داخل است.

در لایه کاربرد در این معماری سرویس‌های پنل کاربری، سرویس فراهم آوری سناریو و پردازش داده و پایگاه داده‌ای \متن‌لاتین{NoSQL} قرار گرفته‌اند که با پروتکل \متن‌لاتین{MQTT} داده‌ها را دریافت می‌کنند.
گرهها داده‌ها را به وسیله‌ی زیرساخت \متن‌لاتین{LoRaWAN} برای دروازه ارسال می‌کنند و داده‌ها از دروازه به وسیله‌ی \متن‌لاتین{MQTT} برای سرورهای ابری ارسال می‌شود.

این پژوهش برای مدیریت سطل‌های زباله شهر را به مربع‌هایی تقسیم کرده است و برای هر مربع یک مدیر در نظر گرفته شده است که با استفاده از داشبرد خود می‌تواند سطل‌های زباله تحت مدیریت خود را نظارت کند.
از سوی دیگر برنامه موبایل نیز برای رانندگان ماشین‌های حمل زباله تعبیه شده است که به واسطه‌ی آن بهترین راه برای رسیدن به سطل‌های زباله پر شده برای آن‌ها پیدا می‌شود.

\زیرقسمت{مرجع \مرجع{Chen2019}}

\کوچک{ارزیابی سیستم اینترنت اشیا، شبکه شناختی}

هدف در پژوهش \مرجع{Chen2019} ارائه‌ی یک معماری عملیاتی برای شبکه‌ی شناختی است. این پژوهش با بررسی فناوری‌های مختلف ارتباطی در شبکه‌های اینترنت اشیا به این تقطه می‌رسد که می‌توان
با استفاده از شبکه شناختی از مجموع ویژگی‌های این ارتباط‌ها بهره‌برداری کرد. این معماری پیشنهادی که از فناوری‌های ارتباطی مختلفی برای کاربردهای مختلف استفاده می‌کند در شکل \رجوع{شکل: معماری شبکه شناختی}
آورده شده است.

\شروع{شکل}
\درج‌تصویر[width=\textwidth]{./img/cognitive-lpwa.png}
\تنظیم‌ازوسط
\شرح{معماری شبکه شناختی پیشنهادی در پژوهش \مرجع{Chen2019}}
\برچسب{شکل: معماری شبکه شناختی}
\پایان{شکل}

در این شبکه شناختی گره در ابتدا با یک فناوری و سپس به واسطه فناوری‌های سلولی به لبه و سرورهای ابری متصل می‌شود. این پژوهش معماری پیشنهادی را به صورت عملی مورد ارزیابی قرار داده است.

\زیرقسمت{مرجع \مرجع{Tseng2021}}

\کوچک{ارزیابی سیستم اینترنت اشیا، ساختمان هوشمند}

پژوهش \مرجع{Tseng2021} قصد دارد مهندسی عمران را با مهندسی برق ترکیب کند تا بتوان یک دانشگاه هوشمند و سبز بسازد. این پژوهش هدف اصلی خود را تاثیر پوشش سقف بتونی با صفحات خورشیدی
بر دمای محیط عنوان می‌کند و برای این ارزیابی از سنسورهای \متن‌لاتین{LoRa} با توجه به مصرف کم و برد بالا استفاده می‌کند. این پژوهش بیان می‌کند فناوری اینترنت اشیا سبز می‌تواند به نظارت که امری مهم
در دانش مهندسی است کمک کند.

با توجه به شرایط واقعی این پژوهش، از سنسورهای دما و رطوبت برای اندازه‌گیری رطوبت و دما تجربه شده توسط انسان در محیط دانشگاه استفاده شده است.
سنسور مورد استفاده \متن‌لاتین{SHT10} بوده است و شبکه‌ی مورد استفاده نیز همانطور که ذکر شد، شبکه‌ی \متن‌لاتین{LoRa} بوده است. در این شبکه داده‌ها
به دروازه مخابره شده و از آنجا با اترنت برای ذخیره‌سازی به دیتابیس ارسال می‌گردند.

این پژوهش از \متن‌لاتین{LoRa} به صورت مستقیم برای ارسال اطلاعات استفاده کرده است و نیازی به استفاده از کلاس‌های \متن‌لاتین{LoRaWAN} نداشته است.
در نهایت هدف اصلی این پژوهش ارزیابی روش‌های ساخت یک دانشگاه سبز و هوشمند بوده است و نتایج و جمع‌بندی در همین حوزه هستند.

\زیرقسمت{مرجع \مرجع{BertrandMartinez2020}}

در پژهش \مرجع{BertrandMartinez2020} پژوهشگران قصد استفاده از یک روش ساختارمند برای ارزیابی دلال‌های پیام \متن‌لاتین{MQTT} را دارند.
این پژوهش بیان می‌کند که در حوزه ارزیابی دلال‌های پیام \متن‌لاتین{MQTT} سه بحث کلی وجود دارد:

\شروع{فقرات}
\فقره ارزیابی کمی
\فقره ارزیابی کیفی
\فقره ساختار ارزیابی
\پایان{فقرات}

در پژوهش‌های پیشین به بحث‌های کمی بسیار پرداخته شده است اما بحث‌های کیفی مانند قابلیت اطمینان نیز اهمیت فراوانی دارند
که کارهای کمتری به آن‌ها پرداخته‌اند. هدف از این پژوهش ارائه یک ساختار برای ارزیابی است که در آینده بتوان بر پایه آن ارزیابی‌هایی
را با اهداف دلخواه ولی بدون از دست دادن پارامترهای مهم و تاثیرگذار صورت داد. مراحل این متد به شرح زیر است:

\شروع{شمارش}
\فقره بیان اهداف: اولین گام در ارزیابی یک سیستم تعریف اهداف مشخص و به دور از بایاس است. این اهداف ممکن است
در طول ارزیابی و با درک بهتر مساله تغییر کنند.
\فقره تعریف سیستم: دومین گام تعریف مرزهای سیستم مورد ارزیابی است. این مرزها می‌توانند معیارهای ارزیابی
سیستم را تعریف کنند.
\فقره تعریف متریک‌ها: سومین گام مشخص کردن متریک‌هایی است که قرار است استفاده شوند. این متریک‌ها شامل
متریک‌های کارایی و متریک‌های ویژگی‌ها هستند. این متریک‌ها می‌بایست همه‌ی مدهای عملیاتی سیستم را شامل شوند.
\فقره تعریف سناریو ارزیابی: در چهارمین گام سناریویی که در آن آزمایش‌ها صورت می‌پذیرد تعریف می‌گردد. از جمله آنچه در این سناریو تعریف می‌شود
می‌توان به لود، متغیرهای تاثیرگذار و همبندی منطقی سیستم اشاره کرد.
\فقره نصب و ارزیابی: بر پایه سناریویی که در گام چهارم مشخص شد ارزیابی در گام پنجم صورت می‌گیرد. در این گام نیاز است تعداد تکرار آزمایش برای
رسیدن به اطمینان در نتایج مشخص شود.
\فقره آنالیز نتایج: در ششمین گام نتایج گام پیش به صورت کامل مورد ارزیابی و بررسی قرار می‌گیرد و مشخص می‌شود آیا برای ارائه نتیجه نهایی و جمع‌بندی
قانع کننده هستند یا خیر، در صورت نیاز ممکن است گام پنجم با تغییراتی برای رسیدن به نتایج مورد نیاز تکرار شود.
\فقره ارائه نتایج: اگر نتایج گام پیشین دید مشخصی از معیارهای مورد بحث می‌دهند، ارائه نتایج و جمع‌بندی می‌تواند به انتخاب بهتر و گسترش دانش کمک کند.
\پایان{شمارش}

این پژوهش متد ذکر شده را برای ارزیابی ۱۲ بستر \متن‌لاتین{MQTT} متن‌باز استفاده می‌کند. برای این ارزیابی سه سطح کیفیت سرویس در پروتکل \متن‌لاتین{MQTT}
مدنظر است و از سوی دیگر نیازمندی‌های غیرکارکردی مانند کیفیت مستندات، گستره تنظیمات، کارکرد بر سیستم‌عامل‌های مختلف و \نقاط‌خ. در نهایت سه بستر انتخاب
با معیارهای کیفی برای مقایسه کارایی استفاده می‌شوند.

سیستم مورد نظر در این پژوهش تمامیت بستر \متن‌لاتین{MQTT}، به عنوان یک جعبه سیاه است. در این پروژهش با اجزای داخل سیستم توجهی نشده و پارامترهایی که بر کارایی
کلاینت‌های تاثیر دارند از این پژوهش خارج هستند.

این پژوهش معیارهای غیرکارکردی را به دو دسته کلی تقسیم می‌کند متریک‌های تکنیکال و متریک‌های اپراتور. متریک‌های اپراتور متریک‌هایی است که اپراتور استفاده کننده از
پلتفرم به صورت روزانه با آن سر و کار دارد مانند کیفیت مستندات، گستردگی تنظیمات، سادگی اعمال تنظیمات و \نقاط‌خ. متریک‌های تکنیکال بحث‌هایی مانند پشتیبانی از سطوح
کیفیت سرویس متنوع، سیستم‌های مختلف و \نقاط‌خ قرار می‌گیرد.

معیارهای کارایی این پژوهش خود به سه دسته متریک‌های بهره‌وری، متریک‌های قابلیت اطمینان و دسترسی‌پذیری تقسیم می‌شوند.
بر اساس ارزیابی صورت پذیرفته از بین ۱۲ بستر متن‌باز اولیه تنها سه بستر \متن‌لاتین{RabbitMQ}، \متن‌لاتین{Mosquitto} و \متن‌لاتین{HiveMQ Community Edition}
برای ارزیابی کارایی انتخاب شده‌اند.

در نهایت این پژوهش برای کارایی این بسترها
از ترافیک زیادی استفاده نکرده است و می‌توان ارزیابی با داده‌های بیشتری نیز انجام داد و در ادامه تعداد کانکشن‌ها و توزیع‌شدگی این بسترها در نظر گرفته نشده است.

\زیرقسمت{مرجع \مرجع{Palmese2021}}

در پژوهش \مرجع{Palmese2021} پژوهشگران قصد مقایسه دو پروتکل انتشار و اشتراک \متن‌لاتین{CoAP} و \متن‌لاتین{MQTT-SN} را به صورت عملیاتی دارند.
پروتکل \متن‌لاتین{CoAP} به تازگی در پیش‌نویسی که توسط \متن‌لاتین{IETF} منتشر شده است از مدل انتشار و اشتراک پشتیبانی می‌کند و \متن‌لاتین{MQTT-SN}
مدل تغییر یافته پروتکل \متن‌لاتین{MQTT} برای شبکه‌های سنسوری است. هر دو این پروتکل‌ها بر پایه \متن‌لاتین{UDP} بوده و مقایسه آن‌ها منصفانه به نظر می‌رسد.
این پژوهش بیان می‌کند پروتکل \متن‌لاتین{CoAP} برای شبکه‌هایی با پویایی بالا انتخاب عاقلانه‌ای به نظر می‌رسد.

می‌توان گفت که امروز در بحث پروتکل‌های اینترنت اشیا، یک تصمیم‌گیری مهم بین استفاده از حالت انتشار و اشتراک یا استفاده از حالت تقاضا و درخواست است.
پروتکل‌های \متن‌لاتین{MQTT} و \متن‌لاتین{CoAP} به ترتیب از پروتکل‌های شناخته‌شده در حوزه انتشار و اشتراک و تقاضا و درخواست هستند.
این پژوهش در واقع قصد دارد آخرین بهبودهای این پروتکل‌ها در حوزه اینترنت اشیا را با یکدیگر مقایسه کند.

در نهایت جمع‌بندی این پژوهش به این شرح است. زمانی ماهیت شبکه تحت ثاثیر پارامترهایی مانند \متن‌لاتین{Duty Cycle} بوده
و اشیا نمی‌توانند یک ارتباط دائم داشته باشند استفاده از \متن‌لاتین{CoAP} گزینه‌ی بهتری است. از سوی دیگر پشتیبانی پروتکل
\متن‌لاتین{CoAP} از قطعه‌بندی آن را برای کاربردهایی با پیام‌های بزرگ مناسب می‌کند.

\زیرقسمت{مرجع \مرجع{Botez2021}}

در پژوهش \مرجع{Botez2021} هدف ارزیابی بحث‌های پردازش در ابر، پردازش در لبه و قسمت‌بندی شبکه در شبکه‌های \متن‌لاتین{5G} است.
به این منظور دو آزمایش صورت پذیرفته است. آزمایش اول استفاده از بستر ابری و لبه برای اجرای پلتفرم اینترنت اشیا
به صورت کانتینرسازی شده و ارزیابی منابع مصرفی آن است. برای بستر لبه از \متن‌لاتین{balenaCloud} و برای بستر ابری از سرویس‌های \متن‌لاتین{AWS}
استفاده شده است.
آزمایش دوم در شبکه‌ی ارتباطی \متن‌لاتین{NB-IoT} و بر بستر
پلتفرم \متن‌لاتین{Kubernetes} که بر پایه \متن‌لاتین{OpenStack} نصب شده است، صورت پذیرفته است.
هدف آزمایش دوم قسمت‌بندی شبکه برای پشتیبانی از انواع مختلف ترافیک بوده است.

در آزمایش اول، سناریو اول زیر ساخت لبه \متن‌لاتین{balenaCloud} که با استفاده از \متن‌لاتین{docker}
کار می‌کند، مورد استفاده است. پردازش روی \متن‌لاتین{RPi} نسخه ۴ صورت می‌گیرد. نتیجه پردازش داخل پایگاه داده‌ای
\متن‌لاتین{InfluxDB} ذخیره می‌شوند. در این سناریو سنسور دما \متن‌لاتین{DHT11} مستقیما
به \متن‌لاتین{RPi} نسخه ۴ متصل است.
در دو سناریو بعدی از زیرساخت ابری \متن‌لاتین{AWS} متعلق به شرکت آمازون مورد استفاده قرار گرفته است.
در این سناریوها داده از \متن‌لاتین{ESP32} به زیرساخت ابری ارسال می‌شود. در این سناریوها به ترتیب
از پروتکل‌های \متن‌لاتین{HTTP} و \متن‌لاتین{MQTT} استفاده شده است.
در زمان استفاده از \متن‌لاتین{MQTT} تاپیک‌های مختلفی برای کیفیت سرویس‌های متفاوت تخصیص داده
شده‌اند و در نهایت کارایی بهتری حاصل شده است.

در آزمایش دوم زیرساخت \متن‌لاتین{NB-IoT} در یک شبکه‌ی نسل چهار با تجهیزات شرکت نوکیا پیاده‌سازی شده است.
در این زیرساخت از \متن‌لاتین{OpenStack} استفاده شده است و بر پایه‌ی آن \متن‌لاتین{Kubernetes} فراهم شده است
تا سرویس‌های شبکه‌ی \متن‌لاتین{Backhaul} در قالب کانتینر بالا بیایند.
این سرویس‌ها \متن‌لاتین{Cloud-Native Network Function}ها نام دارند و سربار بسیار کمتری نسبت ماشین‌های مجازی
و آنچه در \متن‌لاتین{NFV MANO} بحث شده است دارند.

عملا این پژوهش ارزیابی را در حوزه مختلف انجام می‌دهد که وجه مشترکشان استفاده از سرویس‌های ابرزی و بسترهای ابری است.
این دو موضوع بحث تقسیم‌بندی در شبکه‌های نسل جدید و استفاده از پردازش در لبه است.

\زیرقسمت{مرجع \مرجع{Jurva2020}}

پژوهش \مرجع{Jurva2020} به مساله پیچیدگی در مدیریت صحن هوشمند دانشگاه می‌پردازد. این پژوهش با معرفی مدل میکرو اپراتورها و ارزیابی آن در دانشگاه \متن‌لاتین{Oulu} فلاند تلاش می‌کند این مساله را حل کند.
میکرو اپراتور یک سرویس دهنده محلی است که زیرساخت دیجیتال و سرویس‌های دانشگاه را مدیریت می‌کند.

این پژوهش دانشگاه هوشمند را به سه موجودیت تابعی می‌شکند. یک عملیات‌های اصلی دانشگاه، سرویس‌های صحن دانشگاه و محیط صحن دانشگاه.
در عملیات‌های اصلی دانشگاه، دیجیتال‌سازی و شبکه‌ها به آموزش بهتر کمک می‌کنند. دانشگاه می‌تواند از این زیرساخت برای ارائه آموزش در محدوده وسیع‌تر، واقعیت مجازی و عملیاتی ساختن تحقیقات استفاده کند.
سرویس‌های صحن دانشگاه تقریبا مشابه با سرویس‌هایی است که در مغازه‌ها، مرکزهای خرید و \نقاط‌خ در شهر هوشمند ارائه می‌شوند. از جمله این سرویس‌ها می‌توان به سرویس راهنمایی، پارک و شارژ وسایل نقلیه و \نقاط‌خ اشاره کرد.
سرویس‌های شهر هوشمند مانند حمل و نقل می‌تواند با سرویس‌های صحن دانشگاه ارتباط ترکیب شده و سرویس‌های هوشمند‌تری را به وجود بیاورند.

برای پیاده‌سازی دانشگاه هوشمند این پژوهش یک مدل (شکل \رجوع{شکل: چهارچوب دانشگاه هوشمند}) متشکل از لایه‌های افقی و عمودی را پیشنهاد می‌دهد. لایه‌ها عمودی موجودیت‌های تابعی را نمایندگی می‌کنند که پیشتر به آن پرداخته شد و لایه‌های
افقی پلتفرم‌های هستند که دسترسی را یکسان‌سازی می‌کنند. در چهارچوب پیشنهادی مدیریت زیرساخت و شبکه دانشگاه هوشمند توسط میکرو اپراتور صورت می‌پذیرد. این اپراتور می‌بایست با سیاست‌گذاران برای استفاده از پهنای باند در فضای دانشگاه
به توافق برسد، از زیرساخت‌های اپراتورهای فعلی برای سرویس خود بهره بگیرد، برای کسب مجوزهای لازم با مدیران دانشگاه در ارتباط باشد و \نقاط‌خ

\شروع{شکل}
\درج‌تصویر[width=\textwidth]{./img/smart-campus-framework.png}
\تنظیم‌ازوسط
\شرح{چهارجوب دانشگاه هوشمند \مرجع{Jurva2020}}
\برچسب{شکل: چهارچوب دانشگاه هوشمند}
\پایان{شکل}

\زیرقسمت{مرجع \مرجع{Baldo2021}}

در پژوهش \مرجع{Baldo2021} محققان بیان می‌کنند معماری‌های پیشنهادی تا به امروز بر سیستم‌های تک نقش متمرکز بودند و پژوهش حاضر قصد دارد سیستمی متشکل از سرویس‌هایی با توپولوژی‌ها متفاوت را
برای مدیریت هوشمند پسماند ارائه دهد. این معماری کلاس‌های مختلف \متن‌لاتین{LoRaWAN} را پیشتر کمتر به آن‌ها پرداخته شده بود، در برگرفته و شامل گرههای با پیچیدگی‌های رو به رشد است.
این معماری با سطل‌های ساده هوشمند شروع شده، شامل دورریزهایی است که با کاربران مراوده می‌کنند و دوربین‌هایی که سطل‌ها را برای جلوگیری از حریق نظارت می‌کنند.

معماری پیشنهادی بر پایه شبکه‌های \متن‌لاتین{LoRaWAN} از سه گره اصلی تشکیل شده است.
دسته اول گرههایی هستند که در سطل زباله‌های عمومی نصب شده و پارامترهایی مانند
میزان پر بودن، دما و جهت قرار گرفتن سطل را گزارش می‌کنند. این گروه از گرهها از کلاس $A$ از \متن‌لاتین{LoRaWAN} استفاده می‌کنند چرا که نیازی به \متن‌لاتین{Downlink} ندارند.
این گرهها با باتری فعالیت کرده و بازه نمونه‌گیری آن‌ها هر نیم ساعت است.
گرههای دسته دوم پیچیدگی بیشتری داشته و در دورریزها نصب می‌شوند. یکی از وظایف این گروه از گرهها تعامل با کاربران است.
با توجه به این موارد این دسته از گرهها با پنل‌های \متن‌لاتین{Photo-Voltaic} کار کرده و
از کلاس $B$ از \متن‌لاتین{LoRaWAN} استفاده می‌کنند.
در نهایت دسته سوم گرههای نظارت ویدیویی هستند که برای مراقبت و نظارت بر دوریزها و سطل‌های زباله نصب می‌شوند. برای حفظ حریم خصوصی این دوربین‌ها هیچ تصویری را ارسال یا ذخیره نمی‌کنند
بلکه بر پایه \متن‌لاتین{Fog Computing} و توان پردازشی بالایی که دارند این تصاویر را پردازش می‌کنند.
این گرهها با توجه به پردازش بالا و دارا بودن دوربین مصرف زیادی داشته و می‌توانند با هر دو کلاس $A$ و $B$ از \متن‌لاتین{LoRaWAN} فعالیت کنند.

تمامی گرهها از \متن‌لاتین{SF}، هفت استفاده می‌کنند تا بتوانند برد بالایی را با توان مصرفی پایینی پشتیبانی کنند. پهنای باند مورد استفاده ۱۲۵ کیلوهرتز است و از نظر کدگذاری $4/5$ استفاده می‌شود.
داده‌ها توسط دروازه‌ها جمع‌اوری شده و از آنجا توسط پروتکل \متن‌لاتین{MQTT} بدست \متن‌لاتین{Network Server} و \متن‌لاتین{Application Server} می‌رسند.
این پژوهش تمامی قطعات و چگونگی ساخت این سه گره را شرح داده است.

این پژوهش بیان می‌کنند داده‌های ساده‌ی سنسورهایی که میزان پر بودن سطل‌های زباله را نشان می‌دهند زمانی که در بستر هوش مصنوعی قرار می‌گیرند می‌توانند ارزش بالایی خلق کنند.
در نظر داشته باشید دوربین‌ها تنها برای پیشگیری از مواردی مانند آتش‌سوزی یا آسیب عمدی تعبیه شده‌اند اما در نهایت با جمع‌اوری داده‌های این سه گروه از اشیا می‌تواند تصمیم بهتری گرفته و دسته جدیدی از دادگان را خلق نمود.

برای نظارت بر سطح پر بودن سطل‌ها راه‌های متفاوتی وجود دارد که یکی از آن‌ها استفاده از سنسور فراصوت است. این سنسور جهت عملکرد مناسب می‌بایست به صورت ضد آب جایگذاری شود اما با این وجود
این سنسورها یکی از بهترین و مطمئن‌ترین راه‌ها برای ارزیابی سطح پر بودن سطل زباله هستند. در کنار این سنسورها می‌توان از سیستم‌های نظارت ویدیویی نیز بهره جست که در نهایت ترکیب آن با داده‌های سنسور
از سطل‌ها توانایی تصمیم‌گیری بهتری می‌دهد. استفاده از سیستم‌های نظارت ویدیویی اولین دغدغه‌ای که به وجود می‌اورد بحث حفظ حریم خصوصی است که این پژوهش به وسیله‌ی پردازش در لبه به آن پاسخ داده است.

گرههای قرارگرفته روی دورریزها از \متن‌لاتین{RFID} برای احراز هویت کاربران استفاده می‌کنند. آن‌ها در جهت تایید دسترسی نیاز دارند شناسه خوانده شده از \متن‌لاتین{RFID} را با سرور بررسی کنند
از این رو نیاز دارند که از ارتباط کلاس $B$ استفاده کنند تا در صورت وقوع تاخیر در سرور باز هم قابلیت دریافت پاسخ را داشته باشند. در ضمن استفاده از کلاس $B$ به سرور اجازه می‌دهد تا داده‌ها را
در صورت نیاز با درخواست از گره، دریافت کند.

\زیرقسمت{مرجع \مرجع{sensors-19-00007}}

پژوهش \مرجع{sensors-19-00007} قصد ارزیابی بین پروتکل‌های \متن‌لاتین{MQTT} و \متن‌لاتین{CoAP} که به ترتیب بر بسترهای \متن‌لاتین{UDP} و \متن‌لاتین{TCP} فعالیت می‌کنند، را دارد.
ارزیابی بر شبکه زیرساخت \متن‌لاتین{NB-IoT} صورت می‌گیرد که با فعالیت روی باند دارای لایسنس و نبود \متن‌لاتین{duty-cycle}، امکان اجرای پروتکل \متن‌لاتین{tcp} را نیز فراهم می‌آورد.
این پژوهش در نهایت نشان می‌دهد پروتکل \متن‌لاتین{MQTT} نسبت به پروتکل \متن‌لاتین{CoAP} کارآیی کمتری در معیارهای تاخیر، پوشش و ظرفیت سیستم دارد.

در این پژوهش با وجود اهمیت بحث‌های امنیتی بر پروتکل‌های اینترنت اشیا از آن‌ها صرف نظر شده است. برای امنیت پروتکل \متن‌لاتین{CoAP} می‌بایست از \متن‌لاتین{DTLS} و
برای امنیت پروتکل \متن‌لاتین{MQTT} از \متن‌لاتین{TLS} استفاده کرد.
در ضمن این پژوهش تنها به نسخه‌های مشهور این پروتکل‌های اکتفا کرده است و این در حالی است که نسخه‌های دیگری مانند \متن‌لاتین{MQTT-SN}، \متن‌لاتین{MQTT-over-QUIC} و \نقاط‌خ
که گاها برای راهکارهای اینترنت اشیا بهینه شده‌اند، نیازمند ارزیابی هستند.

ارزیابی این پژوهش بر بستر \متن‌لاتین{NB-IoT} به صورت شبیه‌سازی و بر پایه \متن‌لاتین{Ericsson internet event-based radio network system simulator} انجام داده است.

\زیرقسمت{مرجع \مرجع{sensors-20-02078}}

پژوهشگران در \مرجع{sensors-20-02078} بیان می‌کنند که کارهای زیادی در کشاورزی دقیق با هدف مصرف توان پایین پیشنهاد شده‌اند اما تعداد کمی از آن‌ها در عمل تست شده‌اند.
این پژوهش به دنبال تست عملیاتی مصرف توان در کشاورزی و کاهش آن است.

در ابتدا از بین فناوری‌های ارتباطی موجود بیان می‌شود زمانی که داده‌ی زیادی برای ارسال موجود نیست مانند سنسورهای کشاورزی، استفاده از \متن‌لاتین{LoRa} گزینه خوبی است.
در ضمن پوشش \متن‌لاتین{LTE} در نواحی غیرشهری و کشاورزی کافی نیست بنابراین استفاده از \متن‌لاتین{NB-IoT} گزینه خوبی نیست.

این پژوهش حسگرهای صنعتی کشاورزی را بررسی کرده و برای ارسال داده‌ها نرخ منطقی ۳۰ دقیقه را محاسبه می‌کند.

\زیرقسمت{مرجع \مرجع{sensors-18-00772-v3}}

در \مرجع{sensors-18-00772-v3} پژوهشگران برای ارزیابی شبکه \متن‌لاتین{LoRaWAN} در ابتدا یک ارزیابی رادیویی بر پایه مدل انتشار \متن‌لاتین{Okumura-Hata} صورت داده
و در ادامه با استفاده از محیط‌های واقعی شهری، نیمه‌شهری و غیرشهری دست به ارزیابی عملیاتی زده‌اند. ارزیابی اولیه به محققان کمک کرده است تا بتوانند محل خوبی را برای
تنها دروازه این آزمایش پیدا کنند.

در ارزیابی عملیاتی از یک گره متحرک که پارامترهای ارسالش با زمان تغییر می‌کند، استفاده شده است. این گره برای بازه‌های زمانی ثابت می‌ماند تا تاثیر حرکت در پارامترها حذف شود.
در انتها این پژوهش بیان می‌کند برای اجرای یک زیرساخت \متن‌لاتین{LoRaWAN} باید مصالحه‌ای بین کیفیت لینک، نرخ داده‌ی انتقالی و سیار بودن گره برقرار شود.

مدل \متن‌لاتین{Okumura-Hata} در پیاده‌سازی‌هایی معتبر است که دروازه نسبت به ساختمان‌ها و اطراف در ارتفاع بالاتری قرار داشته باشد.
\متن‌لاتین{Pham2020}

\زیرقسمت{مرجع \مرجع{sensors-20-00280-v2}}

پژوهش \مرجع{sensors-20-00280-v2}
اگلوریتم \متن‌لاتین{Static Context Header Compression (SCHC)} را برای \متن‌لاتین{IPv6} پیاده‌سازی کرده است.
این پژوهش از این پیاده‌سازی برای انتقال پروتکل \متن‌لاتین{CoAP} بر بستر \متن‌لاتین{UDP} و \متن‌لاتین{IPv6} استفاده کرده است.
هدف این پژوهش ارزیابی این الگوریتم بوده است.

این پژوهش بیان می‌کند منابع مصرفی در جهت استفاده از \متن‌لاتین{IPv6} نسبت به سود حاصل از آن بسیار کم است. در مقابل کارایی انرژی و داده در قطعه‌بندی کم است.
از سوی دیگر این پژوهش بیان می‌کند برای استفاده از قطعه‌بندی پیشنهادی \متن‌لاتین{IETF} نیاز است که ترتیب بسته‌ها حفظ شود.

از مزایای \متن‌لاتین{IPv6} می‌توان به زمانی اشاره کرد که یک گره بین دروازه‌ها جابجا می‌شود در صورت لزوم پروسه پیوستن به \متن‌لاتین{NS} را انجام می‌دهد که صورت استفاده از یک آدرس \متن‌لاتین{IPv6} ثابت اتصال تضمین خواهد شد.

در نظر گرفتن گرههای متحرک و سناریوهای انتها به انتها چیزی که در این پژوهش به آن پرداخته نشده است. اهمیت اصلی استفاده از پروتکل‌های استاندارد اینترنت در شبکه‌های اینترنت اشیا
در واقع ارزش خود را زمانی نشان خواهد داد که به صورت انتها به انتها استفاده شده و لایه‌های دیگر بتوانند در صورت نیاز با شیوه‌های استاندارد با اشیا ارتباط برقرار کنند.


\زیرقسمت{مرجع‌های \مرجع{SanchezIborra2020} و \مرجع{Santa2020}}

هر دو پژوهش \مرجع{SanchezIborra2020} و \مرجع{Santa2020} توسط یک تیم انجام شده و تکمیل شده یکدیگر است. این پروژه‌ها وسایل حمل و نقل نوظهور مانند دوچرخه و اسکوترهای برقی را هدف قرار می‌دهند.
هدف طراحی یک سیستم \متن‌لاتین{OBU} یا \متن‌لاتین{on-board unit} است که بتوان از آن‌ها برای گردآوری داده از این وسایل استفاده کرد.

این پژوهش‌ها بیان می‌کنند از بین راه‌کارهای توان پایین با برد بالا بهتر از برای گردآوری داده‌های غیرحیاتی از \متن‌لاتین{LoRaWAN} یا \متن‌لاتین{NB-IoT} استفاده کرد
و فناوری‌های سلولی مانند \متن‌لاتین{GSM} یا \متن‌لاتین{LTE} برای ارتباط‌های حیاتی حفظ کرد.

این پژوهش‌ها یک دستگاه \متن‌لاتین{OBU} با هر دو فناوری \متن‌لاتین{LoRaWAN} و \متن‌لاتین{NB-IoT} پیاده‌سازی کرده و آن را در عمل ارزیابی می‌کنند.
برای ارزیابی \متن‌لاتین{NB-IoT} از زیرساخت عملیاتی \متن‌لاتین{Vodafone} استفاده شده است.

\زیرقسمت{مرجع \مرجع{Islam2021}}

پژوهش \مرجع{Islam2021} به بحث استفاده از \متن‌لاتین{UAV}ها، ارتباطات \متن‌لاتین{LoRaWAN} و ارتباطات ماهواره‌ای در کشاوری پرداخته است.
این پژوهش بیان می‌کند که محدودیت‌های ارتباطی در این حوزه به قدر کافی مورد توجه قرار نگرفته است و تلاش دارد چالش‌های ارتباطی موجود در کشاورزی هوشمند را مرور کند.
این محدودیت‌های ارتباطی در کنار حسگرها، برای \متن‌لاتین{UAV}ها نیز مطرح هستند.

این پژوهش به عنوان یک راهکار در بحث مشکلات ارتباطی بحث \متن‌لاتین{Mesh LoRa} را مطرح می‌کند و مشکلات زیر را برای آن برمی‌شمارد.
\شروع{فقرات}
\فقره کارآیی و قابلیت بسیار پایین است.
\فقره نیاز به نگهداری از راه دور برای دروازه‌ها وجود دارد.
\فقره نیاز به پردازش لبه در دروازه‌ها وجود دارد.
\پایان{فقرات}

این پژوهش راهکاری برای این مشکلات ارائه نمی‌دهد و بحث \متن‌لاتین{Mesh LoRa} بسیار کلی ذکر می‌کند و ساختار مشخصی برای آن برنمی‌شمارد.

\زیرقسمت{مرجع \مرجع{Mishra2021}}

در \مرجع{Mishra2021} پژوهشگران دست به ارزیابی کارایی دلال‌های پیام برای پروتکل \متن‌لاتین{MQTT} زده‌اند. در این پژوهش معیارهای نرخ پیام، مصرف \متن‌لاتین{CPU} و تاخیر مورد نظر بوده‌اند.
در این میان تاخیر مدت زمانی است که از ارسال پیام در \متن‌لاتین{Publisher} تا دریافت آن در \متن‌لاتین{Subscriber} طول می‌کشد.
دلال‌های پیامی که برای این پژوهش بررسی شده‌اند در جدول \رجوع{جدول:دلال‌های پیام مورد ارزیابی در Mishra2021} آورده شده‌اند.

\شروع{لوح}
\شرح{دلال‌های پیام مورد ارزبابی در پروژه \مرجع{Mishra2021}}
\برچسب{جدول:دلال‌های پیام مورد ارزیابی در Mishra2021}
\فضای‌و{5mm}
\begin{tabularx}
  {\textwidth}
  {p{3cm}*6{X}}
\خط‌پر
دلال‌پیام \متن‌لاتین{MQTT} & \متن‌لاتین{Mosquitto} & \متن‌لاتین{Bevywise MQTT Route} & \متن‌لاتین{ActiveMQ} & \متن‌لاتین{HiveMQ CE} & \متن‌لاتین{VerneMQ} & \متن‌لاتین{EMQ X} \\
\خط‌پر
متن‌باز & است & نیست & است & است & است & است \\
\خط‌پر
زبان برنامه‌نویسی اصلی & \متن‌لاتین{C} & \متن‌لاتین{C} و \متن‌لاتین{Python} & \متن‌لاتین{Java} & \متن‌لاتین{Java} & \متن‌لاتین{Erlang} & \متن‌لاتین{Erlang} \\
\خط‌پر
نسخه پروتکل \متن‌لاتین{MQTT} & نسخه ۳ و ۵ & نسخه ۳ و ۵ & نسخه ۳ & نسخه ۳ و ۵ & نسخه ۳ و ۵ & نسخه ۳ \\
\خط‌پر
کیفیت‌سرویس‌های پشتیبانی شده & ۰، ۱ و ۲ & ۰، ۱ و ۲ & ۰، ۱ و ۲ & ۰، ۱ و ۲ & ۰، ۱ و ۲ & ۰، ۱ و ۲ \\
\خط‌پر
سیستم‌عامل‌های پشتبیانی شده & \متن‌لاتین{Linux, Mac, Windws} & \متن‌لاتین{Windows, Linux, Mac, Raspberry Pi} & \متن‌لاتین{Windows, Linux} & \متن‌لاتین{Windows, Mac, Linux} & \متن‌لاتین{Linux, Mac} & \متن‌لاتین{Linux, Mac, Windows} \\
\خط‌پر
\end{tabularx}
\پایان{لوح}

برای تست از دو سناریو مختلف استفاده شده است. در هر دو سناریو یک انتشاردهنده، یک مشترک و یک سرور قرار دارد. در سناریو اول برای شبیه‌سازی از لپ‌تاب‌های شخصی استفاده شده است
تا شرایط لبه شبیه‌سازی شود و در سناریو دوم از ماشین‌های مجازی زیرساخت ابری \متن‌لاتین{Google Cloud Platform} استفاده شده است.
در هر دو این سناریو مساله تعداد ارتباط‌های همزمان مورد بحث قرار نگرفته است که با توجه به تعداد زیاد اشیا می‌تواند چالش بزرگی برای منابع سیستم دلال پیام باشد.
مساله دیگر در این ارزیابی‌ها عدم استفاده از فناوری‌های ابری به روز مانند کانتینرها و \متن‌لاتین{Kubernetes} است که امروز جز جدانشدنی از پیاده‌سازی سیستم‌های اینترنت اشیا حتی در لبه هستند.
در کنار این دو مورد استفاده از رمزنگاری در پروتکل \متن‌لاتین{MQTT} ممکن است، که این امر خود می‌تواند باعثث تغییر در کارایی دلال پیام شود و از این رو نیازمند ارزبابی است که این پژوهش به آن نپرداخته است.

در نهایت این پروژهش به این جمع‌بندی میرسد که دلال‌های پیام غیرگسترش‌پذیر مانند \متن‌لاتین{Mosquitto} که از تعداد مشخصی نخ استفاده می‌کنند برای محیط‌های محدود مناسب‌تر هستند.
از سوی دیگر از میان دلال‌های پیام گسترش‌پذیر \متن‌لاتین{ActiveMQ} کارآیی بالایی داشته و \متن‌لاتین{EMQ X}، \متن‌لاتین{VerneMQ} و \متن‌لاتین{HiveMQ} کارآیی مناسبی دارند.

\زیرقسمت{مرجع \مرجع{Cruz2021}}

پژوهشگران در \مرجع{Cruz2021} استفاده آزمایشی از \متن‌لاتین{LoRa} و \متن‌لاتین{LoRaWAN} به عنوان زیرساخت مدیریت پسماند در شهر لیسبون را گزارش می‌دهند.
در حال حاضر پسماند شهر لیسبون با استفاده از زیرساخت \متن‌لاتین{GPRS} فعالیت می‌کند. مدیران شهری قصد دارند این زیرساخت را به یک زیرساخت \متن‌لاتین{LPWAN} تغییر دهند.
در حال حاضر فناوری‌های متنوعی مانند \متن‌لاتین{LoRaWAN}، \متن‌لاتین{Sigfox} و \متن‌لاتین{NB-IoT} در بازار وجود دارند و پژوهشگران قصد دارند کارآیی \متن‌لاتین{LoRaWAN} را در
سناریو مدیریت پسماند به صورت عملی بررسی کنند.

در واقع مشارکت اصلی این پژوهش در ارزیابی واقعی سنسورها، شبکه، کارایی انرژی و فناوری‌های ارتباطی است که در قالب یک مدیریت هوشمند پسماند شهری رخ می‌دهد.
آزمایش‌های یک پژوهش در دو سطح رخ می‌دهند. در سطح اول هدف ارزیابی پوشش شبکه‌ای \متن‌لاتین{LoRa} برای مانیتورینگ مخازن پسماند سطحی و زیرزمینی است.
در سطح دوم هدف بررسی ظرفیت شبکه برای ارسال داده‌های مورد نیاز اپلیکشن‌ها است.

این آزمایش‌های همگی توسط تجهیزات تجاری صورت گرفته است. برای دروازه‌ها از دو دروازه تجاری متفاوت استفاده شده است.
برای سنسورهای سطح پسماند نیز از دو سنسور تجاری متفاوت که قیمت و ویژگی‌های متفاوتی (حداکثر ارتفاع قابل سنج، سنسورهای ثانویه، باتری و \نقاط‌خ) دارند استفاده شده است.
سنسورهای سطح پسماند با نصب شدن بر درب مخزن و با استفاده از فراصوت فاصله خود تا پسماند را اندازه‌گیری می‌کنند. این روش می‌تواند خطا داشته باشد
و مکان سنسور و توان پردازشی آن تاثیر به سزایی در این امر دارد.

سه آزمایش کلی برای پوشش انجام شده است. پوشش کوتاه (نزدیک به ۱۰۰ متر)، میانی (نزدیک به یک کیلومتر) و طولانی (نزدیک به ۵ کیلومتر)، که در هر یک یک گره پسماند قرار دارد.
در آزمایش پوشش کوتاه مشکلی ایجاد نشده و همه اطلاعات دریافت می‌شوند ولی آزمایش برد طولانی عملا هیچ داده‌ای را منتقل نکرده است.
در آزمایش برد میانی گره پسماند یک گره زیرزمینی بوده است و چالش اصلی طراحی گره بوده است.

اندازه‌ی بسته‌های ارسالی از دو حسگر مقدارهای متفاوت ۴ و ۸ بایت بوده است. در شروع تست‌ها از مکانیزم نرخ داده تطبیق‌پذیر استفاده شده است و این مورد در ادامه غیرفعال شده است.

در نهایت این پژوهش بیان می‌کند که می‌توان از \متن‌لاتین{LoRa} برای زیرساخت مدیریت هوشمند پسماند شهری استفاده کرد.
البته این پژوهش در رابطه با تعداد سنسورهایی که می‌توان در شبکه داشت و تداخل آن‌ها مطالعه‌ای انجام نداده است.

\زیرقسمت{مرجع \مرجع{sensors-20-06721}}

پژوهشگران \مرجع{sensors-20-06721} تجربه بیش از دو سال نگهداری از شبکه‌ی سنسورهای فضای بسته دانشگاه \متن‌لاتین{oulu} کشور فلاند مبتنی بر \متن‌لاتین{LoRaWAN} در این پژوهش مرور می‌کنند.
این پژوهش بار زیادی داشته و این تنها مقاله‌ای نیست که از آن به چاپ رسیده است. در این تجربه ۳۳۱ سنسور در سقف در ریل‌های چراغ‌ها با فاصله‌های یک و نیم‌متری در یک محل اجتماعات در دانشگاه نصب شدند.
برای جمع‌آوری داده از یک دروازه با دید مستقیم استفاده شده است. پیش از استقرار سنسورها یک تخمین برای پارامتر \متن‌لاتین{SF} در شبکه \متن‌لاتین{LoRaWAN} نیز انجام شده است.

در این پژوهش هیچ استفاده‌از \متن‌لاتین{ADR}، بسته‌های \متن‌لاتین{downlink} و \متن‌لاتین{ACK} نشده است. نرخ ارسال سنسورها ۱۵ دقیقه‌ای بوده و اندازه بسته‌ی آنها مشخص است.
هر گره در واقع شامل پنج سنسور دما، رطوبت، شدت‌نور، تشخیص حرکت و سطح $CO_{2}$ است. این گرهها برای اتصال به شبکه از فعال‌سازی \متن‌لاتین{OTAA} استفاده می‌کنند.

یکی از موارد مهمی که در این پژوهش به آن اشاره می‌شود، از دست رفتن بسته‌ها به جز در شبکه‌ی دسترسی و در \متن‌لاتین{Backend} است.
منظور از شبکه \متن‌لاتین{Backend} زیرساخت ارتباط میان \متن‌لاتین{NS} و سرور پلتفرم است.
این بازه‌های از دست رفتن بیش از ۵۰ درصد بسته‌ها در \متن‌لاتین{Backend} به صورت دوره‌های ۱.۵ ماه رخ می‌دادند. این پژوهش به بررسی بیشتر این موضوع نپرداخته است و دلیلی ارائه نمی‌دهد.

\زیرقسمت{مرجع \مرجع{Govindan2015}}

پژوهش \مرجع{Govindan2015} بیان می‌کند که پارامترهای شبکه‌ای در راستان تضمین کیفیت \متن‌لاتین{MQTT-SN} نیاز به ارزیابی
انتها به انتها دارند که تا به حال به آن پرداخته نشده است.
هدف این پژوهش طراحی یک سیستم اولویت‌دهی بیماران در بیمارستان است. این سیستم نیاز به علائم حیاتی و شرایط بیمار دارد که
امروزه به واسطه دستگاه‌های باسیم جمع‌آوری می‌شوند و مشکلات خود را دارند.
هدف این پروژهش فراهم آوردن کیفیت سرویس بر پایه اولویت درخواست‌ها، تاخیر انتها به انتها پایین و نرخ دریافت بالا است.
این سرویس بر پایه یک سیستم بی‌سیم اینترنت اشیا خواهد بود.

در معماری این پژوهش برای تبدیل پروتکل \متن‌لاتین{MQTT-SN} به \متن‌لاتین{MQTT} نیاز به دروازه است و
می‌توان سه ساختار متفاوت برای جایگذاری و تخصیص این دروازه‌ها در نظر گرفت.
اگر بخواهیم روش‌های جایگذاری دروازه‌ها را مختصرا مرور کنیم،
در روش اول برای هر شی یک دروازه در نظر گرفته میشود، در روش دوم برای همه اشیا از یک دروازه
مشترک استفاده اما روش سوم که روش بهینه‌تری است، برای هر دسته از اشیا یک دروازه در نظر گرفته می‌شود.
این پژوهش نیز از روش سوم که با نام روش ترکیبی نیز شناخته می‌شود، استفاده می‌کند.

بیماران با استفاده از پروتکل \متن‌لاتین{MQTT-SN} علائم حیاتی خود را برای دروازه و از آنجا برای سرور ارسال می‌کنند.
پزشکان با استفاده از تلفن همراه خود و پروتکل \متن‌لاتین{MQTT-SN} این داده‌ها را از طریق دروازه و از سرور دریافت می‌کنند.
در این سیستم داده‌های ارسالی از بیماران نیازمندی تاخیر و عمر ثابتی دارند.
پارامترهای مهم کیفیت سرویس در این معماری تاخیر رسیدن پیام‌ها از بیماران به پزشکان و نرخ دریافت آن‌ها است.
کیفیت سرویس سطح ۱ (دریافت حداقل ۱ بار) از پروتکل \متن‌لاتین{MQTT} در این پژوهش هدف قرار گرفته شده است.

این پژوهش برای نیازمندی‌های سرویس انتها به انتها و سیستمی مشخص می‌کند که چه تعداد بیمار می‌توان پشتیبانی کرد
و از سوی دیگر قابلیت اطمینانی که می‌توان با تعداد مشخصی از بیماران بدست آورد را مشخص می‌کند.

با توجه به این دروازه‌ها به محض دریافت اطلاعات آن‌ها را منتقل می‌کنند، صف‌ها روی سرور قرار خواهند داشت.
این پژوهش فرض می‌کند بسته‌های داده‌ای از یک \متن‌لاتین{MSS} کمتر بوده و در یک بسته \متن‌لاتین{TCP} یا \متن‌لاتین{UDP}
قرار می‌گیرند.

در این پژوهش فرض شده است برای هر داده یک درخواست ارسال می‌شود و دو حالت کلی در نظر گرفته شده است. در حالت اول
زمانی که درخواست برای داده می‌رسد داده از پیش در سرور وجود دارد (توسط شی ارسال شده است) و در حالت دوم نیاز است
که درخواست داده برای شی ارسال شود. البته این فرض در ساختار پروتکل \متن‌لاتین{MQTT} غیرضروری به نظر می‌رسد چرا که
بعد از درخواست اشتراک سرور به صورت خودکار داده‌ها ارسال می‌کند و نیازی به درخواست نیست.

در ادامه تاخیر بین اتصال و درخواست اشتراک، تاخیر \متن‌لاتین{TCP} بر پایه الگوریتم \متن‌لاتین{Jacobson}،
تاخیر ارتباط بی‌سیم \متن‌لاتین{UDP} و تاخیر صف مدل‌سازی می‌شوند.

برای تاخیر صف در سرور ارسال بسته‌ها توسط اشیا به صورت توزیع پوآسن مدل‌سازی می‌شود و سرور به صورت
یک سیستم \متن‌لاتین{Round-Robin} با قسمت‌های مساوی مدل‌سازی می‌شود.
مدل‌سازی برای سرور در واقعا یک روش \متن‌لاتین{Processor Sharing} است.
از آنجایی که فرض شد داده‌های ارسالی در یک بسته جای می‌گیرند، مدل صف سرور $M/D/1$ خواهد بود.
در نهایت یک شبیه‌سازی در \متن‌لاتین{ns2} با استفاده از پارامترهای شبکه \متن‌لاتین{MQTT} و \متن‌لاتین{Zigbee}
صورت می‌گیرد.
