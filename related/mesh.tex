\قسمت{شبکه‌های \متن‌لاتین{Mesh}}

یکی از راه‌ها افزایش کارایی شبکه‌های \متن‌لاتین{LoRa} استفاده از لایه‌ی فیزیکی \متن‌لاتین{LoRa} و تشکیل یک شبکه \متن‌لاتین{Mesh} است. در این شبکه گرهها با کمک یکدیگر داده‌ها را ارسال کرده و به دست دروازه می‌رسانند.
پروتکل دسترسی همزمان در \متن‌لاتین{LoRaWAN} برای دسترسی تک گام در همبندی ستاره‌ای طراحی شده است و نمی‌تواند بسته را بین دستگاه‌ها جابجا کند.
البته در چنین شبکه‌هایی چالش‌های جدیدتری مانند چگونگی اضافه شدن یا حذف شدن یک گره و چگونگی ساخته شدن شبکه مطرح است.
دسته‌ای از پژوهش‌ها به پیاده‌سازی و ارزیابی چنین شبکه‌هایی پرداخته‌اند.

\زیرقسمت{مرجع \مرجع{Lee2018}}

در \مرجع{Lee2018} پژوهشگران دست به پیاده‌سازی ۱۹ گره شبکه‌ی \متن‌لاتین{Mesh LoRa} در محیط دانشگاه با ارسال داده‌ها در بازه‌های یک دقیقه‌ای زده‌اند.
پژوهشگران ادعا می‌کنند این اولین کاری است که تجربه واقعی در پیاده‌سازی \متن‌لاتین{Mesh LoRa} داشته است و در آن بیان می‌شود با این روش نیاز به افرایش تعداد دروازه‌ها از بین می‌رود.
نویسندگان معتقد هستند که استفاده از \متن‌لاتین{ALOHA} توانایی \متن‌لاتین{LoRa} در هندل کردن تعداد زیادی از اشیا را از بین برده است.
سیستم طراحی شده در این پژوهش به جای \متن‌لاتین{LoRaWAN} بر پایه لایه‌ی فیزیکی \متن‌لاتین{LoRa} است.

گرهها به صورت خودمختار برای انتخاب پدر خود در زمان پیوستن به شبکه‌ی \متن‌لاتین{Mesh} تصمیم می‌گیرند، آن‌ها برای این امر از پارامترهای پیام‌های داده‌ای یا \متن‌لاتین{Beacon}های گرههای متصل به شکبه، که به آن‌ها می‌رسد، استفاده می‌کنند.
در این ساختار هر گره لیستی از فرزندان خود نگهداری کرده و آن را در اختیار \متن‌لاتین{‌Gateway} هم قرار می‌دهد.
این شبکه برای ارتباطات بین گرهها بهینه نشده است و بیشتر هدف آن ارتباط میان دروازه و گرهها است.
بازه ارسال داده‌ها برای گرهها در این شبکه مشخص است و بر پایه عدم دریافت پیام در این بازه می‌توانند وضعیت شبکه‌ای خود را ارزیابی و پدر خود را تغییر دهند.
این پژوهش در مورد مقدار بهینه پارامترهای \متن‌لاتین{RSSI}، \متن‌لاتین{PDR} و \نقاط‌خ صحبت می‌کند و همانطور که بیان شد، ادعا می‌کند که برای رسیدن به این مقدارهای بهینه تنها دو راه افزایش تعداد دروازه‌ها یا استفاده از شبکه‌ی \متن‌لاتین{Mesh} وجود دارد.

در نهایت می‌توان گفت موارد زیر در این پژوهش دیده نشده‌اند:
\شروع{فقرات}
\فقره در این پروژه دروازه برای دریافت داده‌ها به گرهها درخواست می‌دهد و در مورد ارسال خودکار داده‌ها توسط اشیا که می‌تواند منجر به \متن‌لاتین{Collusion} شود بحث نشده است.
\فقره در نظر نگرفتن کلاس‌های کاری و توان مصرفی شبکه‌ی \متن‌لاتین{LoRa}
\پایان{فقرات}

\زیرقسمت{مرجع \مرجع{Marahatta2021}}

پژوهش \مرجع{Marahatta2021} قصد دارد استفاده از شبکه‌های \متن‌لاتین{LoRa} را در مناطق دور افتاده برای کنترل مصرف انرژی ارزیابی کند.
یکی از مسائل پیش رو در این پژوهش برای استفاده از لایه لینک \متن‌لاتین{LoRaWAN} نیاز آن به تعداد دروازه‌های بالا است.
از این رو این پژوهش به استفاده از شبکه‌ی \متن‌لاتین{Mesh LoRa} روی می‌آورد که در آن هر گره به عنوان تکرار کننده برای گرههای همسایه خود عمل کرده
و تعداد دروازه‌های مورد نیاز را کاهش می‌دهد.

این پژوهش روش پیشنهادی خود برای استفاده از \متن‌لاتین{Mesh} را در قالب شبیه‌سازی با \متن‌لاتین{ns-2} در محیط‌های شهری و غیرشهری ارزیابی می‌کند.
در صورت لزوم با توجه به پارامترهای ارتباطی، شبکه به زیرشبکه‌هایی شکسته شده است.
این پژوهش با توجه به زمینه خود در \متن‌لاتین{Smart Grid} از پارامتر جمع‌آوری استاندارد داده در این حوزه برای نرخ ارسال داده استفاده کرده است.


\زیرقسمت{مرجع \مرجع{Famaey2018}}

پژوهش \مرجع{Famaey2018} شبکه‌ی یکپارچه‌ای مبتنی بر فناوری‌های ارتباطی متنوع را تعریف می‌کند. این شبکه در واقع توسط اپراتور مجازی اینترنت اشیا
ساخته می‌شود که زیرساخت ارتباطی خود را از اپراتورهای مختلفی می‌گیرد. این زیرساخت به برنامه‌های کاربردی اجازه می‌دهد تا بتوانند با اشیایی در شبکه‌های ارتباطی مختلف ارتباط داشته باشند.
این اپراتور مجازی برای لایه شبکه از \متن‌لاتین{IPv6} و برای لایه کاربرد از پروتکل \متن‌لاتین{CoAP} استفاده می‌کند.

چالش‌های مختلفی می‌توان برای این اپراتورهای مجازی در نظر گرفت مانند مدل داده‌ای، ساختار داده‌ها و \نقاط‌خ که این پژوهش به آن‌ها نپرداخته است.

\زیرقسمت{مرجع \مرجع{Kim2020}}

پژوهش \مرجع{Kim2020} از لایه‌ی فیزیکی \متن‌لاتین{LoRa} استفاده می‌کند و قصد دارد که با استفاده از آن یک شبکه \متن‌لاتین{Mesh} با هزینه‌ی پایین تشکیل دهد.
این پژوهش قصد دارد که گرهها از \متن‌لاتین{SF}های مختلف روی یک کانال فرکانسی استفاده کنند ولی بتوانند بدون نیاز به گرههای گران قیمت با یکدیگر ارتباط بگیرند.
برای این امر گرهها با دریافت پیشاید سعی بر رمزگشایی آن با مقدارهای متفاوت \متن‌لاتین{SF} می‌کنند و با این روش می‌توانند \متن‌لاتین{SF} مورد نظر ارسال کننده را پیدا کنند.
فرآیند تشخیص به \متن‌لاتین{SF} پایین آغاز شده و تا \متن‌لاتین{SF} بالا ادامه می‌یابد. برای تطبیق پیشایند با هر یک از این \متن‌لاتین{SF}ها نیاز به زمان است
و این پژوهش بیان می‌کند که رویه شروع با \متن‌لاتین{SF} پایین و سپس رسیدن به \متن‌لاتین{SF}های بالاتر در مجموع زمان کمتری نیاز دارد.
از سوی دیگر این پژوهش بیان می‌کند پیشایند ارسالی می‌بایست به قدر کافی طولانی باشد.

یکی از مسائل پیشرو در این پژوهش عدم اطمینان کامل به فاکتور گسترش تشخیص داده شده است.
این پژوهش بیان می‌کند ممکن است به جای فاکتور گسترش اصلی اشتباه یکی از فاکتورهای گسترش همسایه انتخاب شود و
این امر بین \متن‌لاتین{SF}های ۱۰ تا ۱۲ احتمال بالاتری دارد.

برای حل این مشکل، پژوهش الگوریتم جدیدی را پیشنهاد می‌دهد. در این الگوریتم جدید فرآیند تشخیص سه بار تکرار می‌شود.
در این تکرارها اگر فاکتور گسترش نتیجه شده، ۷ یا ۸ باشد فرآیند متوقف می‌شود.
در صورتی که نتیحه حاصل بین ۹ تا ۱۲ باشد فرآیند تا انتهای پیشایند رویه تشخیص را ادامه می‌دهد.
با این روش احتمال تشخیص غلط تقریبا برابر صفر خواهد بود.

ارزیابی این سیستم به شکل عملی صورت پذیرفته است. در این ارزیابی از چیپ‌های \متن‌لاتین{SX1272} و \متن‌لاتین{SX1301} در باند فرکانسی ۸۶۰ تا ۱۰۲۰ مگاهرتز استفاده شده است.
چیپ \متن‌لاتین{SX1272} به صورت تک کانال بوده و از پهنای باندهای ۱۲۵، ۲۵۰ و ۵۰۰ کیلوهرتز پشتیبانی می‌کند.
چیپ \متن‌لاتین{SX1307} یک ارسال و دریافت کننده قوی است. این چیپ می‌تواند کانال \متن‌لاتین{IF8} و کانال‌های \متن‌لاتین{IF0} تا \متن‌لاتین{IF7} را پشتیبانی کند.
پهنای باند کانال‌های \متن‌لاتین{IF0} تا \متن‌لاتین{IF7} می‌تواند تنها ۱۲۵ کیلوهرتز باشد و این در حالی است که کانال \متن‌لاتین{IF08} میتواند پهنای‌باندهای ۱۲۵، ۲۵۰ و ۵۰۰ کیلوهرتز را داشته باشد.

چیپ \متن‌لاتین{SX1272} تنها می‌تواند با یک فاکتور گسترش کار کند این در حالی است که چیپ \متن‌لاتین{SX1307} می‌تواند با فاکتورهای گسترش متفاوتی کار کند.
برای فعالیت بدون ایراد یک شبکه نیاز است که فاکتور گسترش از پیش تعیین شود اما با الگوریتم پیشنهادی این پژوهش این نیاز از بین می‌رود و دستگاه‌ها می‌توانند
این مقدار را در طول فعالیت خود تغییر دهند.

با این روش پژوهش گذردهی انتها به انتها را به وسیله‌ی چیپ‌های ارزان \متن‌لاتین{SX1272} به اندازه چیپ‌های گران قیمت \متن‌لاتین{SX1307} افزایش می‌دهد.
