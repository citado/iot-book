\قسمت{پردازش در لبه}

با توجه به منابع محدود اشیا در شبکه‌های توان پایین با برد بلند مانند \متن‌لاتین{LoRaWAN} امکان استفاده از پردازش لبه کم به نظر می‌رسد اما پژوهش‌هایی به این امر پرداخته‌اند
چرا که امروز با افزایش تعداد اشیا و حجم داده‌ها نمی‌توان تنها به سرورهای ابری برای پردازش اتکا کرده و از سوی دیگر در برنامه‌های حساس به کارایی و پرخطر امکان استفاده از سرورهای
ابری با تاخیر غیرقابل پیش‌بینی وجود ندارد.
\مرجع{Sarker2019}

از سوی دیگر در صورت قطع بودن اینترنت با استفاده از پردازش در لبه شی یا دروازه می‌تواند پردازش‌های اولیه‌ای را انجام داده و تصمیم‌گیری کند.
دروازه می‌تواند پیش از ارسال اطلاعات به سرورهای ابری آن‌ها فشرده‌سازی کرده یا اطلاعاتی که به درستی دریافت نشده‌اند با توجه کد تصحیح خطای آن‌ها
در صورت امکان بازیابی کند.
\مرجع{Sarker2019}

در برخی از پژوهش‌ها، لبه برای اجرای الگوریتم‌های یادگیری ماشین در راستای کاهش هوشمندانه ابعاد داده‌ها و \نقاط‌خ مورد استفاده قرار گرفته است.
در نهایت لبه می‌تواند با توجه به معماری مورد استفاده در شبکه \متن‌لاتین{LPWAN} تعریف منحصر به فرد خود را داشته باشد.

\زیرقسمت{مرجع \مرجع{Taleb2017}}

پژوهش \مرجع{Taleb2017} پردازش در لبه را مستقل از فناوری ارتباط مورد استفاده بحث می‌کند. در این پژوهش هدف در نظر گرفتن جابجایی اشیا در پردازش لبه است و پژوهش قصد دارد
با جابجایی لبه بهترین کیفیت از تجربه را ارائه کند. برای این امر پژوهش حاضر از بحث مجازی‌سازی سبک یا همان کانتینرها و مهاجرت آن‌ها استفاده می‌کند.

در این معماری سرورهایی در لبه تعبیه شده‌اند که منابع پردازشی و ذخیره‌سازی را فراهم می‌کنند و وظیفه اجرای کانتینرها را برعهده دارند. از سوی دیگر برای جابجایی این کانتینرها یک حافظه مشترک
میان این سرورها تعبیه شده است.

مساله مهم در این پژوهش بحث جابجایی زنده است. در جابجایی زنده نیاز است که وضعیت حافظه کانتینر پیگیری شده و در جابجایی منتقل شود تا کانتینر در مقصد کاملا مشابه با مبدا اجرا شود.
این روش می‌بایست بدون خطا و سریع باشد اما این پژوهش هنوز ایده جابجایی حافظه را نگه داشته و تنها با کپی کردن آن روی حافظه مشترک سعی در بهبود سرعت دارد اما هنوز مشکل خطا
اجرا در برنامه وجود دارد.

یکی دیگر از چالش‌هایی که این پژوهش به آن پرداخته است جابجایی گره بین فراهم آورندگان مختلف است که ممکن است نتوان در این شرایط جابجایی زنده داشت و از این رو ممکن است
در این شرایط دوباره نیاز به همگام‌سازی با سرورهای ابری باشد.

در نهایت این پژوهش برای ارزیابی به صورت عملیاتی و با استفاده از کانتینرهای \متن‌لاتین{OpenVZ} اقدام کرده است. زیرساخت شبکه هم \متن‌لاتین{IEEE 802.11} در نظر گرفته شده است.

\زیرقسمت{مرجع \مرجع{Lee20212}}

\کوچک{بدون‌سرور، شروع سرد}

با گسترش هوش مصنوعی در کاربردهای متنوع، محاسبات بدون سرور به عنوان یک جز سازنده در توسعه سرویس‌های هوش مصنوعی ابری در حال مطرح شدن است.
محاسبات بدون سرور از مساله شروع سرد که تاخیر میان دریافت درخواست و اجرای کارکرد است رنج می‌برد و پژوهش \مرجع{Lee2021} قصد پرداختن به این موضوع را دارد.
مساله شروع سرد می‌تواند تاثیر زیادی بر روی یک جریان کاری داشته باشد چرا که شروع سرد می‌تواند روی هر یک از کارکردهای آن رخ بدهد.
ادغام کارکردها یکی از راه‌ها برای تحمل تاخیر شروع سرد در جریان کارها است.
ترکیب کارکردها زمانی که این کارکردها موازی باشند، حتی با وجود از کاهش تاخیر شروع سرد،
می‌تواند باعث افزایش تاخیر شود چرا که این کارکردها به صورت ترتیبی اجرا خواهند شد.
پژوهش پیشرو قصد ارائه یک روش برای تحمل تاخیر شروع سرد با استفاده از ترکیب کارکردها و تمرکز بر کاکردهای موازی را دارد.
