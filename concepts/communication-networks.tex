\قسمت{ارتباطات و شبکه‌ها}

ارتباطات و شبکه بخش مهمی از بلوک‌های کارکردی اینترنت اشیا را تشکیل می‌دهند.
این پروتکل‌ها با توجه به گستردگی اشیا و تنوع قابلیت‌های آن‌ها، بسیار متنوع شده‌اند و البته نباید نیازمندی‌های کیفیت سرویس متنوع برای اشیا را نیز از یاد برد.
پروتکل‌هایی مانند \متن‌لاتین{ModBus} برای اتوماسیون صنعتی، \متن‌لاتین{KNX} برای هوشمند‌سازی ساختمان‌ها و \متن‌لاتین{Wireless M-Bus} برای اندازه‌گیری آب و گاز مصرفی از سال‌های پیش وجود داشته‌اند
این پروتکل‌ها برای یک کاربردهای خاص تعریف شده‌اند.
قطعا پروتکل \متن‌لاتین{IPv6} نقش بسیار مهمی در اینتنرت اشیا خواهد شد و بسیاری از پروتکل‌ها تلاش برای استفاده از آن در محیط‌هایی با محدودیت‌های گوناگون دارند.
این فناوری‌های ارتباطی می‌توانند در سه گروه کلی طبقه‌بندی شوند که در ادامه به همراه مثال‌هایی آورده شده‌اند:

\شروع{فقرات}
\فقره \متن‌سیاه{\متن‌لاتین{Session/Application}}: \متن‌لاتین{MQTT}، \متن‌لاتین{CoAP}، \متن‌لاتین{AMQP}، \متن‌لاتین{HTTP}، \متن‌لاتین{SOAP}
\فقره \متن‌سیاه{\متن‌لاتین{Network}}: \متن‌لاتین{6LowPAN}، \متن‌لاتین{RPL}، \متن‌لاتین{IPsec}، \متن‌لاتین{TCP/UDP}، \متن‌لاتین{DTLS}، \متن‌لاتین{CORPL}
\فقره \متن‌سیاه{\متن‌لاتین{Perception/Things}}: \متن‌لاتین{WiFi}، \متن‌لاتین{Bluetooth Low Energy}، \متن‌لاتین{Z-Wave}، \متن‌لاتین{ZigBee}، \متن‌لاتین{LoRaWAN}، \متن‌لاتین{LTE}
\پایان{فقرات}

به صورت کلی می‌توان این فناوری‌های ارتباطی لایه فیزیکی را در دسته‌های برد کوتاه، متوسط و بلند دسته‌بندی کرد.
از سوی دیگر یکی از پارامترهای مهم برای این دسته از پروتکل‌ها نرخ داده‌ای است. یکی دیگر از پارامترهای مهم در شبکه‌های اینترنت اشیا توان مصرفی شبکه است، به صورت کلی اگر شبکه‌های توان پایین را
در نظر بگیریم، می‌توانیم آن را به دو دسته تقسیم کنیم \مرجع{Augustin2016}\مرجع{Li2014}:

\شروع{فقرات}
\فقره شبکه‌های محلی توان پایین که عموما بردشان زیر یک کیلومتر است. از جمله این شبکه‌ها می‌توان به \متن‌لاتین{IEEE 802.15.4}، \متن‌لاتین{IEEE 802.11ah}،
\متن‌لاتین{Bluetooth} و \متن‌لاتین{BLE} اشاره کرد.
این شبکه‌ها عموما از همبندی \متن‌لاتین{Mesh} استفاده می‌کنند و با این شیوه می‌توان پوشش آن‌ها را گسترش داد.
در شبکه‌های محلی توان پایین، شبکه‌هایی مانند \متن‌لاتین{Zigbee} وجود دارند که بر پایه \متن‌لاتین{IEEE 802.15.4} بوده اما لایه شبکه را نیز افزون بر لایه‌های پیوند داده و فیزیکی دارا هستند. شبکه‌های \متن‌لاتین{Zigbee}
از همبندی‌های متنوعی پشتیبانی می‌کنند و حتی الگوریتم‌های مسیریابی نیز برای آن‌ها وجود دارد.

\فقره شبکه‌های گسترده توان پایین که عموما بردشان بالای یک کیلومتر است و تحت قالب \متن‌لاتین{LPWAN} به آن‌ها خواهیم پرداخت.
فناوری‌های بسیاری در حوزه \متن‌لاتین{LPWAN} به بازار عرضه شده‌اند که از جمله‌ی آن‌ها می‌توان به \متن‌لاتین{SigFox}، \متن‌لاتین{NB-IoT}، \متن‌لاتین{LTE Cat-M} و \متن‌لاتین{LoRaWAN}
اشاره کرد.
\پایان{فقرات}

\شروع{شکل}
\تنظیم‌ازوسط
\درج‌تصویر[height=.5\textwidth]{img/wireless-tech-cov-thr.png}
\شرح{مقایسه فناوری‌های ارتباط بی‌سیم از نظر گذردهی و برد \مرجع{Lee2017}}
\پایان{شکل}

\گرنادرست
\زیرقسمت{\متن‌لاتین{WiFi 7}}

اندکی پس از انتشار \متن‌لاتین{WiFi 6} کارگروه \متن‌لاتین{IEEE 802.11} به همراه \متن‌لاتین{WiFi Alliance} شروع به طراحی نسل بعدی آن در شبکه‌های بی‌سیم محلی با نام \متن‌لاتین{WiFi 7} کردند.
یکی از اجزای \متن‌لاتین{WiFi 7}، \متن‌لاتین{IEEE 802.11be} است. قرار است در این نسل از \متن‌لاتین{Time-Sensitive Networking} یا \متن‌لاتین{TSN} برای ارتباط‌هایی با تاخیر کم و قابلیت
اطمینان بالا پشتیبانی شود
\مرجع{Adame2021}.

\متن‌لاتین{TSN} در ابتدا برای شبکه‌ها اترنت (\متن‌لاتین{IEEE 802.3}) طراحی شده بود اما به آرامی راه خود را به شبکه‌های بی‌سیم باز می‌کند. در \متن‌لاتین{TSN} سعی می‌شود
هیچ بسته‌ای به خاطر ازدحام بافرها از دست نرود، بسته‌های کمی در خرابی تجهیزات از دست بروند و تاخیر انتها به انتها گارانتی شده باشد.
کارگروه \متن‌لاتین{IEEE 802.11be} برای طراحی لایه \متن‌لاتین{MAC} و \متن‌لاتین{PHY} در می ۲۰۱۹ شکل گرفت. یکی از اهداف \متن‌لاتین{WiFi 7} کاهش بدترین حالت تاخیر و \متن‌لاتین{Jitter} است
که برای آن، کارگروه در حال بررسی استانداردهای \متن‌لاتین{TSN} است
\مرجع{Adame2021}.

با وجود اینکه هرگز \متن‌لاتین{WiFi} نخواهد توانست تاخیر محدودی را با توجه به ماهیت خود در استفاده از باندهای فرکانسی بدون مجوز، ارائه دهد اما استفاده از مفاهیم \متن‌لاتین{TSN}
می‌تواند آن را در زمره فناوری‌های پیشرو در \متن‌لاتین{6G} قرار دهد
\مرجع{Adame2021}.

به صورت سنتی \متن‌لاتین{WiFi} برای مدیریت دسترسی همزمان از \متن‌لاتین{Distributed Coordination Function} یا مختصرا \متن‌لاتین{DCF} استفاده می‌کند.
این شیوه بر پایه حس حامل و عقب‌نشینی نمایی عمل می‌کند. از مشکلات اصلی آن می‌توان به عدم قابلیت برای اولویت‌دهی ترافیک و از سوی دیگر غیرقابل پیش‌بینی بودن
آن اشاره کرد. در واقع در \متن‌لاتین{DCF} چند ایستگاه می‌توانند باعث اشباع شدن کانل شده و بنابراین نمی‌توان گارانتی از نظر زمانی برای داده‌ها ارائه داد
\مرجع{Adame2021}.

برای حل این مشکل روش \متن‌لاتین{EDCF} یا \متن‌لاتین{Enhanced DCF} در \متن‌لاتین{IEEE 802.11e} پیشنهاد شد. در این روش امکان اولویت‌دهی بر پایه
کاتالوگ‌های دسترسی اضافه شد. در ادامه این شیوده در \متن‌لاتین{IEEE 802.11aa} برای ارتباطات صدا و تصویر بهبود بیشتری یافت.
با این حال هیچ یک از این استانداردها کیفیت سرویس را در شرایطی که \متن‌لاتین{WiFi} دارای بار اضافه است، گارانتی نمی‌کنند
\مرجع{Adame2021}.

در لایه انتقال وجود بافر در پروتکل \متن‌لاتین{TCP} باعث تاخیرهای زیادی می‌شود و این امر کار برای انتقال جریان‌های ترافیکی \متن‌لاتین{TCP}
با استانداردهای \متن‌لاتین{TSN} سخت می‌کند. از سوی دیگر تکنیک‌های شبکه‌های سیمی مانند روش‌های نوین مدیریت صف و \نقاط‌خ در اینجا
کارایی زیادی ندارد
\مرجع{Adame2021}.

در استاندارد \متن‌لاتین{IEEE 802.11be} حالت عملیاتی چند کاناله وجود دارد. با استفاده از این حالت امکان افزایش بهره‌وری با ارسال همزمان
روی چند کانال به وجود می‌آید و از سوی دیگر می‌توان یک بسته یکسان را در چند کانال ارسال کرده تا از رسیدن آن مطمئن شد. در نهایت ارسال‌کننده
می‌تواند کانال با تاخیر کمتر را انتخاب کرده و تاخیر را کاهش دهد. این حالت عملیاتی خود می‌تواند در دو حالت همزمان و غیرهمزمان استفاده شود.
در حالت همزمان بعد از ارسال از کانال اصلی یک مدتی صبر شده و بعد می‌توان از کانال ثانویه استفاده کرد این در حالتی است که در حالت غیرهمزمان
هر دو کانال می‌توانند همزمان استفاده شوند ولی امکان تداخل میان آن‌ها وجود دارد
\مرجع{Adame2021}.
\رگ
