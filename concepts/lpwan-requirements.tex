\قسمت{نیازمندی‌های شبکه‌های \متن‌لاتین{LPWAN}}

یکی از بحث‌های اصلی این رساله شبکه‌های با گستره بالا و توان پایین است. این شبکه نیازمندی‌های خاصی دارند و فناوری‌های پیشنهاد شده در این حوزه
هر یک روش‌هاص خاص خود را برای پاسخ به این نیازمندی‌های دنبال کرده‌اند.

\زیرقسمت{نیازمندی‌های ترافیک}

در شبکه‌های اینترنت اشیا، حسگرها و سنسورها همگی از رفتار ترافیکی یکسانی پیروی نمی‌کنند. برخی نیاز دارند که پیام‌هایشان بلافاصله به مقصد برسد و این در حالی است که برخی می‌توانند تاخیر را تحمل کنند.
از سوی دیگر با افزایش تعداد اشیا برای فرآهم آوردن کیفیت سرویس نیاز به یک زمان‌بند با الویت است. پیام‌های ارسالی خود می‌توانند تعداد یا نرخ و اندازه‌های متفاوتی داشته باشند.
در نتیجه شبکه‌های اینترنت اشیا می‌بایست ظرفیت این تنوع ترافیکی را داشته باشند
\مرجع{Chaudhari2020}.

به صورت پایه‌ای شبکه‌های می‌بایست نوعی از مدیریت ترافیک کاربران و کنترل پذیرش را داشته باشند.
این امر البته به معماری شبکه دسترسی هم وابسته است، برخی از پروتکل‌های ارتباطی بدون هیچ نوعی از کنترل پذیرش
داده‌ها را ارسال یا دریافت می‌کنند و در برخی نیاز به نوعی پذیرش وجود دارد.
در شبکه‌هایی که از اشیا متنوع پشتیبانی می‌کنند ممکن است این نیازمندی گسترش پیدا کند، به طور مثال
شبکه ممکن است پذیرش درخواست‌ها را بر پایه اولویت انجام دهد یا شرایط اذحام را مدیریت کند
\مرجع{Chaudhari2020}.

\زیرقسمت{ظرفیت و چگالی}

یکی از نیازمندی‌ها پشتیبانی از تعداد بالا اشیا است. شبکه‌ها می‌بایست بتوانند بدون اختلال در فعالیت گرههای کنونی گرههای جدیدی را به شبکه اضافه کنند.
با توجه به سادگی گرهها بار اصلی این کار بر عهده دروازه‌ها و نقاط دسترسی است
\مرجع{Chaudhari2020}.

در شبکه‌های بزرگ نیاز به یک شناسه‌ی یکتا جهانی وجود دارد.
در این شبکه‌ها با توجه به تعداد زیاد اشیا لینک‌های ارتباطی می‌بایست قابل اطمینان و کارا باشند.
از سوی دیگر تداخل در این شبکه‌ها می‌تواند بسیار باشد و نیاز است با شیوه‌هایی همچون استفاده از کانال‌های مختلف،
ارسال‌های تکراری و استفاده از روش‌های تطبیقی با این امر مقابله کرد.
برای روش‌های تطبیقی نیاز به ذخیره‌سازی وضعیت ارتباطی دستگاه‌ها در ایستگاه‌های پایه است و این امر با توجه به تعداد زیاد
اشیا نیاز به بهینه‌سازی دارد
\مرجع{Chaudhari2020}.

\زیرقسمت{توان مصرفی}

این شبکه‌ها نیاز دارند توان مصرفی پایینی داشته باشند چرا که بیشتر گرهها با باتری فعالیت می‌کنند
و در برخی موارد جایگزینی باتری هم عمل سختی است
\مرجع{Chaudhari2020}.

دستگاه‌ها در این شبکه‌ها نرخ ارسال پایین دارند و برای ارسال نیز داده‌ها بسیار کوچک هستند
برای همین استفاده از مدهای خواب مناسب و غیر فعال کردن ماژول‌های پر مصرف مثل ماژول
ارتباطی برای زمان‌هایی که از آن‌ها استفاده نمی‌شود می‌تواند به میزان زیادی توان مصرفی را کاهش دهد
\مرجع{Chaudhari2020}.

اشیا در این شبکه‌ها می‌توانند از منابعی مانند باد یا انرژی خورشیدی برای شارژ کردن باتری خود استفاده کنند.
این بهبودها سربار دارند و می‌بایست بین این سربار و افزایش طول عمر باتری مصالحه نمود
\مرجع{Chaudhari2020}.

به صورت کلی مصرف توان برای پردازش‌های دستگاه بسیار کمتر از مصرف توان در ارسال و دریافت داده‌ها است
بنابراین ساختاربندی داده‌ها پیش از ارسال می‌تواند به کاهش توان مصرفی کمک کند. از سوی دیگر می‌توان پیچیدگی‌های
ارسال را با در نظر گرفتن لایه‌ی مدیریت دسترسی همزمان ساده‌تر کاهش داد
\مرجع{Chaudhari2020}.

\زیرقسمت{پوشش‌دهی}

پوشش این شبکه‌ها در مناطق غیر شهری بین ۱۰ تا ۴۰ کیلومتر و
در مناطق شهری بین ۱ تا ۵ کیلومتر است.
این شبکه گاهاً نیاز دارند در مناطقی با دسترسی سخت یا داخل ساختمان‌ها
عملیاتی شوند. این شبکه‌ها با استفاده از باندهای زیرگیگاهرتز تلاش می‌کنند تا پوشش بیشتری را با توان کمتری بدست بیاورند
\مرجع{Chaudhari2020}.

مفهوم پوشش‌دهی خود می‌تواند به افزایش پوشش جفرافیایی، دسترسی به محیط‌های اطراف موانع و پوشش محیط‌های داخلی اشاره کند.
تکنیک‌های غلبه بر تضعیف سیگنال مانند افزایش توان ارسال، افزایش حساسیت آنتن‌ها، ارسال چندباره یک بسته و کاهش نرخ ماژولیشن اینجا کمک کننده هستند
\مرجع{Chaudhari2020}.

\زیرقسمت{موقعیت‌یابی}

یکی از نیازمندی‌ها دنبال کردن اشیا یا تشخیص رویدادهایی همچون تغییر موقعیت مکانی آن‌ها است.
برای این امر می‌توان از سیستم \متن‌لاتین{GPS} یا زیر ساخت شبکه استفاده کرد و دقت‌های مختلفی
از سانتی‌متر تا متر بدست آورد
\مرجع{Chaudhari2020}.

برای موقعیت یابی می‌توان از اطلاعات زمانی مربوط به زمان رسیدن سیگنال‌ها در جهت مکان‌یابی استفاده کرد. این روش
یکی از کم هزینه‌ترین روش‌ها است. در زمانی که از شبکه‌های سلولی استفاده می‌کنیم می‌توان از اطلاعات سلول نیز
در مکان‌یابی استفاده کرد. راهکارهای مبتنی بر ماهواره در کاربردهایی که حساس به توان مصرفی نباشد می‌تواند استفاده شود
\مرجع{Chaudhari2020}.

\زیرقسمت{امنیت و حریم خصوصی}

امنیت یکی از مسائل مهم در این شبکه‌ها است و می‌بایست مسائل پایه‌ای
مثل احراز هویت، سطوح دسترسی، اطمینان، محرمانگی، امنیت داده‌ها و عدم همسان‌سازی را در نظر بگیرد.
از سوی دیگری مسائلی چون حملات توزیع شده جلوگیری از دسترسی یا تزریق کد مخرب به شبکه و \نقاط‌خ
نیز وجود دارند که باید برای آن‌ها چاره اندیشی شود.
داده‌های می‌بایست در ارسال و دریافت رمزگذاری شوند تا امنیت آن‌ها و حریم خصوصی کاربران تضمین شود
\مرجع{Chaudhari2020}.

\زیرقسمت{هزینه مناسب}

از آنجایی که تعداد زیادی از اشیا در این شبکه دخیل هستند، هزینه نگهداری از آن‌ها و شبکه می‌بایست کم باشد.
از سوی دیگر به روزرسانی‌های نرم‌افزاری از ویژگی‌های مهم اشیا است که باعث کاهش هزینه‌ها می‌شود
\مرجع{Chaudhari2020}.

استفاده از سخت‌افزار ساده‌تر یکی از مهمترین گام‌ها در کاهش هزینه‌ها است. از سوی دیگر به روزرسانی‌های
نرم‌افزاری که باعث کاهش تعداد به روزرسانی‌های سخت‌افزاری شوند می‌توانند به کاهش هزینه‌ها کمک کنند.
کارهای پردازشی و ارتباطی نیاز دارند که ساده باشند. سربار لایه‌های مختلف می‌بایست به گونه‌ای طراحی شوند
که کمینه باشند
\مرجع{Chaudhari2020}.

\زیرقسمت{اشیا با سخت‌افزارهایی با پیچیدگی کم}

از آنجایی که قصد داریم تعداد زیادی از اشیا را در برد بلندی و با هزینه پایین پوشش دهیم طراحی دستگاه‌ها با پیچیدگی کم
و ابعاد کوچک از نیازمندی‌های اساسی به شمار می‌آید. این اشیا نیازی به توان پردازشی بالا ندارند، معماری شبکه‌ای و پروتکل‌های
ساده‌ای را می‌بایست پشتیبانی کنند. دستگاه‌های ارتباطی آن‌ها ساده بوده و باید بتوانند به صورت نرم‌افزاری تنظیم شوند
\مرجع{Chaudhari2020}.

استفاده از تکنیک‌های رادیوهای نرم‌افزار بنیان می‌تواند اینجا بسیار کمک کننده باشد
البته باید در نظر داشت که استفاده از سخت‌افزارهای ساده و تکنیک‌های نرم‌افزاری
باعث ایجاد فرکانس‌های رادیویی غیر ایده‌آل می‌شود که برای رفع آن‌ها نیاز به استفاده
از پردازش‌های نرم‌افزاری سنگین است. بنابراین در اینجا مصالحه‌ای برای پیاده‌سازی
این تکنیک‌های ساده وجود دارد
\مرجع{Chaudhari2020}.

\زیرقسمت{گستردگی راه‌حل‌ها}

اشیا می‌بایست از شبکه‌های داری لایسنس و بدون لایسنس پشتیبانی کنند.
این اشیا می‌بایست از همبندی‌های متفاوت شبکه مانند \متن‌لاتین{Mesh}، \متن‌لاتین{Tree} و \متن‌لاتین{Star} پشتیبانی کنند
\مرجع{Chaudhari2020}.

راه‌های کارهای زیادی برای پیاده‌سازی شبکه وجود دارد بنابراین گرهها می‌بایست چندین حالت و فرکانس را پشتیبانی کنند.
حالت‌های مختلف به گرهها اجازه می‌دهد که در شبکه‌های مختلف که هر یک ویژگی‌های منحصر به فرد خود را دارند
فعالیت کند و از سوی دیگر فرکانس‌های مختلف به گره اجازه می‌دهد در یک فناوری از چندین فرکانس مختلف استفاده کند
\مرجع{Chaudhari2020}.

\زیرقسمت{عملکرد، ارتباطات و روابط بین شبکه‌ای}

امروز شبکه‌های مختلفی با ویژگی‌های متفاوت وجود دارند اما با گسترگی \متن‌لاتین{IP} انتخاب آن به عنوان
یک استاندارد ارتباطی مطرح است. شبکه‌ها تلاش می‌کنند تا به شبکه‌های \متن‌لاتین{IP} و پروتکل‌هایی مانند
\متن‌لاتین{CoAP} متصل شوند و آن‌ها را پشتیبانی کنند.
در شبکه‌های \متن‌لاتین{LPWAN} اندازه داده‌ها کوچک است و از این رو برای ارسال بسته‌های \متن‌لاتین{IP} و
به خصوص \متن‌لاتین{IPv6} نیاز به فشرده‌سازی وجود دارد
\مرجع{Chaudhari2020}.
