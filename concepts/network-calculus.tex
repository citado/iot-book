\قسمت{\متن‌لاتین{Network Calculus}}

\متن‌لاتین{Network Calculus} مجموعه‌ای از پیشرفت‌های اخیر است که دید عمیقی در مساله‌های جریان در شبکه‌ها ایجاد می‌کند. پایه \متن‌لاتین{Network Calculus} در تئوری ریاضی \متن‌لاتین{Dioid}ها و مشخصا \متن‌لاتین{Min-Plus dioid} نهفته است.
در ادامه به مرور مفاهیم اصلی این حوزه می‌پردازیم.

\زیرقسمت{منحنی ورودی}

جریان با تابع تجمعی $R(t)$، دارای $\alpha$ به عنوان جریان ورودی (بیشین) است اگر:

\[
  R(t) - R(s) \le \alpha(t - s) \forall t,s \ge 0
\]

که در آن $\alpha$ یک تابع صعودی است. به عنوان مثال اگر فرض کنیم جریان ورودی با الگوریتم \متن‌لاتین{Leaky Bucket} با پارامترهای $r$ و $b$، محدود شده است داریم:

\[
  \alpha(t) = rt + b
\]

جریان‌های ورودی را می‌توان با یکدیگر جمع کرد.

\زیرقسمت{پیچش \متن‌لاتین{Min-Plus}}

پیچش دو جریان $f_{1}$ و $f_{2}$ در جبر \متن‌لاتین{Min-Plus} به شکل زیر تعریف می‌شوند:

\[
  f(t) = \inf_{s \ge 0}(f_{1}(s) + f_{2}(t-s))
\]
\[
  f = f_{1} \otimes f_{2}
\]

این پیچش، ویژگی‌ها خوب پیچیش معمول را دارد:

\[
  (f_{1} \otimes f_{2}) \otimes f_{3} = f_{1} \otimes (f_{2} \otimes f_{3})
\]
\[
  f_{1} \otimes f_{2} = f_{2} \otimes f_{1}
\]

با توجه به این تعریف می‌توان گفت $\alpha$ یک منحنی ورودی برای $R$ خواهد بود اگر و تنها اگر

\[
  R \le R \otimes \alpha
\]

\قسمت{مدل‌های انتشار رادیویی}

امواج رادیویی مسیرهای مختلفی را بین گیرنده و فرستنده پیمایش می‌کنند که در نتیجه آن از دست رفت زیادی در سیگنال دریافتی رخ می‌دهد.
این از دست رفت می‌تواند ناشی از تاثیرات موانعی مانند ساختمان‌ها، درخت‌ها و تپه‌ها باشد
\مرجع{ElChall2019}.

اندازه‌گیری‌های کانال با هدف درک رفتار کانال و توسعه مدل‌های کانال واقعی و قابل اطمینان است. به صورت کلی مدل‌های انتشار به دو گروه
مدل‌های قطعی و تجربی تقسیم می‌شوند. مدل‌های قطعی بسیار پیچیده هستند و نیاز به اطلاعات دقیقی از مکان، ابعاد و پارامترهای فیزیکی همه موانع
در ناحیه دارد. این در حالی است که در مدل‌های تجربی، مقدار پارامترها با متناسب کردن داده‌های اندازه‌گیری شده در یک تابع مناسب برای یک محیط
مشخص بدست می‌آید.
این روش مدل عمومی‌تری ارائه می‌کند که می‌تواند در سیستم‌هایی که در نواحی شبیه به یکدیگر فعالیت می‌کنند، استفاده شود
\مرجع{ElChall2019}.

\زیرقسمت{مدل از دست رفت مسیر فضای آزاد}

مدل فضای آزاد یا \متن‌لاتین{Free-Space} یک مدل پایه‌ای است که از دست رفت مسیر را برای یک گیرنده و فرستنده که در مسیر دید یکدیگر قرار داشته و مانعی بین آن‌ها
نیست، اندازه‌گیری می‌کند. این مدل بر پایه‌ی معادله ارسال فضای آزاد \متن‌لاتین{Friis} در فضای لگاریتیمی است به صورت زیر است:

\begin{align}
  PL_{FS}(d)[dB] = 20\log_{10}(f) + 20\log_{10}(d) + 32.44
\end{align}

که در آن $f$ فرکانس به مگاهرتز و $d$ فاصله بین گیرنده و فرستنده به کیلومتر است
\مرجع{ElChall2019}.

\زیرقسمت{مدل از دست رفت لگاریتم فاصله}

مدل لگاریتم فاصله یا \متن‌لاتین{Log-Distance} که از آن با نام \متن‌لاتین{one-slop} نیز یاد می‌کنند یک مدل کلی
است که به دفعات در محیط‌های سربسته و روباز مورد استفاده قرار گرفته است.
این مدل فرض می‌کند که از دست رفت مسیر مطابق رابطه زیر با فاصله به صورت نمایی تغییر می‌کند:

\begin{align}
  PL(d)[dB] = 10n\log_{10}(d/d_0) + PL_0 + X_{\sigma}
\end{align}

که در آن $n$ نمای از دست رفت مسیر، $d$ فاصله بین ارسال کننده و دریافت کننده، $PL_0$
از دست رفت مرجع در فاصله $d_0$ است.
\متن‌لاتین{Shadow Fading} به واسطه یک متغیر تصادفی گاوسی $X_{\sigma}$ با میاگین صفر و انحراف معیار $\sigma$ (با واحد $dB$)
نمایش داده می‌شود.
پارامترهای از دست رفت مسیر به وسیله‌ی رگراسیون یا متناسب‌سازی منحنی روی داده‌های اندازه‌گیری شده محاسبه شده و به محیط بستگی دارد.
برای نمونه مقدارهای $n$ برابر $2.32$ و $PL_0$ برابر با $128.95$ برای شهر \متن‌لاتین{Oulu} در فلاند بدست آمده‌اند و یا
مقدارهای $n$ برابر $2.65$ و $PL_0$ برابر با $132.25$ برای شهر دورتموند در آلمان بدست آمده‌اند
\مرجع{ElChall2019}.

\زیرقسمت{مدل از دست رفت چند دیوار و طبقه}

مدل چند دیوار و طبقه یا \متن‌لاتین{Multi-Wall-and-Floor} برای مشخص کردن از دست رفت مسیر در ساختمان‌ها بوده و دقیق‌ترین
روش برای در نظر گرفتن تضعیف ناشزی از دیوارها و طبقات است. از دست رفت مسیر به صورت زیر مدل می‌شود:

\begin{align}
  PL(d)[dB] = 10n\log_{10}(d/d_0) + PL_0 + WAF + FAF
\end{align}

که در آن $WAF$ و $FAF$ به ترتیب فاکتورهای تضعیف دیوار و طبقه بوده که بر پایه تعداد دیوارها $n_w$ و طبقات $n_f$ بین فرستنده و گیرنده هستند.
این معیارها می‌توانند با استفاده از تکنیک‌های \متن‌لاتین{ray tracing} یا اندازه‌گیری‌های تجربی بدست آیند.
در واقع از دست رفت در نفوذ به دیوارها و طبقات به پارامترهای مختلفی چون فرکانس، ضخامت و مصالح بستگی دارد.
از نمونه‌های این مدل می‌توان به \متن‌لاتین{Cost 231-MWF} و \متن‌لاتین{Motley-Kennan} اشاره کرد.
در مدل \متن‌لاتین{Cost 231-MWF} داریم:

\[
  WAF = \sum n_{wi}L_{wi}
\]
\[
  FAF = L_{f}n_{f}^{((n_f + 2) / (n_f + 1) - b)}
\]

و در مدل \متن‌لاتین{Motley-Kennan} داریم:

\[
  WAF = \sum n_{wi}L_{wi} \\
\]
\[
  FAF = n_fL_f
\]

که در آن‌ها $n_{wi}$ تعداد دیوارها از جنس $i$ام و $L_{wi}$ از دست رفت دیوار از جنس $i$ام است
\مرجع{ElChall2019}.
