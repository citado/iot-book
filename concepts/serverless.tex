\قسمت{محاسبات بدون سرور\پانویس{Serverless Computing}}

محاسبات بدون سرور یک الگو جدید در اجرای کد ابری است که بر اساس تقاضا، توسط فراهم آورنده ابری اجرا می‌شود.
محاسبات بدون سرور با توجه به استفاده پویا از منابع محاسباتی و قابلیت گسترش‌پذیری خودکار برای سرویس‌های هوش‌مصنوعی
و برنامه‌های محاسبات لبه که از هوش‌مصنوعی استفاده می‌کنند، مناسب است.
در محاسبات بدون سرور، یک کارکرد تا زمانی که در حال اجرا نیست منابعی را اشغال نمی‌کند و منابع مورد نیاز آن در لحظه توسط
فراهم‌آورنده ابری تخصیص داده می‌شوند و پس از اجرا این منابع دوباره باز پس‌گرفته می‌شوند.
در زمان‌هایی که درخواست زیادی وجود دارد بستر بدون سرور کارکرد را تکثیر کرده و آن‌ها را به صورت موازی اجرا می‌کند
\مرجع{Lee20212}.

برای محاسبات بدون سرور چالش‌هایی را نیز می‌توان در نظر گرفت.
یکی از چالش‌ها محدودیت زمان اجرا است. اگر محاسبات بدون سرور روی یک زیرساخت ابری عمومی
استفاده شود، یک کارکرد محدودیت حداکثر زمان اجرا خواهد داشت.
چالش دیگر تاخیر شروع سرد است، که مصالحه‌ای با تخصیص منابع در صورت درخواست است.
همانطور که پیشتر اشاره شد، یک کارکرد زمانی که در حال اجرا نباشد منابعی را اشغال نمی‌کند و منابع
با رسیدن اولین درخواست تخصیص داده می‌شوند.
در اینجا تاخیری میان رسیدن درخواست و اجرای کارکرد وجود دارد که این تاخیر، تاخیر شروع سرد نامیده می‌شود.
با توجه به این که سرویس‌ها و برنامه‌های کاربردی هوش مصنوعی، به خصوص برای محاسبات در لبه،
از یک کارکرد تنها تشکیل نشده و به شکل یک جریان کاری هستند مساله تاخیر شروع سرد می‌تواند در آن‌ها باعث
کاهش کارایی چشم‌گیری شود
\مرجع{Lee20212}.
