\قسمت{شبکه‌های \متن‌لاتین{LPWAN}}

نیاز کاربردهای اینترنت اشیا روز به روز به فناوری‌هایی که می‌توانند عملکرد توان پایین داشته باشند
و دستگاه‌های انتهایی که بتوانند ارتباط بی‌سیم در مسافت‌های طولانی را با هزینه و پیچیدگی پایین برقرار کنند، بیشتر می‌شود.
در بیشتر کاربردها، دستگاه‌های انتهایی اینترنت اشیا، حسگرهایی هستند که با باتری فعالیت می‌کنند و پروفایل مصرف توان آن‌ها در جهت افزایش طول عمر
باتریشان می‌بایست با دقت طراحی شده باشد.
برد ارتباطی نیاز دارد از چند صد متر تا چندین کیلومتر را شامل شود چرا که دستگاه‌های ارتباطی در محیط عملیاتی بزرگی گسترده‌اند.
با نظر گرفتن همه ویژگی‌های نامبرده، این امر تنها با استفاده از فناوری‌های حوزه شبکه‌های توان پایین با برد بالا\پانویس{LPWAN} ممکن است
\مرجع{sensors-18-03995}.

بحث شبکه‌های مدرن \متن‌لاتین{LPWAN} از سال ۲۰۰۹ و با \متن‌لاتین{Sigfox} آغاز شد \مرجع{Fujdiak2022}.
به صورت کلی می‌توان فناوری‌های این حوزه را به سه دسته تقسیم کرد. دسته اول بر پایه زیرساخت شبکه‌های سلولی فعالیت می‌کنند،
دسته دوم به زیرساختی که توسط یک شخص ثالث تهیه شده است، احتیاج دارند و در نهایت دسته سوم به زیرساخت ثالثی احتیاج ندارند
\مرجع{Almuhaya2022}.

ارتباطات در این شبکه‌ها به صورت نامتقارن است. با هدف کاهش مصرف توان در بیشتر این راهکارها تمرکز بر ارتباط \متن‌لاتین{uplink}
بوده است و به این ترتیب \متن‌لاتین{downlink} بسیار محدود بوده و زمان مورد نیاز برای ``گوش‌دادن'' و دریافت داده را کاهش می‌دهد.
واضح است که بیشتر جریان‌های داده‌ای از سمت اشیا به شبکه می‌رسند اما در صورت وجود عملگر در کنار حسگرها نیاز به \متن‌لاتین{downlink}
احساس می‌شود و از سوی دیگر می‌توان از آن برای به روزرسانی‌های نرم‌افزاری اشیا نیز استفاده کرد
\مرجع{SanchezIborra2016}.

به طور خلاصه، برتری‌های اصلی زیرساخت‌های \متن‌لاتین{LP-WAN} را می‌توان به شرح زیر دانست \مرجع{SanchezIborra2016}:

\شروع{شمارش}
\فقره گسترش‌پذیری و پوشش‌دهی بالا، که برای شبکه‌های بسیار شلوغ که در محیط‌های گسترده نصب می‌شوند، لازم است.
\فقره فراگَرد، که برای ردیابی محصولات کاربردی است.
\فقره هشدارهای رویداد همزمان، که به وسیله مشتریان راه‌اندازی شده و از طریق سیستم مدیریتی اپراتور \متن‌لاتین{LP-WAN} فعال می‌شود.
\فقره هزینه و مصرف توان پایین در اشیا
\پایان{شمارش}

\متن‌لاتین{SigFox} قصد دارد یک پوشش جهانی را در قالب یک اپراتور شبکه که در کشورهای مختلف با استفاده از شرکت‌های تابعه اجرا می‌شود، فراهم آورد.
این شبکه به صورت کامل از سخت‌افزار تا لایه شبکه در انحصار همین شرکت است و همکاری با آن تنها راه برای عملیاتی کردن این شبکه است.
\متن‌لاتین{NB-IoT} توسط شرکت‌های مخابراطی به عنوان یک جایگزین در ارتباطات اینترنت اشیا، نسبت به فناوری‌های زیرگیگاهرتز \متن‌لاتین{LPWAN} ارائه می‌شود.
از آنجایی \متن‌لاتین{NB-IoT} در طیف فرکانسی دارای لایسنس فعالیت می‌کند، می‌تواند قابلیت اطمینان بیشتری در ترافیک نسبت به سایر فناوری‌های زیرگیگاهرتز ارائه دهد.
برخلاف \متن‌لاتین{SigFox} و \متن‌لاتین{NB-IoT}، \متن‌لاتین{LoRaWAN} قابلیت ارائه به صورت شبکه‌های خصوصی و ادغام آسان با پلتفرم‌های شبکه‌ای جهانی مانند \متن‌لاتین{The Things Network} را فراهم می‌آورد.
به همین دلیل و از سوی دیگر باز بودن استاندارد، \متن‌لاتین{LoRaWAN} توجه جامعه محققان را از اولین نمود خود در بازار جلب کرده است
\مرجع{sensors-18-03995}
\مرجع{Mekki2019}.

در جدول \رجوع{جدول: مقایسه فناوری‌های LPWAN} فناوری‌های مطرح \متن‌لاتین{LPWAN} در معیارهای مختلف مقایسه شده‌اند. این مقایسه نشان می‌دهد فناوری‌های \متن‌لاتین{LoRaWAN}
و \متن‌لاتین{Sigfox} نسبت به سایر فناوری‌ها در طول عمر دستگاه، ظرفیت شبکه، نرخ داده تطبیق‌پذیر و هزینه برتری دارند \مرجع{Almuhaya2022}.

\begin{table}
\caption{مقایسه فناوری‌های \متن‌لاتین{LPWAN} \مرجع{SanchezIborra2016} \مرجع{Mekki2019} \مرجع{Naik2018} \مرجع{Almuhaya2022}}
\label{جدول: مقایسه فناوری‌های LPWAN}
\begin{latin}\begin{tabularx}
  {\textwidth}
  {|*{6}{X|}}
  \toprule

  &
  LoRaWAN &
  Sigfox &
  NB-IoT &
  Ingenu &
  Telensa \\

  \midrule

  Band &
  Sub-GHz ISM &
  Sub-GHz ISM &
  Licensed &
  2.4 GHz ISM &
  Sub-GHz ISM \\

  \midrule

  Data Rate (uplink) &
  50 (FSK) 0.3--37.5 (LoRa) kbps &
  100 bps &
  64 kbps &
  624 kbps &
  62.5 bps \\

  \midrule

  Range &
  5 km &
  10 km &
  35 km &
  15 km &
  1 km \\

  \midrule

  Number of Channels &
  8 &
  360 &
  --- &
  40 &
  --- \\

  \midrule

  MAC &
  ALOHA &
  none &
  Non-Access Stratum &
  --- &
  --- \\

  \midrule

  Topology &
  Star-of-Stars &
  Star &
  Star &
  Star / Tree &
  Star / Tree \\

  \midrule

  Adaptive Data Rate &
  Yes &
  No &
  No &
  Yes &
  No \\

  \midrule

  Payload Length &
  256 B &
  12 B &
  1600 B &
  10 kB &
  65 kB \\

  \midrule

  Handover &
  No &
  No &
  Yes &
  Yes &
  --- \\

  \midrule

  Authentication / Encryption &
  AES 128 &
  No &
  LTE Encryption &
  --- &
  --- \\

  \midrule

  Over the air update &
  --- &
  --- &
  --- &
  --- &
  --- \\

  \midrule

  Battery life &
  10Y+ &
  10Y+ &
  --- &
  --- &
  10Y+ \\

  \midrule

  Bi-Directional &
  Yes &
  Yes &
  Yes &
  Yes &
  Yes \\

  \bottomrule
\end{tabularx}\end{latin}
\end{table}

شبکه‌های \متن‌لاتین{Sigfox} پهنای باند بسیار کمی داشته و محدودیت‌های زیادی برای اندازه بسته و تعداد بسته‌ها در نظر گرفته است.
برای \متن‌لاتین{uplink} در این شبکه‌ها در روز ۱۴۰ بسته و برای \متن‌لاتین{downlink} در روز تنها ۴ بسته ظرفیت وجود دارد.
با توجه به مدل تجاری خاص آن که پیشتر
به آن پرداخته شد، توجه‌ها بیشتر به سوی \متن‌لاتین{LoRaWAN} معطوف شده است
\مرجع{Adelantado2017}.
نرخ ارسال داده در این شبکه‌ها ۱۰۰ بیت بر ثانیه بوده و بسته‌ها می‌توانند حداکثر ۱۲ بایتی باشند
که البته این محدودیت‌ها به طول عمر بالای باتری و پوشش‌دهی گسترده منجر شده است
\مرجع{SanchezIborra2016}.

\متن‌لاتین{Dash7} یک استاندارد باز ارائه شده توسط \متن‌لاتین{Dash7 Alliance} است.
برخلاف فناوری‌های مطرح دیگر در این حوزه، \متن‌لاتین{Dash7} از همبندی درختی دو گامی استفاده می‌کند
که به معماری شبکه‌های سنتی \متن‌لاتین{WSN} شباهت دارد.
برتری اصلی این پروتکل پوشش‌دهی بیشتر در مقایسه با راهکارهای \متن‌لاتین{WSN} به خاطر استفاده از باندهای
زیرگیگاهرتزی و امکان ارتباط مسقتیم میان اشیا است
\مرجع{SanchezIborra2016}.

مستقل از ارتباط رادیویی که برای شکل دادن شبکه‌ی \متن‌لاتین{M2M} از آن استفاده شده است، دستگاه انتهایی یا ماشین می‌بایست داده خود را از طریق اینرتنت قابل دسترسی کنند.
دستگاه اینترنت اشیا عموما منابع محدودی دارند و این به آن معناست که باید با حافظه، توان پردازشی، توان شبکه‌ای و باتری محدودی فعالیت کنند.
بنابراین کارایی ارتباط ماشین به ماشین وابستگی زیادی به پروتکل زیرین مورد استفاده در اپلیکشن اینرتنت اشیا دارد
\مرجع{Mishra2021}.

پروتکل‌های ارتباطی زیادی در لایه شبکه و کاربرد اینترنت اشیا مطرح است که می‌توان از بین آن‌ها به \متن‌لاتین{MQTT}، \متن‌لاتین{CoAP}، \متن‌لاتین{AMQP} و \متن‌لاتین{HTTP} اشاره کرد
\مرجع{Mishra2021}. در ادامه این رساله به این پروتکل‌ها نیز پرداخته خواهد شد.
