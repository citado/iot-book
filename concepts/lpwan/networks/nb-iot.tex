\زیرقسمت{\متن‌لاتین{NB-IoT}}

همانطور که اشاره شد شبکه \متن‌لاتین{NB-IoT} از باند دارای لایسنس استفاده می‌کند و همین موضوع باعث می‌شود که مساله ازدحام حل شود و سرویس‌های
قابل اطمینان بیشتری برای برنامه‌های حساس به وجود بیاید.
فرکانس رادیویی این شبکه از \متن‌لاتین{Narrowband}های ۱۸۰ کیلوهرتزی تشکیل شده است که برابر با یک بلاک ریسورس فیزیکی یا \متن‌لاتین{PRB} در شبکه \متن‌لاتین{LTE} است.
سه وضعیت عملیاتی برای شبکه‌های \متن‌لاتین{NB-IoT} وجود دارد.
در وضعیت اول یا \متن‌لاتین{Stand-Alone} از طیف فرکانسی اختصاصی برای \متن‌لاتین{NB-IoT} استفاده می‌شود، که این طیف می‌تواند از باندهای فرکانسی \متن‌لاتین{GSM} باشد.
در وضعیت دوم یا \متن‌لاتین{Guard-Band} از منابع استفاده نشده در باند محافظ \متن‌لاتین{LTE} استفاده می‌شود.
در وضعیت سوم یا \متن‌لاتین{In-Band} از ۲۰۰ کیلوهرتز پهنای باند مربوط به \متن‌لاتین{LTE} استفاده می‌شود. در این شیوه نیاز به طیف بیشتری نیاز نیست اما نگاشت منابع می‌بایست به گونه‌ای باشد
که عمود بودن با شبکه فعلی را تضمین کند و ممکن است پهنای باند \متن‌لاتین{LTE} کاهش پیدا کند
\مرجع{Mekki2019}
\مرجع{Lee2017}.

لایه‌ی کنترلی شبکه‌های سلولی حاضر برای سرویس‌هایی با منابع بالا مانند سرویس دسترسی به اینترنت، چندرسانه‌ای و صوتی
طراحی شده‌اند.
لایه‌ی کنترلی شبکه‌های \متن‌لاتین{LTE} پیش از ارسال پیام نیاز به جابجایی ۱۱ پیام دارد، این امر در شبکه‌های \متن‌لاتین{NB-IoT}
بهینه شده است و لایه کنترلی این شبکه‌ها تنها نیاز به ۴ پیام دارد. این امر در شکل \رجوع{شکل: مقایسه شمای ارسال در LTE و NB-IoT}
نمایش داده شده است
\مرجع{Lee2017}.

\شروع{شکل}
\تنظیم‌ازوسط
\درج‌تصویر[width=\textwidth]{img/nbiot-deployment-scenarios.png}
\شرح{سناریوهای استقرار شبکه \متن‌لاتین{NB-IoT} \مرجع{Lee2017}}
\پایان{شکل}

\شروع{شکل}
\تنظیم‌ازوسط
\درج‌تصویر[width=\textwidth]{img/cp-nb-iot-vs-lte.png}
\شرح{مقایسه شِمای ارسال در \متن‌لاتین{LTE} \متن‌لاتین{(a)} و \متن‌لاتین{NB-IoT} \متن‌لاتین{(b)} \مرجع{Lee2017}}
\برچسب{شکل: مقایسه شمای ارسال در LTE و NB-IoT}
\پایان{شکل}

شبکه \متن‌لاتین{NB-IoT} در قیاس با سایر شبکه‌های \متن‌لاتین{3GPP} بیشینه‌ی نرخ داده‌ی کمتری داشته،
پوشش آن بیشتر بوده و پیچیدگی سخت‌افزاری آن کاهش پیدا کرده است.
بنابراین \متن‌لاتین{NB-IoT} می‌تواند هزینه و انرژی مصرفی را کاهش دهد که دو مساله اصلی در استفاده از فناوری‌های
شبکه سلولی در دستگاه‌های اینترنت اشیا است
\مرجع{Lee2017}.

گرهها در شبکه‌ی \متن‌لاتین{NB-IoT} می‌توانند دو حالت \متن‌لاتین{eDRX} یا \متن‌لاتین{PSM} را برای صرفه‌جویی انتخاب کنند.
در مد \متن‌لاتین{eDRX} دستگاه برای مدت تا ۱۷۵ دقیقه مودم خود را خاموش می‌کند.
در \متن‌لاتین{DRX} که پیشتر هم در شبکه‌های سلولی وجود داشته است همین رویه برای بازه‌ی کوتاه $2.56$ ثانیه خاموش می‌شده است
و تفاوت \متن‌لاتین{eDRX} در همین مدت زمان است.
در نظر داشته باشید که مساله زمان از این جهت مطرح است که در صورت خاموش بودن مودم پاسخ \متن‌لاتین{downlink} با تاخیر مواجه می‌شود.
در روش \متن‌لاتین{PSM} مودم برای مدتهای طولانی مانند چندین ماه خاموش می‌شود.
این حالت برای سنسورهایی که صرفا در شرایط مشخصی \متن‌لاتین{uplink} دارند، کاربرد دارد
\مرجع{Lee2017}.
