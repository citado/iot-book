\زیرقسمت{\متن‌لاتین{LoRaWAN}}

\متن‌لاتین{LoRaWAN} پروتکل لایه لینک و شبکه بوده که شامل پروتکل کنترل دسترسی چندگانه\پانویس{MAC} نیز است.
این پروتکل اجازه می‌دهد تا دستگاه‌هایی با لایه فیزیکی \متن‌لاتین{LoRa} با برنامه‌های کاربردی ارتباط برقرار کنند.
این پروتکل توسط \متن‌لاتین{LoRa Alliance}، گروهی متشکل از \متن‌لاتین{IBM}، \متن‌لاتین{Semtech}، \متن‌لاتین{Actility} و \نقاط‌خ،
توسعه پیدا کرده و برای همگان قابل استفاده است.
این پروتکل برای ارتباط دستگاه به دستگاه ایجاد نشده است و تنها هدف آن ارتباط اشیا با دروازه و \متن‌لاتین{Network Server} است.
در صورت نیاز به ارتباط بین دستگاه‌ها می‌بایست از دروازه و \متن‌لاتین{Network Server} استفاده کرد یا اینکه
تنها لایه‌ی فیزیکی \متن‌لاتین{LoRa} را مورد استفاده قرار داد.
از سوی دیگر \متن‌لاتین{LoRaWAN} از لایه‌های فیزیکی \متن‌لاتین{LoRa}، \متن‌لاتین{FSK} و \متن‌لاتین{LR-FHSS} پشتیبانی می‌کند
\مرجع{Cruz2021}
\مرجع{Augustin2016}.

یک شبکه‌ی \متن‌لاتین{LoRaWAN} در ساده‌ترین شکل از اجزای زیر تشکیل شده است \مرجع{Fujdiak2022}:

\شروع{شمارش}
\فقره یک دستگاه حسگر یا عملگر که توان مصرفی و محاسبات محدودی دارد.
\فقره یک دروازه که عنصر شبکه‌ای برای دریافت و ارسال اطلاعات از و به دستگاه‌ها است. این عنصر شبکه برای ارتباط سرور شبکه
از زیرساخت \متن‌لاتین{IP} و لایه فیزیکی با گذردهی بالا مانند \متن‌لاتین{Ethernet} استفاده می‌کند.
\فقره سرور شبکه که پیام‌های دریافت شده از یک مجموعه دروازه‌ها را به برنامه‌های کاربردی می‌رساند و برعکس.
سرور شبکه وظیفه حذف بسته‌های تکراری، ارسال \متن‌لاتین{Ack}
و رمزگشایی آن‌ها برعهده دارد. از سوی دیگر بسته‌های ارسالی به دستگاه‌ها در سرور شبکه ساخته و صف می‌شوند.
\فقره سرور اتصال که پروسه‌های فعال‌سازی بر پایه \متن‌لاتین{OTAA} و \متن‌لاتین{ABP} را برای دستگاه‌های انتهایی مدیریت می‌کند.
\فقره سرور برنامه کاربردی که می‌تواند در بستر اینترنت قرار داشته باشد و داده‌ها را از طریق سرور شبکه برای اشیا ارسال و دریافت کند.
این سرور برنامه کاربردی کلید رمزنگاری متقارنی با اشیا داشته و می‌تواند داده‌ها را رمزگذاری و رمزگشایی کند. استاندارد در رابطه با چگونگی ادغام
کاربران انتهایی با سرور برنامه کاربردی صبحتی نکرده اما عموما در پیاده‌سازی‌ها از پروتکل \متن‌لاتین{MQTT} یا یک ارتباط ساده \متن‌لاتین{TCP/IP} استفاده شده است \مرجع{Carvalho2019}.
\پایان{شمارش}

\شروع{شکل}
\تنظیم‌ازوسط
\درج‌تصویر[width=\textwidth]{./img/nrm-home.png}
\شرح{مدل مرجع شبکه \متن‌لاتین{LoRaWAN} --- شبکه‌ی خانگی}
\برچسب{شکل: مدل مرجع شبکه LoRaWAN شبکه‌ی خانگی}
\پایان{شکل}

\شروع{شکل}
\تنظیم‌ازوسط
\درج‌تصویر[width=\textwidth]{./img/nrm-roaming.png}
\شرح{مدل مرجع شبکه \متن‌لاتین{LoRaWAN} --- شبکه‌ی فراگرد}
\برچسب{شکل: مدل مرجع شبکه LoRaWAN شبکه‌ی فراگرد}
\پایان{شکل}

\شروع{شکل}
\درج‌تصویر[width=\textwidth]{./img/lora-architecture-osi.png}
\تنظیم‌ازوسط
\شرح{معماری شبکه \متن‌لاتین{LoRaWAN} از نگاه مدل لایه‌ای \متن‌لاتین{OSI} \مرجع{Ertrk2019}}
\پایان{شکل}

برخلاف شبکه‌های سلولی سنتی، در \متن‌لاتین{LoRaWAN} ارتباطی میان دروازه و دستگاه‌های انتهایی شکل نمی‌گیرد.
دروازه‌ها در واقع نقش رله‌ای در لایه لینک را ایفا می‌کنند و بعد از افزودن اطلاعات مبنی بر کیفیت پیام دریافتی آن
را به سرور شبکه ارسال می‌کنند. بنابراین دستگاه‌های انتهایی با سرور شبکه ارتباط دارند که وظیفه آن رمزگشایی بسته‌ها، حذف بسته‌های تکراری و
انتخاب دروازه مناسب جهت ارسال بسته به دستگاه انتهایی است.
بنابراین می‌توان گفت که در شبکه‌ی \متن‌لاتین{LoRaWAN} عملا دروازه از دید دستگاه‌های انتهایی پنهان است
\مرجع{Augustin2016}.

در حوزه امنیت \متن‌لاتین{LoRaWAN}، دولایه از امنیت را تعریف می‌کند. لایه اول امنیت میان شی و شبکه است در حالی که لایه دوم میان شی و برنامه کاربردی است.
به این صورت می‌توان مطمئن شد که تنها برنامه کاربردی است که می‌تواند داده‌های ارسالی توسط دستگاه را رمزگشایی کند
\مرجع{Cruz2021}. شبکه‌ی \متن‌لاتین{LoRaWAN} از معدود سیستم‌های اینترنت اشیا است که رمزنگاری انتها به انتها را پیاده‌سازی کرده است
\مرجع{Kufakunesu2020}.

در ضمن \متن‌لاتین{LoRaWAN} ویژگی‌های دیگری مانند نرخ داده تطبیقی\پانویس{ADR} را اضافه می‌کند. در نرخ داده تطبیقی شبکه با دستگاه در رابطه با پارامترهای لایه‌ی فیزیکی \متن‌لاتین{LoRa} مذاکره می‌کند
که در نتجیه آن کارآیی مصرف بهینه می‌شود. شکل \رجوع{شکل: لایه‌های لورا} مدل لایه‌ای \متن‌لاتین{LoRa} و \متن‌لاتین{LoRaWAN} را نمایش می‌دهد
\مرجع{Cruz2021}.

\شروع{شکل}
\درج‌تصویر[width=\textwidth]{./img/lora-layers.png}
\تنظیم‌ازوسط
\شرح{مدل لایه‌ای \متن‌لاتین{LoRa} و \متن‌لاتین{LoRaWAN} \مرجع{Cruz2021}}
\برچسب{شکل: لایه‌های لورا}
\پایان{شکل}

در شبکه‌های \متن‌لاتین{LoRaWAN} سه کلاس کاری می‌توان برای اشیا در نظر گرفت.

\شروع{فقرات}
\فقره در کلاس ($A$ یا \متن‌لاتین{All}) شی هر زمان که به خواهد شروع به ارسال داده کرده و دو پریود متوالی آینده را برای دریافت \متن‌لاتین{Downlink} خواهد داشت. این کلاس پایین‌ترین مصرف انرژی را دارد چرا که شی تنها در زمان‌هایی که لازم است
روشن می‌شود و می‌تواند دوباره خاموش شود. با توجه به ساختار دریافت \متن‌لاتین{Downlink} در این کلاس می‌توان به سادگی برای پیام‌ها \متن‌لاتین{Ack} دریافت کرد اما برای سایر پیام‌های \متن‌لاتین{Downlink} می‌بایست تا تصمیم شی برای ارسال داده صبر کرد.
\فقره در کلاس ($B$ یا \متن‌لاتین{Beacon}) گره به صورت همگام می‌تواند \متن‌لاتین{Downlink} دریافت کند برای اینکار گره به جز دو بازه دریافت که در کلاس $A$ تعریف شده بود یک بازه دریافت قابل پیش‌بینی نیز دارد.
این کلاس مصرف بالاتری دارد چرا که نیاز است یک بازه دوره‌ای برای دریافت \متن‌لاتین{Downlink} روشن شود. مزیت این کلاس قابلیت دریافت \متن‌لاتین{Downlink} حتی در زمان‌هایی که ارسالی ندارد، است. برای همگام‌سازی اشیا کلاس $B$ از \متن‌لاتین{Beacon}های دوره‌ای استفاده می‌شود.
\فقره در کلاس ($C$ یا \متن‌لاتین{Continues}) بیشترین مصرف توان را داشته و شی در هر زمان می‌تواند داده دریافت کند. این کلاس از اشیا از قابلیت‌های کلاس $A$ پشتیبانی کرده اما قابلیت‌های کلاس $B$ را ندارند.
\پایان{فقرات}

پیشتر به ساختار بسته‌ها در لایه‌ی فیزیکی \متن‌لاتین{LoRa} پرداختیم. در \متن‌لاتین{LoRaWAN} برای پیام‌های \متن‌لاتین{uplink} استفاده از سرآیند و \متن‌لاتین{CRC} اجباری است
و بنابراین نمی‌توان در \متن‌لاتین{LoRaWAN} از فاکتور گسترش ۶ استفاده کرد. از سوی دیگر تنها استفاده از سرآیند در بسته‌های \متن‌لاتین{downlink} اجباری است.
نرخ کدگذاری در \متن‌لاتین{LoRaWAN} مشخص نشده و قابل تغییر است
\مرجع{Augustin2016}.

\شروع{شکل}
\درج‌تصویر[width=\textwidth]{./img/lorawan-packet.png}
\تنظیم‌ازوسط
\شرح{بسته‌های پروتکل \متن‌لاتین{LoRaWAN} (سایز فیلدها به بیت آمده است) \مرجع{Augustin2016}}
\برچسب{شکل: بسته‌های LoRaWAN}
\پایان{شکل}

ساختار پیام‌های \متن‌لاتین{LoRaWAN} در شکل \رجوع{شکل: بسته‌های LoRaWAN} آورده شده است. \متن‌لاتین{DevAddr} آدرس دستگاه است.
\متن‌لاتین{FPort} پورت برای مالتی‌پلکس است و در صورتی که مقدار آن صفر باشد نشان‌دهنده‌ی این موضوع است که بسته تنها شامل
دستورات لایه‌ی \متن‌لاتین{MAC} است. بسته‌های \متن‌لاتین{uplink} ادرس مقصد و بسته‌های \متن‌لاتین{downlink} آدرس مبدا ندارند
چرا که هر شی تنها با یک سرور شبکه در ارتباط است
\مرجع{Augustin2016}.

در \متن‌لاتین{LoRaWAN} از الگوریتم کنترل دسترسی همزمان \متن‌لاتین{ALOHA} استفاده میشود. در این الگوریتم بسته‌ها می‌توانند اندازه‌های متغیر داشته باشند، گرهها هر زمان که قصد داشته باشند داده ارسال کنند و نیازی به همگام‌سازی زمانی ندارد.
مشکل اصلی این روش در تعداد برخوردهای بالا در زمانی است که شبکه گرههای زیادی دارد، از این رو روش‌های ``گوش‌دادن پیش از حرف‌زدن'' مانند \متن‌لاتین{CSMA/CA} کارایی بهتری دارند.
چنین الگوریتم‌هایی در مقابل نیاز به همگام‌سازی دارند به این معنی که می‌بایست اشیا یک مرجع زمان محلی مشترک داشته باشند
\مرجع{Beltramelli2021}.
ارسال اطلاعات همگام‌سازی در صورتی که از طریق همان بستر بی‌سیم رخ بدهد به آن همگام‌سازی داخل باند گفته می‌شود. در این روش نیاز است که به محدودیت‌های بستر رادیویی احترام گذاشت.
در این شیوه ممکن است کیفیت ارتباط رادیویی به علت ارسال همین اطلاعات به میزان غیرقابل قبولی افت کند. برای حل این مشکل می‌توان از راه‌حل‌های خارج از باند استفاده کرد
\مرجع{Beltramelli2021}.

به صورت کلی می‌توان الگوریتم‌های دسترسی همزمان را در سه گروه اصلی قرار داد:

\شروع{فقرات}
\فقره الگوریتم‌های قطعه‌بندی کانال
\فقره الگوریتم‌های دسترسی تصادفی
\فقره الگوریتم‌های نوبت‌دهی
\پایان{فقرات}

که الگوریتم \متن‌لاتین{ALOHA} یا \متن‌لاتین{CSMA} در دسته الگوریتم‌های دسترسی تصادفی است.
روش \متن‌لاتین{ALOHA} خود به دو گونه \متن‌لاتین{Pure ALOHA} و \متن‌لاتین{Slotted ALOHA}
قابل پیاده‌سازی است. در روش \متن‌لاتین{Pure ALOHA} که در \متن‌لاتین{LoRaWAN} نیز استفاده می‌شود، گرهها هر زمان که داده‌ای داشته باشند می‌توانند آن را ارسال کنند و در صورت رخ دادن تصادم
با احتمال $p$ داده را باز ارسال می‌کنند. در روش \متن‌لاتین{Slotted ALOHA} نیاز است که گرهها با یکدیگری همگام بوده و در ابتدای بازه‌های مشخصی شروع به ارسال کنند و در صورت
وقوع تصادم بعد از صبر کردن در ابتدای بازه‌ی بعدی با احتمال $p$ باز ارسال را شروع می‌کنند. کارایی روش \متن‌لاتین{Slotted ALOHA} از روش \متن‌لاتین{ALOHA} بیشتر بوده است ولی نیاز به همگام‌سازی گرهها دارد
که خود موضوعی هزینه‌بر است. پژوهش‌های متعددی روش‌های همگام‌سازی برای \متن‌لاتین{LoRaWAN} در جهت استفاده از \متن‌لاتین{Slotted ALOHA} پیشنهاد داده‌اند.
می‌توان نشان داد کارایی پروتکل \متن‌لاتین{ALOHA} برابر $\frac{1}{2e}$ است و از سوی دیگر کارایی پروتکل \متن‌لاتین{Slotted ALOHA} تقریبا دو برابر بوده و برابر $\frac{1}{e}$ است.

ذکر این نکته خالی از لطف نیست که تفاوت آنچه در \متن‌لاتین{LoRaWAN} به عنوان \متن‌لاتین{ALOHA} پیاده‌سازی شده است و آنچه به عنوان \متن‌لاتین{ALOHA} شناخته می‌شود،
در اندازه بسته‌ها است،
در \متن‌لاتین{LoRaWAN} اندازه بسته‌ها متغیر است ولی در \متن‌لاتین{ALOHA} اندازه‌ی بسته‌ها ثابت در نظر گرفته می‌شود
\مرجع{Augustin2016}.

پروتکل \متن‌لاتین{LoRaWAN} یک پروتکل تک گام است و از مسیریابی یا ارسال چند گامی پیام پشتیبانی نمی‌کند. از دیگر ویژگی‌های مهم این پروتکل می‌توان به تشخیص
موقعیت مکانی بدون نیاز به \متن‌لاتین{GPS} و تنها با ۳ دروازه اشاره کرد
\مرجع{Ertrk2019}.
اولین نسخه از استاندارد \متن‌لاتین{LoRaWAN} در سال ۲۰۱۵ منتشر شد. در طی این سال‌ها بهبودهایی در آن حاصل شد که مهمترین نسخه‌های آن ۱.۰.۳ و ۱.۱ است
\مرجع{Ertrk2019}.

\شروع{شکل}
\درج‌تصویر[width=\textwidth]{./img/lorawan-gps.png}
\تنظیم‌ازوسط
\شرح{تشخیص موقعیت مکانی در \متن‌لاتین{LoRaWAN} \مرجع{Ertrk2019}}
\پایان{شکل}

برای پیوستن یک شی به شبکه \متن‌لاتین{LoRaWAN} نیاز است که آن شی فعال‌سازی شود. دو راه برای فعال‌سازی اشیا وجود دارد: \متن‌لاتین{Over-The-Air Activation} یا مختصرا \متن‌لاتین{OTAA}
و \متن‌لاتین{Activation By Personalization} یا مختصرا \متن‌لاتین{ABP}
\مرجع{Augustin2016}.

پروسه فعال‌سازی می‌بایست اطلاعات زیر را در اختیار اشیا قرار بدهد:
\شروع{فقرات}
\فقره آدرس دستگاه انتهایی (\متن‌لاتین{DevAddr}): یک شناسه‌ی ۳۲ بیتی که ۷ بیت آن نماینده شبکه و ۲۵ بیت آن آدرس دستگاه انتهایی در شبکه است.
\فقره شناسه برنامه کاربردی (\متن‌لاتین{AppEUI}): شناسه عمومی برنامه کاربردی که از فضای آدرس \متن‌لاتین{IEEE EUI64} انتخاب شده و به صورت یکتا صاحب این دستگاه انتهایی را مشخص می‌کند.
\فقره کلید نشست شبکه (\متن‌لاتین{NetSKey}): کلیدی که میان دستگاه و سرور شبکه برای محاسبه و اطمینان از یکپارچگی اطلاعات استفاده می‌شود.
\فقره کلید نشست برنامه کاربردی (\متن‌لاتین{AppSKey}): کلیدی که میان دستگاه و برنامه کاربردی برای رمزگذاری و رمزگشایی اطلاعات استفاده می‌شود.
\پایان{فقرات}

در \متن‌لاتین{OTAA} فرآیند عضویت شامل یک درخواست عضویت و پاسخی مبتنی بر پذیرفته شدن عضویت است که برای هر نشست جدید استفاده می‌شود.
بنابر پاسخی که مبتنی بر پذیرفته شدن عضویت می‌آید، دستگاه انتهایی کلیدهای جدید نشست شبکه و برنامه کاربردی را دریافت می‌کند.
در فرایند \متن‌لاتین{ABP} این کلیدها به صورت مستقیم روی دستگاه‌های انتهایی ذخیره شده‌اند
\مرجع{Augustin2016}.

در \متن‌لاتین{LoRaWAN} دستورات \متن‌لاتین{MAC}ای تعریف شده است که اجازه می‌دهد پارامترهای دستگاه‌های انتهایی سفارشی شود.
از بین این دستورات تنها \متن‌لاتین{LinkCheckReq} از سمت شی ارسال شده و هدف آن بررسی ارتباط است.
سایر این دستورات از سوی سرور شبکه ارسال می‌شوند و می‌توانند پارامترهایی مانند نرخ داده تنظیم پذیر را تغییر دهند
\مرجع{Augustin2016}.

آنجه در \متن‌لاتین{LoRaWAN} بیان می‌شود در رابطه با دستگاه‌ّها است و در رابطه با سرور شبکه حرفی به میان نیامده است اما سرور شبکه نقشه حیاتی در کارکرد شبکه ایفا می‌کند.
اگر بخواهیم در شبکه میلیون دستگاه را هندل کنیم در این صورت بهینه‌سازی این دستگاه‌ّها بر عهده سرور شبکه خواهد بود. در صورتی که سرور شبکه دستورات \متن‌لاتین{MAC}
درستی را ارسال نکند یا پارامترها را به شکل اشتباهی تغییر دهد می‌تواند کاملا کارکرد شبکه را مختل کند
\مرجع{Augustin2016}.

از سوی دیگر در رابطه با نقش دروازه‌ها در \متن‌لاتین{LoRaWAN} تنها به عنوان یک رله اکتفا می‌شود و مسئولیت انتخاب بهترین دروازه برای ارسال
برعهده سرور شبکه خواهد بود. تنها مسئولیت دروازه زمان‌بندی صحیح در جهت ارسال است و این زمان‌بندی می‌بایست به گونه‌ای باشد که از تصادم در هنگام ارسال
\متن‌لاتین{Downlink} جلوگیری کند. البته استاندارد \متن‌لاتین{LoRaWAN} در رابطه با این زمان‌بندی صحیتی به میان نمیاورد
\مرجع{Augustin2016}.

با وجود اینکه \متن‌لاتین{LoRaWAN} عموما برای سناریوهای ارتباطات ثابت استفاده می‌شود، کاربردهایی از آن در سناریوهای متحرک نیز وجود دارد.
\متن‌لاتین{LoRaWAN} می‌تواند برای سناریوهای متحرکی چون نظارت بر ناوگان، اجاره دوچرخه، آتش‌سوزی جنگل‌ها و دنبال کردن حیوانات
استفاده شود \مرجع{Almojamed2021}.

پیام‌هایی که از سمت گره به دروازه ارسال می‌شوند، \متن‌لاتین{Uplink} یا اختصارا \متن‌لاتین{UL} نامیده می‌شوند و
پیام‌هایی که از سمت دروازه به گره ارسال می‌شوند، \متن‌لاتین{Donwlink} یا اختصارا \متن‌لاتین{DL} نامیده می‌شوند.
به صورت کلی دو نوع پیام در شبکه‌های \متن‌لاتین{LoRaWAN} وجود دارد، پیام‌های بدون تاییدیه که در آن‌ها گره نیازی به دریافت پاسخ
از \متن‌لاتین{NS} ندارد و پیام‌های با تاییدیه که نیاز به دریافت پاسخ دارند
\مرجع{Kufakunesu2020}.

یکی از مسائلی که \متن‌لاتین{LoRaWAN} به آن پرداخته است، مساله فراگردی است. در \متن‌لاتین{LoRaWAN} این امکان وجود دارد
که یک شی بتواند بین شبکه‌هایی با مدیریت‌های مختلف جابجا شود. در این شرایط نیاز است که لایه پیوند داده شی توسط یک
سرور شبکه کنترل شود و اطلاعات شی روی یک سرور شبکه‌ای دیگر قرار داشته باشد. مدل مرجع شبکه‌ی \متن‌لاتین{LoRaWAN}
به ترتیب در زمان فراگرد و در زمان استفاده از شبکه‌ی خانگی در شکل‌های \رجوع{شکل: مدل مرجع شبکه LoRaWAN شبکه‌ی فراگرد}
و \رجوع{شکل: مدل مرجع شبکه LoRaWAN شبکه‌ی خانگی} آمده است.
با توجه به تعریف فراگَردی در پروتکل \متن‌لاتین{LoRaWAN} این پروتکل به یک گزینه خوب در کنترل و مدیریت ناوگان بدل شده است.

از سال ۲۰۱۵ تا به امروز نسخه‌های مختلفی از استاندارد \متن‌لاتین{LoRaWAN} منتشر شده است.
به ترتیب نسخه $1.0$ در سال ۲۰۱۵، نسخه $1.0.1$ در سال ۲۰۱۶، نسخه $1.0.2$ در سال ۲۰۱۶،
نسخه $1.1$ در سال ۲۰۱۷، نسخه $1.0.3$ در سال ۲۰۱۸، نسخه $1.0.4$ در سال ۲۰۲۰ منتشر شده‌اند.
نسخه $1.1$ ویژگی‌های فراگردی و بهبودهای امنیتی را به \متن‌لاتین{LoRaWAN} اضافه کرد اما با توجه به عدم مهاجرت
صنعت استانداردهای $1.0$ ادامه پیدا کردند
\مرجع{Fujdiak2022}.
