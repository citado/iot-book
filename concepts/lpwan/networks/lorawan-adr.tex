\زیرقسمت{مکانیزم نرخ داده تطبیق‌پذیر در \متن‌لاتین{LoRaWAN}}

\متن‌لاتین{LoRaWAN} از مکانیزم نرخ داده تطبیق‌پذیر برای بهینه‌سازی نرخ‌داده، زمان ارسال و انرژی مصرفی به صورت پویا استفاده می‌کند.
در این مکانیزم پارامترهای ارسال که عبارتند از پهنای باند (\متن‌لاتین{BW})، فاکتور گسترش (\متن‌لاتین{SF})،
توان ارسال (\متن‌لاتین{TP}) و نرخ کدگذاری (\متن‌لاتین{CR})، کنترل می‌شوند.
بهینه‌سازی مکانیزم نرخ داده تطبیق‌پذیر ظرفیت شبکه را افزایش می‌دهد چرا که بسته‌هایی که با فاکتورهای گسترش مختلف ارسال می‌شوند،
بر یکدیگر عمود بوده و بنابراین می‌توانند به صورت همزمان دریافت شوند و زمان ارسال کاهش پیدا کند.
اگر بخواهیم بهتر بیان کنیم، مکانیزم نرخ داده تطبیق‌پذیر پارامترهای ارسال \متن‌لاتین{uplink} دستگاه‌های \متن‌لاتین{LoRa}
را با توجه به بودجه لینک کنترل می‌کند. برای استفاده از مکانیزم نرخ داده تطبیق‌پذیر، این مکانیزم می‌بایست در دستگاه انتهایی فعال باشد
\مرجع{Kufakunesu2020}.

الگوریتم نرخ داده تطبیق‌پذیر عملا در سمت گره و در سمت سرور شبکه پیاده‌سازی می‌شود. استاندارد \متن‌لاتین{LoRaWAN}
در رابطه با پیاده‌سازی آن در سمت سرور شبکه صحبتی نکرده است و در نهایت این دو الگوریتم می‌توانند به صورت غیرهمگام با یکدیگر
اجرا شوند. در شکل \رجوع{شکل: فلوچارت الگوریتم نرخ داده تطبیق‌پذیر در سمت گره} فلوچارت الگوریتم نرخ‌داده تطبیق‌پذیر
در سمت گره آورده شده است
\مرجع{Kufakunesu2020}.
این مکانیزم به گونه‌ای است که در سمت گره می‌توان نرخ داده را کاهش داده و در سمت سرور شبکه این نرخ را افزایش داد
\مرجع{Potsch2017}.

\شروع{شکل}
\تنظیم‌ازوسط
\درج‌تصویر[height=.5\textwidth]{img/lorawan-node-adr-flowchart.png}
\شرح{فلوچارت الگوریتم نرخ داده تطبیق‌پذیر در سمت گره \متن‌لاتین{LoRaWAN} \مرجع{Kufakunesu2020}}
\برچسب{شکل: فلوچارت الگوریتم نرخ داده تطبیق‌پذیر در سمت گره}
\پایان{شکل}

الگوریتم‌های زیادی برای نرخ داده تطبیق‌پذیر پیشنهاد شده‌اند که هر یک برای برآورده شدن هدف و شاخص کارایی متفاوتی در شبکه‌ی \متن‌لاتین{LoRaWAN}
تلاش می‌کنند.
در بستر متن‌باز \متن‌لاتین{The Things Network}
الگوریتمی بر پایه الگوریتم توصیه‌شده \متن‌لاتین{Semtech} برای مکانیزم نرخ داده تطبیق‌پذیر پیاده‌سازی شده است
\مرجع{Kufakunesu2020}. در ادامه به مرور تعدادی از این الگوریتم‌ها می‌پردازیم.

الگوریتمی با هدف تشخیص سطح تراکم در شبکه با استفاده از یادگیری ماشین و در ادامه اعمال نتیجه برای کنترل نرخ داده پیشنهاد شده است.
این الگوریتم از نرخ داده، قدرت سیگنال دریافتی و تعداد ارتباطات در دروازه به عنوان شاخص‌های یادگیری استفاده می‌کند.
در زمان تشخیص ازدحام، الگوریتم پیشنهادی به جای کاهش دادن نرخ داده، زمان عقب‌کشیدن نمایی را افزایش می‌دهد.
نتایج حاکی از افزایش کارایی شبکه و دقت در کنترل نرخ داده است. یادگیری در سرور شبکه صورت می‌گیرد و نتایج را به دستگاه‌هایی انتهایی می‌دهد.
محاسبات یادگیری ماشین روی یک سیستم متمرکز صورت می‌پذیرد. نقطه قوت این الگوریتم در نظر گرفتن سطح تراکم در شبکه و نقطه ضعف آن استفاده
از پیام‌های \متن‌لاتین{Downlink} است
\مرجع{Kufakunesu2020}.

الگوریتمی با هدف تخصیص بهینه فاکتور گسترش در راستای بهینه‌سازی تصادم‌ها و میرایی پیشنهاد شده است.
مدل پیشنهادی قصد دارد کیفیت گرههای انتهایی توزیع شده در شبکه را با انتصاب فاکتورهای گسترشی که کیفیت ارسال را بیشینه می‌کنند، بهینه کند
\مرجع{Kufakunesu2020}.

یکی از الگوریتم‌های پیشنهادی هدف کاهش زمان همگرایی در مکانیزم نرخ داده تطبیق‌پذیر برای گره و دروازه را دنبال می‌کند.
در ادامه این پژوهش با استفاده از شبیه‌سازی (به وسیله‌ی \متن‌لاتین{ns3}) نشان می‌دهد که با استفاده از رویش پیشنهادی این زمان همگرایی برای گره کاهش به سزایی دارد
\مرجع{Kufakunesu2020}.

الگوریتم پیشنهادی با ارائه روش‌های ابتکاری تلاش می‌کند تا فاکتورهای گسترش را میان گرههای در پوشش یک دروازه به گونه‌ای تقسیم کند
که با توجه به شبه‌عمود بودن این فاکتورهای گسترش بتوان بیشترین بهره از کانال رادیویی برد. نقطه ضعف این روش پشتیبانی آن از تنها یک دروازه
است
\مرجع{Kufakunesu2020}.
