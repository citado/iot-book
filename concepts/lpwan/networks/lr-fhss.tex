\زیرقسمت{\متن‌لاتین{LR-FHSS}}

یکی از تکنیک‌ها در شبکه‌های بی‌سیم استفاده از \متن‌لاتین{Frequency Hopping} است. در این تکنیک با هماهنگی در میان ارسال کننده و گیرنده فرکانس‌های ارسال در زمان
تغییر می‌کند. پیاده‌سازی این شیوه در شبکه‌های \متن‌لاتین{LoRaWAN} در قالب \متن‌لاتین{LR-FHSS} یا
\متن‌لاتین{Frequency Hopping Spread Spectrum}
صورت می‌پذیرد. در این روش هر کانال به تعدادی زیرکانال شکسته شده و سرآیند بسته روی همه این زیرکانال‌ها ارسال می‌شود.
خود داده اما قطعه قطعه شده و هر قطعه به وسیله‌ی یک زیرکانال ارسال می‌گردد.
از آنجایی که دروازه روی همه‌ی این کانال‌ها گوش می‌دهد می‌تواند بسته را دوباره بازسازی کند بنابراین
در دروازه نیازی به از پیش دانستن ترتیب این پرش‌ها یا فرکانس و پهنای باند دقیق این کانال‌ها ندارد
و این اطلاعات از طریق سرآیندهای بسته که به صورت تکراری ارسال می‌گردند قابل بازیابی است.
ذکر این نکته نیز خالی از لطف نیست که این ماژولیشن سربار بیشتری برای تشخیص سیگنال در گیرنده نسبت به \متن‌لاتین{LoRa} دارد
\مرجع{Boquet2021}.

\متن‌لاتین{LR-FHSS} در سال ۲۰۲۰ توسط \متن‌لاتین{Semtech} برای پشتیبانی از شبکه‌های وسیع پیشنهاد شده است.
ای ماژولیشن با \متن‌لاتین{LoRa} سازگار بوده و تنها برای \متن‌لاتین{uplink} پیشنهاد شده است و برای \متن‌لاتین{downlink} از همان
\متن‌لاتین{LoRa} استفاده خواهد شد
\مرجع{Boquet2021}.

پژوهش \مرجع{Boquet2021} با استفاده از شبیه‌سازی دست به ارزیابی این لایه فیزیکی زده است و یادآور می‌شود که هدف از ارائه این لایه فیزیکی
افزایش تعداد بسته‌های ارسالی توسط یک شی نیست بلکه افزایش کلی ظرفیت شبکه است. در شکل \رجوع{شکل: مقایسه LoRa و LR-FHSS} این افزایش ظرفیت شبکه بر پایه همین
شبیه‌سازی کاملا مشهود است. در شکل \رجوع{شکل: مقایسه LoRa و LR-FHSS} همچنین مشخص است که با افزایش تعداد اشیا در شبکه کارایی \متن‌لاتین{LR-FHSS} بیشتر از \متن‌لاتین{LoRa}
می‌گردد. از سوی دیگر این پژوهش بیان می‌کند در صورت افزایش حجم داده از ۱۰ بایت به ۵۰ بایت این بهبود بسیار زودتر به وقوع می‌پیوندد
\مرجع{Boquet2021}.

\شروع{شکل}
\درج‌تصویر[width=\textwidth]{./img/lr-fhss-limits.png}
\تنظیم‌ازوسط
\شرح{مقایسه دریافت صحیح اطلاعات با افزایش تعداد اشیا برای بسته‌های ۱۰ بیتی در \متن‌لاتین{LoRa} و \متن‌لاتین{LR-FHSS} \مرجع{Boquet2021}}
\برچسب{شکل: مقایسه LoRa و LR-FHSS}
\پایان{شکل}

از گپ‌های تحقیقاتی این حوزه می‌توان به مشخص کردن دنباله‌ی پرش‌های فرکانسی در \متن‌لاتین{LR-FHSS} و همزیستی \متن‌لاتین{LoRa} و \متن‌لاتین{LR-FHSS} اشاره کرد.
