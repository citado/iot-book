\زیرقسمت{\متن‌لاتین{LoRaWAN Mesh}}

همانطور که بیان شد شبکه‌های \متن‌لاتین{LoRaWAN} از دستگاه‌های انتهایی تشکیل شده است که داده را برای دروازه‌ها ارسال می‌کنند و یک همبندی ستاره تشکیل می‌دهند.
استاندارد تنها اجازه یک گام فاصله میان دروازه و دستگاه انتهایی را می‌دهد.
شبکه‌های چندگامی برای افزایش پوشش‌دهی و بهبود توان مصرفی شبکه‌های بی‌سیم،
با توجه به افزایش طول عمر باتری به دلیل توان ارسالی کمتر در مقایسه با شبکه‌های تک گامی،
شناخته شده هستند.
شبکه‌های چندگامی همچنین به بهبود گسترش‌پذیری، ظرفیت و قابلیت اطمینان نیز کمک می‌کنند
\مرجع{Cotrim2020}.

پژوهش‌های زیادی به ارائه‌ی راه‌کارهای چندگامی در شبکه‌های \متن‌لاتین{LoRaWAN} پرداخته‌اند که در آن‌ها بعضی از دستگاه‌ها به عنوان گره میانی برای سایر دستگاه‌ها ایفای نقش می‌کنند و
به این ترتیب پوشش شبکه را گسترش می‌دهند. این گره‌های میانی می‌توانند کارکردهای یک رله ساده و یا پروتکل‌های مسیریابی پیچیده را پیاده‌سازی کنند.
گره‌های رله بسته‌های دریافتی را به گره‌بعدی بدون نیاز به شکل دادن یک مسیر، باز ارسال می‌کنند.
با وجود اینکه این روش از مسیریابی ساده‌تر است، این روش می‌تواند برای شبکه‌های \متن‌لاتین{LoRaWAN} کاربردی باشد؛ چرا که پروتکل‌های مسیریابی می‌توانند شامل پیچیدگی‌های غیرضروری بوده
و منابع بیشتری از دستگاه‌ها مصرف کنند
\مرجع{Cotrim2020}.

پژوهش‌های این حوزه را می‌توان به صورت کلی در دسته‌های زیر قرار داد \مرجع{Cotrim2020}:

\شروع{فقرات}
\فقره بخش زیادی از پژوهش‌هایی که به معرفی شبکه‌های چندگامی پرداخته‌اند، استفاده از پروتکل‌های مسیریابی مبتنی بر پروتکل‌های موجود در شبکه‌های توری و موردی یا حتی راه‌کارهای جدید، را پیشنهاد می‌دهند.
\پایان{فقرات}
