\زیرقسمت{ابزارهای شبیه‌سازی \متن‌لاتین{LoRaWAN}}

شبیه‌سازی مدل‌سازی کامپیوتری یک راه‌حل مناسب برای بهبود تجربه
کاوش در کارایی سیستم و ارزیابی تکنیک‌های عملکردی آن در روش‌های مورد پیش‌بینی یا تصور است
\مرجع{Almuhaya2022}.

در حوزه \متن‌لاتین{LoRa} شبیه‌سازهای بسیار تخصصی و رایگان وجود دارند.
این شبیه‌سازهای \متن‌لاتین{LoRa} در این حوزه تحقیقاتی برای ارزیابی سناریوهای مختلف \متن‌لاتین{LoRa}
توسعه یافته و مورد استفاده قرار گرفته‌اند
\مرجع{Almuhaya2022}.

\زیرزیرقسمت{شبیه‌ساز \متن‌لاتین{LoRaSIM}}

\متن‌لاتین{LoRaSIM} بر پایه \متن‌لاتین{SimPy} به عنوان یک شبیه‌ساز رویداد گسسته به وسیله‌ی پایتون برای
شبیه‌سازی، تحقیق و آنالیز گسترش‌پذیری و کارکرد تصادم در شبکه‌های \متن‌لاتین{LoRaWAN} توسعه پیدا کرده است.
این شبیه‌ساز رابط گرافیکی نداشته اما می‌تواند نمودارها و اشکال را برای نمایش با سایر ابزارها خروجی دهد.
مدل انتشار رادیویی بر پایه مدل‌های شناخته‌شده \متن‌لاتین{path-loss} در مسیرهای طولانی در \متن‌لاتین{LoRaSIM}
پیاده‌سازی شده است
\مرجع{Almuhaya2022}.

\زیرزیرقسمت{شبیه‌ساز \متن‌لاتین{ns-3}}

\متن‌لاتین{ns-3} یک شبیه‌ساز شبکه رویداد گسسته است که در اواسط سال ۲۰۰۶
به زبان‌های \متن‌لاتین{C++} و \متن‌لاتین{python} ایجاد شده اما هنوز هم به شدت در حال توسعه است.
این شبیه‌ساز از گستره‌ای از پروتکل‌ها مانند \متن‌لاتین{WiFi}، \متن‌لاتین{LTE}، \متن‌لاتین{IEEE 802.15.4}،
\متن‌لاتین{Sigfox}، \متن‌لاتین{LoRa} و بسیاری شبکه‌های دیگر پشتیبانی می‌کند.
در \متن‌لاتین{ns-3} هم قابلیت شبیه‌سازی و هم قابلیت تقلید به واسطه \متن‌لاتین{socket}ها وجود دارد.
\متن‌لاتین{ns-3} می‌تواند با \متن‌لاتین{C++} خالص کار کرده یا برخی از ماژول‌های آن با \متن‌لاتین{python}
نوشته شوند.
در \متن‌لاتین{ns-3} می‌توان از \متن‌لاتین{pcap} برای جمع‌آوری بسته‌ها به منظور خطایابی استفاده کرد.
ماژول \متن‌لاتین{LoRaWAN} در \متن‌لاتین{ns-3} پیاده‌سازی شده است تا ابزاری قدرتمند را برای
شبیه‌سازی واقعی یک شبکه \متن‌لاتین{LoRaWAN} به جای شبیه‌سازی ساده شده لایه \متن‌لاتین{MAC} آن
فراهم آورد
\مرجع{Almuhaya2022}.

یک ماژول \متن‌لاتین{LoRaWAN} در \متن‌لاتین{ns-3} پیاده‌سازی و توسعه پیدا کرده است تا یک ابزار قدرتمند
برای شبیه‌سازی واقعی شبکه \متن‌لاتین{LoRaWAN} به جای شبیه‌سازی پروتکل ساده شده \متن‌لاتین{MAC} آن
فراهم آورد.
این ماژول به محققان و برنامه‌نویسان این فرصت را می‌دهد که بتوانند درک بهتری از رفتار لایه‌های \متن‌لاتین{MAC}
و فیزیکی در شبکه‌های \متن‌لاتین{LoRa} بدست آورند
\مرجع{Almuhaya2022}.

\زیرزیرقسمت{شبیه‌ساز \متن‌لاتین{FLoRa}}

شبیه‌ساز \متن‌لاتین{FLoRa} برای ارزیابی کارایی شبکه \متن‌لاتین{LoRa} به همراه مکانیزم‌های \متن‌لاتین{ADR} طراحی شده است.
شبیه‌ساز \متن‌لاتین{FLoRa} بر پایه شبیه‌ساز شبکه‌ای \متن‌لاتین{OMNeT++} و از اجزای سیستم \متن‌لاتین{INET} استفاده می‌کند.
کد \متن‌لاتین{FLoRa} با \متن‌لاتین{C++} نوشته شده است و اجازه توسعه شبکه‌های \متن‌لاتین{LoRa} با پشتیبانی از ادغام
گرههای \متن‌لاتین{LoRa}، دروازهها و ماژول‌های سرور شبکه را فراهم می‌آورد
\مرجع{Almuhaya2022}.

برای شبیه‌سازی شبکه \متن‌لاتین{LoRa}، پارامترهای زیادی می‌بایست انتخاب شود.
از جمله این پارامترها می‌توان به زمان شبیه‌سازی، بازه گرم کردن، فاکتور گسترش، توان ارسال برای هر یک از دستگاه‌های \متن‌لاتین{LoRa}،
تنظیمات شبکه \متن‌لاتین{backhaul} و لینک‌ها اشاره کرد
\مرجع{Almuhaya2022}.

\زیرزیرقسمت{شبیه‌ساز \متن‌لاتین{CupCarbon}}

شبیه‌ساز \متن‌لاتین{Carbon} یک شبیه‌ساز رو به رشد در حوزه شبیه‌سازی شهر هوشمند و شبکه‌های سنسور بی‌سیم اینترنت اشیا است.
هدف آن طراحی، تجسم، عیب‌یابی و صحت‌سنجی الگوریتم‌های توزیع‌شده برای مشاهدات محیطی و جمع‌آوری داده است.
شبیه‌ساز \متن‌لاتین{CupCarbon} از دو محیط شبیه‌سازی برای طراحی سناریوهای متحرک و شبیه‌سازی رویداد گسسته پشتیبانی می‌کند.
شبکه‌ها می‌توانند به واسطه رابط کاربری گرافیکی ارگونومیک و کاربرپسند \متن‌لاتین{CupCarbon} شبیه‌سازی شوند و حسگرها می‌توانند
به راحتی بر روی نقشه‌های برگرفته از سایت \متن‌لاتین{OpenStreetMap} جایگذاری شوند.
\متن‌لاتین{CupCarbon} از محاسبه توان مصرفی پشتیبانی کرده و می‌تواند آن را به عنوان تابعی از زمان شبیه‌سازی نمایش دهد.
این عملکردها اجازه می‌دهد ساختار شبکه شفاف شده و پیاده‌سازی واقعی پیش از استفرار واقعی صورت بپذیرد.
\متن‌لاتین{CupCarbon} از محیط‌های شبیه‌سازی چند عامل پشتیبانی کرده و اجازه می‌دهد در ضمن صورت پذیرفتن شبیه‌سازی رویدادها و
تنظیمات دنبال شوند.
این شبیه‌ساز اجازه تولید محیط‌های سه بعدی شامل ساختمان‌ها و طبقات را می‌دهد
\مرجع{Almuhaya2022}.

\زیرزیرقسمت{شبیه‌ساز \متن‌لاتین{PhySimulator}}

شبیه‌ساز \متن‌لاتین{PhySimulator} به عنوان ارزیاب لایه پیوند داده \متن‌لاتین{LoRa} استفاده می‌شود.
\متن‌لاتین{PhySimulator} در \متن‌لاتین{MATLAB} نوشته شده است.
این شبیه‌ساز نشان داده است که با وجود اینکه از تئوری فاکتورهای گسترش مختلف می‌توانند عمود بریکدیگر در نظر گرفته شوند
اما یکی از مسائل واقعی در شبکه‌های \متن‌لاتین{LoRa} تداخل بین فاکتورهای گسترش مختلف است
\مرجع{Almuhaya2022}.

\زیرزیرقسمت{جمع‌بندی}

تمام این شبیه‌سازها رویداد گسسته بوده، از پروتکل \متن‌لاتین{LoRaWAN} پشتیبانی می‌کنند و می‌توانند شبکه را در قالب
دنباله‌ای از رویدادهای گسسته در فضای زمان مدل کنند.
تمامی این شبیه‌ساز از زبان‌های برنامه‌نویسی شناخته شده استفاده می‌کنند که اجازه می‌دهد در صورت لزوم به سرعت بتوان
ماژول‌هایی را به عنوان افزونه در جهت گسترش توانایی‌های شبیه‌ساز پیاده‌سازی کرد
\مرجع{Almuhaya2022}.

به صورت خلاصه \متن‌لاتین{CupCarbon} با زبان \متن‌لاتین{Java}، \متن‌لاتین{FLoRa} با زبان \متن‌لاتین{C++}،
\متن‌لاتین{PhySimulator} با \متن‌لاتین{MATLAB}، \متن‌لاتین{LoRaSim} با زبان \متن‌لاتین{Python}
و \متن‌لاتین{ns-3} با زبان‌های \متن‌لاتین{C++} و \متن‌لاتین{Python} پیاده‌سازی شده‌اند
\مرجع{Almuhaya2022}.

شبیه‌ساز \متن‌لاتین{CupCarbon} می‌تواند محیط‌های سه بعدی و دو بعدی را شبیه‌سازی کند.
شبیه‌سازهای \متن‌لاتین{FLoLa} به واسطه \متن‌لاتین{OMNet++} و \متن‌لاتین{NS-3} به واسطه
\متن‌لاتین{NetAnim} و \متن‌لاتین{PyViz} رابط‌های کاربری گرافیکی گسترده‌ای را برای ماژول‌های \متن‌لاتین{Python}
و \متن‌لاتین{C++} فراهم می‌آورند.
در حالی که شبیه‌سازهای \متن‌لاتین{LoRaSim} و \متن‌لاتین{PhySimulator} تنها توانایی کشیدن نمودار دارند
\مرجع{Almuhaya2022}.

از بین این شبیه‌سازها، \متن‌لاتین{LoRaSim}، \متن‌لاتین{NS-3} و \متن‌لاتین{FLoRa} متن باز هستند.
\متن‌لاتین{CupCarbon} برای کاربردهای آموزشی رایگان بوده و \متن‌لاتین{PHY Simulator} رایگان است.
شبیه‌سازهای متن‌ باز روی \متن‌لاتین{Github} قرار دارند. از بین این شبیه‌سازها \متن‌لاتین{NS-3}
جامعه بزرگتری داشته و تحقیقات بیشتری از آن استفاده کرده‌اند
\مرجع{Almuhaya2022}.

\زیرقسمت{چالش‌های \متن‌لاتین{LoRa} و \متن‌لاتین{LoRaWAN}}

با وجود گسترش و استفاده روزافزون شبکه‌های \متن‌لاتین{LoRaWAN} هنوز چالش‌هایی در این شبکه وجود دارد. اولین چالش مربوط به تاثیر
\متن‌لاتین{Duty Cycle} بر اندازه شبکه است. با افزایش تعداد گرهها در شبکه نرخ پیام‌هایی که به درستی دریافت می‌شوند توسط تصادم و
\متن‌لاتین{Duty Cycle} محدود می‌شوند. همانطور که پیشتر بیان شد \متن‌لاتین{LoRaWAN} از \متن‌لاتین{ALOHA} استفاده می‌کند
که خود یکی از دلایل اصلی در ایجاد تصادم با افزایش تعداد گرهها در شبکه است. آنچه پژوهش \مرجع{Adelantado2017} در این حوزه
با شبیه‌سازی بر پایه سه کانال با بسته‌های ۱۰ بایتی محاسبه کرده است در شکل \رجوع{شکل: محدودیت‌های LoRaWAN} قابل مشاهده است
\مرجع{Adelantado2017}.

\شروع{شکل}
\درج‌تصویر[width=.5\textwidth]{./img/lora-limits-1}
\تنظیم‌ازوسط
\شرح{تاثیر افزایش اشیا و نرخ ارسال با در نظر گرفتن سه کانال بر بسته‌های دریافتی در پروتکل \متن‌لاتین{LoRaWAN} \مرجع{Adelantado2017}}
\برچسب{شکل: محدودیت‌های LoRaWAN}
\پایان{شکل}

دومین چالش شبکه‌های \متن‌لاتین{LoRaWAN} بحث قابلیت اطمینان است. قابلیت اطمینان در این شبکه‌ها توسط \متن‌لاتین{Acknowledgement} تامین می‌شود.
اما باز به دلیل وجود \متن‌لاتین{Duty Cycle} و تاثیر آن بر دروازه طراحی شبکه و برنامه‌های می‌بایست به گونه‌ای باشد که تعداد \متن‌لاتین{Acknowledgement}ها کمینه شوند.
همین مشکل استفاده از \متن‌لاتین{LoRaWAN} برای کاربردهایی که قابلیت اطمینان بسیار بالا می‌خواهند را زیر سوال می‌برد
\مرجع{Adelantado2017}.

اگر کاربردهایی مانند کشاورزی هوشمند یا کنترل محیط را در نظر بگیریم که داده‌های تولیدی به صورت دوره‌ای یا غیردوره‌ای بوده و محدودیت‌های سخت‌گیرانه‌ای از نگاه تاخیر ندارند و تنها نیاز به پوشش بالا وجود دارد،
استفاده از \متن‌لاتین{LoRaWAN} گزینه‌ی مناسبی است. در رابطه با کاربردهایی مانند اتوماسیون صنعتی که نیاز به تاخیر مشخص و محدود شده وجود دارد، استفاده از \متن‌لاتین{LoRaWAN} نیاز به در نظر گرفتن تمهیدات
خاصی است از جمله، کوچک نگاه داشتن فاکتور گسترش یا به عبارت دیگر نزدیکی اشیا به دروازه که باعث کاهش زمان ارسال و تاثیر \متن‌لاتین{Duty Cycle} می‌گردد و از سوی دیگر تعداد کانال‌ها می‌بایست
با دقت انتخاب شود تا با شکست ارسال در یک کانال بتوان بدون مشکل \متن‌لاتین{Duty Cycle} از کانال دیگری برای ارسال بهره برد
\مرجع{Adelantado2017}.
