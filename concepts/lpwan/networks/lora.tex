\زیرقسمت{\متن‌لاتین{LoRa}}

لایه‌ی فیزیکی \متن‌لاتین{LoRa} که در \متن‌لاتین{LoRaWAN} استفاده می‌شود، در سال ۲۰۱۴ توسط \متن‌لاتین{Semtech} به ثبت رسید
و بنابراین برای بررسی‌ها کاملا باز نیست. مطالبی که در ادامه می‌آید بخشی بر اساس قسمت‌های باز استاندارد و بخشی بر اساس آزمایش‌های
تجربی بدست آمده‌اند.
از ویژگی‌های \متن‌لاتین{LoRa} می‌توان به توان عملیاتی پایین، نرخ پایین داده و برد ارتباطی بالا اشاره کرد
\مرجع{sensors-18-03995}
\مرجع{Adelantado2017}.

از سال ۲۰۱۵ جامعه تحقیقاتی شروع به مطالعه در رابطه با کارآیی و ویژگی‌های مختلف فناوری‌های \متن‌لاتین{LoRa} و \متن‌لاتین{LoRaWAN} کرد.
از آن تاریخ مقلات متعددی در ژورنال‌ها و کنفرانس‌های عملی در سراسر دنیا چاپ و ارائه شده‌اند
\مرجع{sensors-18-03995}.

ماژولیشن \متن‌لاتین{LoRa} بر پایه \متن‌لاتین{Chirp Spread Spectrum} بوده و به صورت دوره‌ای سیگنال‌های \متن‌لاتین{chirp}ای تولید می‌کند که همه آن‌ها بازه زمانی یکسانی دارند.
\متن‌لاتین{chirp} یک سیگنال سینوسی است که فرکانس آن با زمان به صورت خطی افزایش یا کاهش پیدا می‌کند.
یک \متن‌لاتین{chirp} به وسیله‌ی پروفایل زمانی فرکانس لحظه‌ای آن که در بازه‌ی زمانی \متن‌لاتین{T} از فرکانس $f_0$ به فرکانس $f_1$
تغییر می‌کند، شناخته می‌شود.
در \متن‌لاتین{LoRa} دو نوع \متن‌لاتین{chirp} تعریف شده است. \متن‌لاتین{chirp} بالا که فرکانس پروفایل زمانی آن با فرکانس مینیمال
\(f_{\min} = -\frac{BW}{2}\)
شروع شده و با فرکانس ماکسیمال
\(f_{\max} = \frac{BW}{2}\)
خاتمه می‌یابد و \متن‌لاتین{chirp} پایین که فرکانس آن پروفایل زمانی آن با فرکانس ماکسیمال
شروع شده و با فرکانس مینیمال خاتمه می‌یابد.
برای ورودی‌های دیجیتال مختلف، یک ماژولاتور \متن‌لاتین{chirp}های مختلفی تولید می‌کند که نسبت به \متن‌لاتین{chirp} پایه شیف فرکانسی خورده‌اند
\مرجع{sensors-18-03995} \مرجع{Kufakunesu2020}.
برای هر فاکتور گسترش \متن‌لاتین{SF} به اندازه‌ی $2^{SF}$، \متن‌لاتین{chirp} وجود دارد که با فرکانس‌های مختلفی آغاز می‌شوند.
در ماژولیشن \متن‌لاتین{LoRa}، نرخ \متن‌لاتین{chirp} بر ثانیه برابر با پهنای باند است.

\شروع{شکل}
\درج‌تصویر[width=\textwidth]{./img/lora-mod.png}
\تنظیم‌ازوسط
\شرح{ماژولیشن \متن‌لاتین{LoRa}}
\پایان{شکل}

\شروع{شکل}
\درج‌تصویر[width=\textwidth]{./img/lora-chirp-sf.png}
\تنظیم‌ازوسط
\شرح{\متن‌لاتین{chirp}های پایه}
\پایان{شکل}

\متن‌لاتین{LoRa} از باند فرکانسی بدون مجوز استفاده می‌کند بنابراین برای راه‌اندازی شبکه‌ی آن نیاز به تهیه هیچ مجوزی نیست. البته باید در نظر داشته که نرخ پیام در این باندهای بدون مجوز توسط قانون‌گذاران محدود شده است.
یکی از محدودیت‌های مهم در شبکه‌های \متن‌لاتین{LoRa} محدودیت \متن‌لاتین{Duty Cycle} است که استفاده از کانال را محدود می‌کند. این محدودیت بیان می‌کند برای استفاده از کانال به مدت $T_{a}$ شی می‌بایست
حداقل به اندازه $T_{s}$ که از رابطه \رجوع{معادله: چرخه وظیفه} بدست می‌آید، ارسالی نداشته باشد
\مرجع{Cruz2021}
\مرجع{Adelantado2017}.

\begin{align}
  \label{معادله: چرخه وظیفه}
  T_{s} = T_{a}\left( \frac{1}{d} - 1 \right)
\end{align}

لایه فیزیکی \متن‌لاتین{LoRa} با توجه به ویژگی‌های گسترده‌ای که دارد در راهکارهای دیگری به جز \متن‌لاتین{LoRaWAN} نیز استفاده شده است که از جمله‌ی آن می‌توان به
شبکه‌های \متن‌لاتین{Meshed LoRa} اشاره کرد
\مرجع{Beltramelli2021}.

پارامترهای فاکتور گسترش یا به اختصار \متن‌لاتین{SF}، پهنای باند و نرخ‌کدگذاری قابل تنظیم هستند و می‌توانند روی زمان ارسال بسته، نرخ ارسال، مصرف انرژی و برد ارتباطی تاثیر داشته باشند.
در ادامه به مرور این پارامترها و تاثیرشان می‌پردازیم.

به صورت غیر رسمی فاکتور گسترش لگاریتم مبنای ۲ تعداد \متن‌لاتین{chirp}ها در هر علامت است. مقدار فاکتور گسترش بین ۷ تا ۱۲ است.
با افزایش فاکتور گسترش پوشش‌دهی بیشتر می‌شود اما بهای آن کاهش نرخ بیت و افزایش زمان ارسال\پانویس{Time on Air} است (معادله \رجوع{معادله: زمان علامت در LoRa})
\مرجع{Augustin2016}.

در بسته‌های \متن‌لاتین{LoRa} از تصحیح خطا جلورونده یا مختصرا \متن‌لاتین{FEC} استفاده می‌شود.
در این فرآیند بیت‌های تصحیح خطا به داده‌های ارسال اضافه می‌شوند.
این بیت‌های اضافه شده کمک می‌کنند تا داده‌های از دست رفته به خاطر تداخل بازگردانی شوند.
بیت‌های بیشتر این پروسه بازگردانی را ساده‌تر می‌کنند اما باعث هدر رفت پهنای باند و عمر باتری می‌شوند.
در \متن‌لاتین{LoRa} ما نرخ‌های کدگذاری $4/5$، $4/6$، $4/7$ و $4/8$ را داریم.

\begin{table}
\caption{توانایی \متن‌لاتین{LoRa} در تشخیص و تصحیح خطا \مرجع{Pham2020}}
\begin{latin}\begin{tabularx}
  {\textwidth}
  {|*{3}{X|}}
  \toprule
  Coding rates &
  Error detection (bits) &
  Error correction (bits) \\
  \midrule
  $4/5$ &
  0 &
  0 \\
  \midrule
  $4/6$ &
  1 &
  0 \\
  \midrule
  $4/7$ &
  2 &
  1 \\
  \midrule
  $4/8$ &
  3 &
  1 \\
  \bottomrule
\end{tabularx}\end{latin}
\end{table}

پهنای باند در \متن‌لاتین{LoRa} می‌تواند بین ۱۲۵، ۲۵۰ و ۵۰۰ کیلوهرتز باشد و با توجه به استفاده از باند بدون لایسنس این پهنای باند وابسته به پارامتر‌های منطقه‌ای و فاکتور گسترش است.
به طور مثال در باند فرکانسی ۸۶۸ مگاهرتز ۸ کانال متفاوت وجود دارد که ۷ کانال ابتدایی تنها با پهنای باند ۱۲۵ کیلوهرتز کار می‌کنند و کانال آخر می‌تواند با پهنای باند‌های
۱۲۵، ۲۵۰ و ۵۰۰ کیلوهرتز کار کند. از این بین ۳ کانال ۱۲۵ کیلوهرتزی اجباری بوده و می‌توان در صورت لزوم کانال‌های بیشتری را نیز فعال کرد.

ارسال‌هایی که روی یک کانال صورت می‌پذیرند اما فاکتورهای گسترش متفاوتی دارند می‌توانند تقریبا بدون تداخل ارسال شوند پس زوج کانال و فاکتور گسترش در \متن‌لاتین{LoRa} می‌تواند به عنوان
یک کانال مجازی شناخته شود. از آنجایی که این کانال‌های مجازی فاکتورهای گسترش متفاوتی دارند، نرخ بیت ارسالی روی همه آن‌ها یکسان نخواهد بود
\مرجع{Augustin2016}.

\شروع{شکل}
\درج‌تصویر[width=\textwidth]{./img/lora-868-channels.jpg}
\تنظیم‌ازوسط
\شرح{کانال‌های \متن‌لاتین{LoRa} در باند فرکانسی ۸۶۸ مگاهرتز}
\پایان{شکل}

در \متن‌لاتین{LoRa} نرخ باد یا نرخ علائم از رابطه‌ی \رجوع{معادله: نرخ باد یا علائم در LoRa}
و زمان یک علامت از رابطه‌ی \رجوع{معادله: زمان علامت در LoRa}
محاسبه می‌گردد.
که در آن‌ها \متن‌لاتین{BW} پهنای باند و \متن‌لاتین{SF} فاکتور گسترش است
\مرجع{Augustin2016}.

\begin{align}
  \label{معادله: نرخ باد یا علائم در LoRa}
  R_{s} = \frac{BW}{2^{SF}}
\end{align}

\begin{align}
  \label{معادله: زمان علامت در LoRa}
  T_{s} = \frac{2^{SF}}{BW}
\end{align}

در ادامه نرخ داده‌ی ارسالی را می‌توان با استفاده از رابطه \رجوع{معادله: نرخ داده در LoRa} محاسبه کرد.

\begin{align}
  \label{معادله: نرخ داده در LoRa}
  R_{b} = SF \times \frac{BW}{2^{SF}} \times CR
\end{align}

در رابطه \رجوع{معادله: نرخ داده در LoRa} \متن‌لاتین{CR} نرخ کدگذاری، \متن‌لاتین{SF} فاکتور گسترش و \متن‌لاتین{BW} پهنای باند است
\مرجع{Augustin2016}.

\شروع{شکل}
\درج‌تصویر[width=.5\textwidth]{./img/lora-packet.png}
\تنظیم‌ازوسط
\شرح{ساختار بسته \متن‌لاتین{LoRa} \مرجع{Augustin2016}}
\برچسب{شکل: بسته LoRa}
\پایان{شکل}

رابطه \رجوع{معادله: تعداد علائم مورد نیاز در LoRa} مشخص می‌کند برای ارسال یک داده به چه تعداد علامت نیاز داریم. این پارامتر با $n_{s}$ نمایش داده می‌شود.

\begin{align}
  \label{معادله: تعداد علائم مورد نیاز در LoRa}
  n_{s} = 8 + \max\left( \left\lceil \frac{8PL - 4SF + 8 + CRC + H}{4 \times (SF - DE)} \right\rceil \times \frac{4}{CR}, 0 \right)
\end{align}

در رابطه \رجوع{معادله: تعداد علائم مورد نیاز در LoRa} در صورت فعال بودن \متن‌لاتین{CRC} مقدار آن برابر ۱۶ و در غیر این صورت برابر صفر است.
\متن‌لاتین{CR} نرخ کدگذاری،
\متن‌لاتین{PL} اندازه داده،
\متن‌لاتین{SF} فاکتور گسترش است.
در این رابطه \متن‌لاتین{H} اندازه سرآیند بوده که در صورت فعال بودن برابر ۲۰ و در غیر این صورت صفر است.
در این رابطه \متن‌لاتین{DE} در صورت فعال بودن حالت نرخ داده پایین یا \متن‌لاتین{low data rate} برابر ۲ و در غیر این صورت برابر صفر است
\مرجع{Augustin2016}
\مرجع{Pham2020}.

همانطور که محاسبات دیده می‌شود، استفاده از مقدارهای بالاتر برای فاکتور گسترش زمان ارسال را بیشتر کرده و تاثیر \متن‌لاتین{Duty Cycle} را بیشتر می‌کند.
با استفاده از فاکتورهای گشترش بالاتر نرخ ارسال کاهش پیدا کرده اما قابلیت اطمینان در مسافت‌های بالاتر بیشتر می‌شود.
در فاکتورهای گسترش پایین مسافت‌های کمتری قابل دسترس بوده اما زمان ارسال پایین‌تر می‌آید. در این فاکتورهای گسترش
امکان دست‌یابی به نرخ داده بالاتر وجود دارد اما دروازه می‌بایست حساسیت بالاتری داشته باشد.
پژوهش \مرجع{Adelantado2017} بیان می‌کند احتمال استفاده از فاکتورهای گسترش بالاتر بیشتر است.

ساختار فریم \متن‌لاتین{LoRa} در شکل \رجوع{شکل: بسته LoRa} قابل مشاهده است. استفاده از سرآیند اختیاری بوده است و در صورتی که مواردی مانند اندازه بسته،
نرخ کدگذاری و وجود \متن‌لاتین{CRC} از پیش هماهنگ شده باشند نیازی به استفاده از آن نیست. سرآیند از نرخ کدگذاری $4/8$ استفاده کرده و دارای یک \متن‌لاتین{CRC}
برای خود است.

یکی دیگر از پارامترهای قابل تنظیم در \متن‌لاتین{LoRa}، توان ارسال است. توان ارسال می‌تواند در بازه $-4dBm$ تا $20dBm$
باشد اما این مقادیر با محدودیت‌های سخت‌افزاری و قانونی محدود می‌شوند
\مرجع{Kufakunesu2020}.

\شروع{لوح}
\تنظیم‌ازوسط
\شروع{لاتین}\شروع{جدول}{ccccc}
\toprule
Data Rate & Spreading Factor & Bandwidth[KHz] & Bit rate[kbps] & Sensivity [dBm] \\
\midrule
DR0 & 12 & 125 & 0.293 & -137 \\
DR1 & 11 & 125 & 0.537 & -134.5 \\
DR2 & 10 & 125 & 0.976 & -132 \\
DR3 & 9  & 125 & 1.757 & -129 \\
DR4 & 8  & 125 & 3.125 & -126 \\
DR5 & 7  & 125 & 5.4680 & -123 \\
DR6 & 7  & 250 & 10.936 & -122 \\
\bottomrule
\پایان{جدول}\پایان{لاتین}
\شرح{نرخ داده و حساسیت \متن‌لاتین{LoRa} بر پایه پارامترهای متفاوت برای باند فرکانسی ۸۶۸ مگاهرتز \مرجع{ElChall2019}}
\پایان{لوح}
